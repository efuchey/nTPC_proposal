%version 1.0 June 1, 2020
{\hskip 2. in {\large {\bf Abstract}}} \\


We propose to make a high precision measurement of the two-photon exchange contribution (TPE) in elastic positron-neutron scattering at two four-momentum transfers $Q^2$ of 3.0~and 4.5~$GeV^2$. This measurement purports to complete and extend the measurement of the two-photon exchange in electron-neutron scattering submitted to and approved by PAC48 in 2020, and recorded in 2022 (experiment E12-20-010, currently under analysis), which was the first experiment to examine the impact of the TPE in the neutron form factors experimentally. This program means to address the open question of the discrepancy between GE /GM ratios measured in elastic electron-nucleon scattering via Rosenbluth separation on the one-hand and polarization transfer on the other hand, with the former known to be sensitive to the TPE contribution while the latter isn’t.
The comparison between the Rosenbluth slope in the positron-neutron measured by the proposed experiment and the Rosenbluth slope in electron-neutron measured by E12-20-010 will provide significant insight on the TPE contribution in the neutron form factors, completing the efforts of MUSE at PSI and the proposed E12+23-008 to measure the TPE contribution on the proton form factors.

The proposed experiment shall be performed in Hall A and will measure simultaneous positron-proton and positron-neutron scattering off deuterium, extracting the Rosenbluth slope of positron-neutron quasi-elastic scattering at two beam energies of 3.3~and 4.4~GeV for $Q^2$~=~3.0~$GeV^2$, and 4.4~and 6.6 GeV for $Q^2$~=~4.5~$GeV^2$. In the proposed approach, systematic errors for positron neutron scattering are greatly reduced compared to those in the traditional single arm configuration.
The experimental setup of the proposed experiment will be identical to that of the E12-20-010 experiment, using the BigBite (BB) spectrometer to detect the scattered positrons and the Super BigBite Spectrometer (SBS) to detect the protons and neutrons, combined with the proposed positron beam-upgrade for CEBAF. Using the maximum proposed intensity of 1 $\mu$A unpolarized positrons on 15 cm cryogenic deuterium target, this measurement requires six days on both kinematics.
In addition, the measurement requires two extra days with electron beams and 15 cm cryogenic hydrogen target (with 10 μA intensity) for calibrations and nucleon detection efficiency measurements, plus two extra days for kinematic change. The analysis of the proposed experiment will greatly benefit on the return of experience of the ongoing analysis of E12-20-010.


\iffalse
We propose make a high precision measurement of the two-photon exchange contribution (TPE) in elastic electron-neutron scattering at a four-momentum transfer \qsq~= 4.5 \gevcsq. 
While significant efforts to study the two-photon-exchange have focused around elastic electron-proton scattering, the impact of TPE on neutron form factors was never examined experimentally. 
The proposed experiment will provide the very first assessment of the two-photon exchange in electron-neutron scattering, which will be important for understanding of the nucleon form factor physics. \par
The proposed experiment will be performed in Hall A using the BigBite (BB) spectrometer to detect the scattered electrons and the Super-BigBite (SBS) to detect the protons and neutrons. 
The experiment should run concurrently with the E12-09-019 $G_M^n$ and E12-17-004 $G_E^n$-Recoil experiments, which are expected to run in 2021. 
The experimental setup of the proposed experiment will be identical to that of E12-09-019 experiment. \par
The ``ratio" method will be used to extract the electric form factor of the neutron \gen~by scattering unpolarized electrons from deuterium quasi-elastically at two beam energies 4.4 and 6.6 GeV and electron scattering angles 41.9 and 23.3 degrees respectively. 
In the proposed approach, systematic errors are greatly reduced compared to those in the traditional single electron arm configuration. 
Several experiments at Mainz and JLab have used the ratio method to measure the neutron magnetic form factor in the past years. 
The method can be extended to extract the neutron electric form factor even with less stringent requirements 
on the knowledge of the absolute neutron detection efficiency and experimental kinematics.  
\fi
% We propose to add a kinematic point at $Q^2$ = 4.5 (GeV/c)$^2$ with high $\epsilon$ value, which will be used along with one of E12-09-019 kinematic points to extract the form factors ratio $g=G_E/G_M$ using Rosenbluth method. 
