%v1.9
\section{Beam time request}

{\bf We request 48 hours total time (32 PAC hours of beam-on target)} to measure the two-photon effect (and \gen~in one-photon approximation) 
at \qsq~= 4.5 \gevcsq~through a measurement of the cross sections of the reaction D(e,e'N) at a large value of the virtual photon polarization $\epsilon$=0.84.
{\em The measurement at \qsq~= 4.5 \gevcsq,~$\epsilon$=0.60 is already scheduled as part of the SBS $G_M^n$ experiment E12-09-019}~\cite{E12-09-019}.

We plan to take 12 hours of data at a full luminosity of $2.86~\times~10^{38}~\mathrm{cm}^{-2}\mathrm{s}^{-1}$, which corresponds to a beam intensity of $I_{exp} =~30~\mu$A on a liquid deuterium target with length $l_{tgt}~=~15$~cm and density $d_{tgt}~=~0.169~\mathrm{g.cm}^{-3}$. 
To have a better handle on our backgrounds, we also plan to take 12 hours of data at half luminosity (basically by lowering the beam intensity by a factor 2).
In each of these configurations, we also need to take data on a ``dummy'' target ({\it i.e.} on a target cell identical to the one used for production, but empty) to understand the contamination of our data from the target walls.

In addition to this beam time, we also require 16 hours (two shifts) to change the experimental configuration.
This configuration change means:
%
\begin{itemize}
%\item{SBS magnet and the hadronic calorimeter (HCal) angle change;}
\item{BigBite spectrometer angle and distance change;}
\item{Beam pass change (from 4.4 GeV/2~pass to 6.6 GeV/3~pass);}
\end{itemize}
%
If this experiment is approved, we plan to insert this experiment in the GMn E12-09-019 run plan in such a way that we can avoid a change in configuration for the SBS magnet and the HCal, which is very time consuming. Table.~\ref{tab:gmnplan} displays an excerpt of the GMn run plan, and points out where the nTPE measurement inserts. 
%
\begin{table}[h]
\caption{Excerpt of GMn E12-09-019 run plan table, showing where the $\epsilon = 0.84$ measurement would be inserted in our run plan. The $\epsilon = 0.60$ measurement is also emphasized in blue.}
\label{tab:gmnplan}
\centering
\begin{tabular}{|l|c|c|c|c|c|c|}
\hline
Step \# & task & \qsq & $\theta_{BB}$~/~$\theta_{SBS}$ & Beam & Time & Tech work \\
&  & \gevcsq & degrees & GeV & hours & time (h) \\
\hline
%1 & beam line &  & 41.9 / 24.7 & 4.4 & 24 &  \\
%2 & BigBite &  & 41.9 / 24.7 & 4.4 & 48 &  \\
%2 & HCal at 14 m &  & 41.9 / 24.7 & 4.4 & 48 &  \\
%\hline
\hline
%4a (move 14 m to 8.5 m) & HCal &  & 41.9 / 24.7 & - & 4 & 4 \\
4b (install GEn-RP) & GEn-RP &  & 41.9 / 24.7 & - & 4 & 4 \\
4c (GEn-RP) & Production & 4.5 & 41.9 / 24.7 & 4.4 & 104 & \\
4d (remove GEn-RP) & GEn-RP &  & 41.9 / 24.7 & - & 56 & 24 \\
\hline
\textcolor{blue}{4e (GMn/nTPE low $\epsilon$)} & \textcolor{blue}{Production} & \textcolor{blue}{4.5} & \textcolor{blue}{41.9 / 24.7} & \textcolor{blue}{4.4} & \textcolor{blue}{64 (calendar)} & \\
&  &  &  &  & \textcolor{blue}{(32 PAC hours)} &  \\
\hline
\hline
5a (conf. change) & BB/SBS/HCal &  & 32.5 / 31.2 & - & 32 & 16 \\
5b (beam tune)  & beam &  & 32.5 / 31.2 & 4.4 & 4 &  \\
5c (GMn) & Production & 3.5 & 32.5 / 31.2 & 4.4 & 64 (calender) &  \\
 &  &  &  &  & (32 PAC hours) &  \\
\hline
\hline
6a (pass change) & beam/BB &  & 23.2 / 31.2 & 6.6 & 8 & 4 \\
6b (beam tune) & beam &  & 23.2 / 31.2 & 6.6 & 8 &  \\
\textcolor{red}{6c see Table.~X} & \textcolor{red}{Production} & \textcolor{red}{4.5} & \textcolor{red}{23.2 / 31.2} & \textcolor{red}{6.6} & \textcolor{red}{64 (calendar)} & \\
&  &  &  &  & \textcolor{red}{(32 PAC hours)} &  \\
\hline
\hline
7a (conf. change) & BB/SBS/HCal &  & 30.5 / 24.7 & 6.7 & 32 & 16 \\
7b (beam tune) & beam & - & 30.5 / 24.7 & 6.6 & 4 &  \\
7c  & Production & 6.1 & 30.5 / 24.7 & 6.6 & 50 (calendar) &  \\
 &  &  &  &  & (25 PAC hours) &  \\
%\hline
\hline
\end{tabular} 
\end{table}
%
The beam pass change and the BigBite move may be done in parallel, and should take one shift (eight hours). We require an additional shift (eight hours) for beam tuning after beam pass change. 
The projected use of this time is summarized in Table.~XI.%\ref{tab:beamtime}.
%
\begin{table}[h]
\caption{Summary table for the beam time request. Setting changes include SBS and BigBite angles change, as well as a beam pass change from 4.4~GeV (2~pass) to 6.6~GeV (3~pass). This beam pass change can mostly be done in parallel to the SBS}
\label{tab:beamtime}
\centering
\begin{tabular}{|l|c|c|c|}
\hline
Task & Target & $I_{exp}$ & time requested \\
\hline
Data taking (Prod.) & 15~cm LD$_2$ & $30~\mu$A & 12 hours \\ 
\hline
Data taking (Syst.) & 15~cm ``Dummy'' & $30~\mu$A & 4 hours \\ 
\hline
Data taking (Prod.) & 15~cm LD$_2$ & $15~\mu$A & 12 hours \\ 
\hline
Data taking (Syst.) & 15~cm ``Dummy'' & $15~\mu$A & 4 hours \\ 
\hline
\multicolumn{3}{|l|}{Setting changes (BigBite move, beam pass change)} & 8 hours \\
\multicolumn{3}{|l|}{Beam tune after beam pass change} & 8 hours \\
%\multicolumn{3}{|l|}{SBS/HCal/BigBite movement} & 32 hours \\
\hline
\hline
\multicolumn{3}{|l|}{\bf Total} & {\bf 48 hours} \\ 
\hline
\end{tabular} 
\end{table}

This experiment will take place in Hall A, along the already scheduled SBS \gmn experiment E12-09-019, utilizing the BigBite spectrometer to detect electrons scattered off 
the liquid deuterium target, and HCal calorimeter to detect the recoiling neutron and proton.

Data taking (if approved by PAC48) will take place in summer 2021 during the approved and scheduled run of the GMn, E12-09-019, experiment,
which is going to measure the $e-n$ elastic scattering cross section at \qsq~= 4.5 \gevcsq~at $\epsilon$=0.60.

The set of instrumentation and required beam current for proposed measurement is identical to one in the GMn experiment.
The beam energy of 6.6 GeV will be used.
One of two data points required for the cross section LT separation is already in the data taking plan of GMn.

There are no other measurements of TPE in the $e-n$ elastic scattering and knowledge of the TPE is essential for the understanding 
of the elastic electron scattering from neutron (and proton) and hadron structure.  
Furthermore, it is a necessary input in the analysis and interpretation of a wide range of electron scattering processes. 

The kinematics of our measurements emphasize the same \qsq~range where TPE in $e-p$ elastic scattering was observed to dominate in Rosenbluth slope.
Measuring at this high momentum transfers will provide unique input for testing TPE calculations~\cite{Blunden:2005ew}.

We propose to measure the Rosenbluth slope and extract (in one-photon approximation) $\delta$\gen/\gmn~to an accuracy of 0.15, which would bring its precision to a level comparable with that of the double polarization experiments GEN-RP and GEN-He3 at such value of \qsq.
Such precision should be sufficient to detect the TPE contribution to the $e-n$ Rosenbluth slope on the three sigma level.
