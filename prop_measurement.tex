\section{Proposed Measurements} 
\label{prop}

We propose to use the same experimental setup of the past E12-09-019/E12-20-010 experiments.
We have two kinematic points at \qsq~= 4.5 \gevcsq~at two beam passes (4.4~GeV/2~pass and 6.6~GeV/3~pass), and two kinematic points at \qsq~= 3.0 \gevcsq~at two beam passes (3.3~GeV/1.5~pass and 4.4~GeV/2~pass), obtaining two $\epsilon$ values for each \qsq~value.
Each of these kinematics will be run with unpolarized positron beams at the maximum intensity available for Hall~C, {\it i.e.} 1~$\mu$A. 
Note that the two kinematic points at \qsq~= 3.0 \gevcsq~will also be run with electronsm at an intensity of 10~$\mu$A.
(the two kinematic points at \qsq~= 4.5 \gevcsq~will only need to be run for positrons as the past E12-20-010 experiment already took those measurements for electron).
This will allow us to perform the standard Rosenbluth method to obtain (in one-photon approximation) the neutron electric and magnetic form factors.
In addition, the ratio method (Sec.~\ref{sec:exp_method}), in which the systematic errors are greatly reduced, will be implemented to calculate the two photon exchange (TPE) contribution. The study of the $\epsilon$ dependence of the reduced cross section will help examine the two photon exchange contribution to the neutron form factor ratio \gen/\gmn.
%The  additional point along with the data point of the E12-09-019 experiment will allow us to perform the standard Rosenbluth method to obtain (in one-photon approximation) the neutron electric and magnetic form factors. In addition, the ratio m1ethod (Sec.\ref{sec:exp_method})
Table.~I %~\ref{tab:propkin}
displays the kinematic settings of the proposed experiment.

\begin{table}[!h] 
\centering
\begin{tabular}{|c|c|c|c|c|c|c|c|}
\hline
Kinematic & $e^{+}/e^{-}$ - I$_{beam}$ & $Q\textsuperscript{2}$  & E & E$'$  & $\theta_{BB}$ & $\theta_{SBS}$ & $\epsilon$ \\
& ($mu$A) & (GeV/c)$^2$ & (GeV) & (GeV)  & degrees &  degrees   &   \\
\hline
{\bf1+} & $e^{+}$ (1.0) & 4.5 & 4.4 & 2.0 & 41.9 & 24.7 & 0.600 \\
\hline
{\bf2+} & $e^{+}$ (1.0) & 4.5 & 6.6 & 4.2 & 23.3 & 31.2 & 0.838 \\
\hline
{\bf3+} & $e^{+}$ (1.0) & 3.0 & 3.3 & 1.7 & 42.8 & 29.5 & 0.638 \\
\hline
{\bf3-} & $e^{-}$ (10.0) & 3.0 & 3.3 & 1.7 & 42.8 & 29.5 & 0.638 \\
\hline
{\bf4+} & $e^{+}$ (1.0) & 3.0 & 4.4 & 2.8 & 28.5 & 31.2 & 0.808 \\
\hline
{\bf4-} & $e^{-}$ (10.0) & 3.0 & 4.4 & 2.8 & 28.5 & 31.2 & 0.808 \\
\hline
\end{tabular} 
\caption{Kinematic settings of the proposed experiment.}
\label{tab:propkin}
\end{table}

%Q2 = 3 GeV2, Low energy: d(e, e'n) rate = 5.09738 Hz; d(e, e'p) rate =  14.8865 Hz 
%Q2 = 3 GeV2, High energy: d(e, e'n) rate = 8.00224 Hz; d(e, e'p) rate =  23.3301 Hz 
%Q2 = 4.5 GeV2, Low energy: d(e, e'n) rate = 0.972556 Hz; d(e, e'p) rate =  3.08244 Hz 
%Q2 = 4.5 GeV2, High energy: d(e, e'n) rate = 1.88723 Hz; d(e, e'p) rate =  6.22764 Hz 
