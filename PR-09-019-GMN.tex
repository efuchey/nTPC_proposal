%\input{../../33331/hwk/preamble.tex}


%\documentclass[12pt,letterpaper,oneside,draft]{report}
%\documentclass[12pt,letterpaper,oneside,draft]{article} % no actual figures!!
\documentclass[12pt,letterpaper,oneside]{article}
\pagestyle{plain}
\setlength{\oddsidemargin}{-.25in}    % LaTeX add one inch to this
\setlength{\evensidemargin}{-.25in}   % LaTeX add one inch to this
\setlength{\topmargin}{0.in}        % LaTeX add one inch to this
\setlength{\footskip}{0.5in}
\setlength{\textheight}{9.in}
\setlength{\textwidth}{7.0in}
\setlength{\topskip}{0.0in}
\setlength{\headsep}{0.0in}
\setlength{\headheight}{0.0in}
\setlength{\marginparwidth}{1.2in}

\usepackage{epsfig}
%\usepackage{graphX}
\usepackage{wrapfig}
\usepackage{subfigure}
\usepackage{rotating}
\usepackage{amsmath}
\usepackage{cite}


%%\setcounter{secnumdepth}{3}

%\reversemarginpar
\begin{document}
%\baselineskip 24 pt
%\parskip 24 pt
%\baselineskip 18 pt
%\parskip 18 pt
%\baselineskip 18 pt
%\parskip 18 pt

\newcommand{\mMeV} {\,\mbox{MeV}/\mbox{c}^{2}}
\newcommand{\pMeV} {\,\mbox{MeV}/\mbox{c}}
\newcommand{\eMeV} {\,\mbox{MeV}}
\newcommand{\mGeV} {\,\mbox{GeV}/\mbox{c}^{2}}
\newcommand{\pGeV} {\,\mbox{GeV}/\mbox{c}}
\newcommand{\eGeV} {\,\mbox{GeV}}
%\newcommand{\q2} {Q$^2$}
\newcommand{\gmn} {G$^{\mbox{\scriptsize n}}_{_{\mbox{\tiny M}}}~$}
\newcommand{\gmnc} {G$^{\mbox{\scriptsize n}}_{_{\mbox{\tiny M}}}$}
\newcommand{\gen} {G$^{\mbox{\scriptsize n}}_{_{\mbox{\tiny E}}}~$}
\newcommand{\genc} {G$^{\mbox{\scriptsize n}}_{_{\mbox{\tiny E}}}$}
\newcommand{\gmp} {G$^{\mbox{\scriptsize p}}_{_{\mbox{\tiny M}}}~$}
\newcommand{\gmpc} {G$^{\mbox{\scriptsize p}}_{_{\mbox{\tiny M}}}$}
\newcommand{\gep} {G$^{\mbox{\scriptsize p}}_{_{\mbox{\tiny E}}}~$}
\newcommand{\gepc} {G$^{\mbox{\scriptsize p}}_{_{\mbox{\tiny E}}}$}
%\newcommand{\gmn} {G$^{\mbox{n}}_M~$}


\title{A new proposal to the Jefferson Lab Program Advisory Committee
  (PAC34)\\
{\bf Precision Measurement of the\\
 Neutron Magnetic Form Factor\\
 up to $\mathbf Q^2=18.0\mbox{ \bf (GeV/c)}^2$ by the Ratio Method}}



\vspace{0.1in}

\author{ 
%$\,$ \hskip 1.in G.B.~Franklin, B.~Quinn (spokesperson),\\
F.~Benmokhtar, G.B.~Franklin, B.~Quinn (spokesperson),
R.~Schumacher %\hskip 1.in $\,$
\\ {\it Carnegie Mellon University, Pittsburgh, PA 15213 } \\
%
\and A.~Camsonne, J.P.~Chen, E.~Chudakov, C.~DeJager,\\
 P.~Degtyarenko, J.~Gomez, O.~Hansen, D.~W.~Higinbotham,\\ 
M.~Jones, J.~LeRose, R.~Michaels,  S.~Nanda, A.~Saha, V.~Sulkosky,\\
 B.~Wojtsekhowski (spokesperson and contact person) 
\\{\it Thomas Jefferson National Accelerator Facility, Newport News, VA 23606}
%
\and L.~El Fassi, R.~Gilman (spokesperson), G.~Kumbartzki, R.~Ransome, E.~Schulte
\\ {\it Rutgers University, Piscataway, NJ  08854}
%
\and T. Averett, C.F.~Perdrisat, L.P.~Pentchev
\\ {\it College of William and Mary}
%
\and E.~Cisbani, F.~Cusanno, F.~Garibaldi, \\ 
S.~Frullani, G.~M.~Urciuoli, M.~Iodice, M.L.~Magliozzi
\\ {\it INFN, Rome, Italy}
%
\and H.~Baghdasaryan, G.~Cates, D.~Day, N.~Kalantarians, R.~Lindgren, N.~Liyanage, \\
V.~Nelyubin, B.~E.~Norum, K.~D. Paschke, S.~Riordan, M.~H.~Shabestari, X.~Zheng
\\ {\it University of Virginia, Charlottesville, VA 22901}
%
\and W.~Brooks\\
{\it Universidad Tecnica Federico Santa Maria, Valparaiso, Chile}
%
\and V.~Punjabi, M.~Khandaker \\
{\it Norfolk State University}
%
\and W.~Boeglin, P.~Markowitz, J.~Reinhold \\
{\it Florida International University, Fl }
%
\and J.~Annand, D.~Hamilton, D.~Ireland, R.~Kaiser, K.~Livingston, \\
I.~MacGregor, B.~Seitz, and G.~Rosner \\
{\it University of Glasgow, Glasgow, Scotland}
%
\and D.~Nikolenko, I.~Rachek, Yu.~Shestakov
\\ {\it Budker Institute, Novosibirsk, Russia}  \\
%
\and $\,$  R.~De~Leo, L.~La~Gamba, S.~Marrone, E.~Nappi $\,$
\\ {\it INFN, Bari, Italy}   \\
%
\and $\,$ M.~Mihovilovi\v{c}, M.~Potokar, S.~\v{S}irca $\,$
\\ {\it Jo\v{z}ef Stefan Institute and Dept. of Physics, University of Ljubljana, Slovenia}   \\
%
%\and J.~Lachniet
%\\ {\it Old Dominion University, Norfolk, VA}
%
\and J.~Gilfoyle
\\ {\it University of Richmond, Richmond, VA}
%
%\and $\,$ \hskip 1.in E.~Piasetzky, G.~Ron \hskip 1.in $\,$
\and $\,$ J.~Lichtenstadt, I.~Pomerantz, E.~Piasetzky, G.~Ron \hskip 1.in $\,$
\\ {\it Tel Aviv University, Israel} \\
%
\and $\,$ \hskip 1.in A.~Glamazdin \hskip 1.in $\,$
\\ {\it Kharkov Institute of Physics and  Technology, Kharkov 310077, Ukraine}
%
\and $\,$ \hskip 1.in J.~Calarco, K.~Slifer \hskip 1.in $\,$
\\ {\it University of New Hampshire, Durham, NH 03824}
%
\and $\,$ \hskip 1.in B.~Vlahovic \hskip 1.in $\,$
\\ {\it North Carolina Central University, Durham, NC 03824}
%
\and $\,$ \hskip 1.in A.~Sarty \hskip 1.in $\,$
\\ {\it Saint Mary's University, Nova Scotia, Canada B3H 3C3}
%
%\and $\,$ \hskip 1.in Piotr Decowski \hskip 1.in $\,$
%\\ {\it Smith College, Northampton, MA 01063 }
%
\and $\,$ \hskip 1.in K.~Aniol and D.~J.~Magaziotis \hskip 1.in $\,$
\\ {\it Cal State University, Los Angeles, CA 90032} 
%
%\and S.~Abrahamyan, A.~Ketikyan, S.~Mayilyan, A.~Shahinyan, H.~Voskanyan
\and S.~Abrahamyan, S.~Mayilyan, A.~Shahinyan, H.~Voskanyan
\\ {\it Yerevan Physics Institute, Yerevan, Armenia}
%
\and $\,$ \hskip 1.in D.~Glazier and D.~Watts \hskip 1.in $\,$
\\ {\it University of Edinburgh, Edinburgh, Scotland} 
%
\and $\,$ \hskip 1.in B.~Sawatzky \hskip 1.in $\,$
\\ {\it Temple University, Philadelphia, PA 19122} 
%
\and $\,$ \hskip 1.in B.~Bertozzi and S.~Gilad \hskip 1.in $\,$
\\ {\it Massachusetts Institute of Technology, Cambridge, MA 02139} 
%
\and and\\
{\bf The Hall A Collaboration} \\% * - to be confirmed
\\
}
\maketitle


\newpage
\begin{abstract}
We propose to make a high-precision measurement of the
  neutron's magnetic form factor, \gmnc, at nine kinematic points:
  $Q^2=3.5$, 4.5, 6.5, 8.5, 10., 12., 13.5, 16. and 18.0 (GeV/c)$^2$.  Little data on \gmn
  exists in this kinematic range and the existing data have large
  systematic uncertainties.
In the proposed experiment, systematic errors are 
greatly reduced by the use of the ``ratio''
  method in which \gmn is extracted from the ratio of
  neutron-coincident to proton-coincident quasi-elastic electron
  scattering from the deuteron.
The experiment would be performed in  Hall A using the BigBite
  spectrometer to detect the scattered electrons and the BigHAND to
  detect both neutrons and protons.  A large aperture dipole magnet
  on the nucleon flight path will greatly enhance particle
  identification by slightly deflecting the protons.  Calibration reactions
  will be used to produce tagged neutrons and protons at essentially
  the same momentum as the quasi-elastic nucleons.  The efficiency of
  BigHAND is expected to be high and stable and to be well
  calibrated.  Projected systematic errors on the
  measured ratio of cross sections vary from under 2\% to 5\% 
(1.\% to 2.5\% on the ratio of the magnetic form
  factor to that  of the proton).   
  Statistical errors are projected to be similar, or smaller. 
  This proposal overlaps and significantly extends the kinematics range of the
  already approved 12 GeV CLAS proposal, E12-07-104, which predicts similar
  uncertainties up to 13.5 (GeV/$c$)$^2$.  The present proposal
  includes measurements at Q$^2$ beyond the range of that experiment 
and  will have significantly
  greater statistics for the highest-Q$^2$ points covered by E12-07-104.
 We request a total of 49 days
  divided among four beam energies, 4.4, 6.6, 8.8 and 11. GeV.
\newpage
\tableofcontents
\newpage

\end{abstract}







\section{Introduction}

%The elastic form factors of the nucleons give us some of the best
%opportunities to make precise measurements related to the structure of
%hadrons.
The elastic form factors probe the four-current distributions of the
nucleons, fundamental quantities that provide
%The distribution of electromagnetism in the nucleon is a fundamental quantity
%that provides 
one of the best opportunities to test our understanding of 
nucleon structure.
A number of theoretical techniques exist to describe the
nucleon's electromagnetic structure,
%the distribution of electromagnetism in momentum space,
including quark models, perturbative Quantum Chromo-Dynamics (pQCD), lattice QCD,
effective field theories, vector-meson dominance (VMD) models, etc.
Each at present has limitations, and its validity must be confirmed by experiment.
In the examples given,
\begin{itemize}
\item  quark models, as constructed, are phenomenological with no firm basis in QCD,
\item  pQCD is limited to high four-momentum transfer, and it is unknown at what
momentum transfer it becomes valid,
\item  lattice QCD is presently limited, by computational requirements, to describing the isovector 
(proton minus neutron) form factors, since the effects of disconnected quark lines
largely cancel in these,
\item  effective field theories are limited to small momentum transfer, and
\item  VMD models are constructed as fits to the existing data base.
\end{itemize}

Experimentally, the nucleon electromagnetic form factors are a central
part of the Jefferson Lab 12 GeV program, and it is desirable to measure
all four nucleon form factors over the widest possible $Q^2$ range, to similar 
precision.
This goal is particularly motivated so that one can construct the isovector form factors
for comparison with lattice calculations.


In the one-photon approximation the cross section for scattering of
electrons from
a spin-$\frac{1}{2}$ target can be written as
$$\frac{d\sigma}{d\Omega}=\eta \frac{\sigma_{\mbox{\scriptsize
    Mott}}}{1+\tau}\left((G_E)^2+\frac{\tau}{\epsilon}(G_M)^2
  \right)$$
where
\begin{itemize}
\item $\eta=\frac{1}{1+2\frac{E}{M_N}\sin^2(\theta/2)}$ is the recoil
factor
\item $\epsilon=(1+\vec{q}^2/Q^2\tan^2(\theta/2))^{-1}=
(1+2(1+\tau)\tan^2(\theta/2))^{-1}$
    is the longitudinal polarization of the virtual photon,
\item $\tau=Q^2/4 M_N^2$, and
\item $G_E(Q^2)$ and $G_M(Q^2)$ are the Sachs Electric and Magnetic form factors.
\end{itemize}
Alternately, the helicity conserving 
$F_1$ and helicity nonconserving $F_2$ form
factors can be written as simple linear combinations of the
 electric $G_E$ and magnetic $G_M$ form factors.
The measurement of these form factors for the proton and neutron
    probes their electromagnetic structures.

Little is known of the neutron's magnetic form factor, \gmnc, (and less of
its electric form factor) for $Q^2>4$ (GeV/c)$^2$.
{\it We propose to make several high precision measurements of  \gmn in the range
$4.5<Q^2<18.0$ (GeV/c)$^2$, the largest $Q^2$ proposed measurements of \gmn to date.}


Since the form factors are functions only of $Q^2$, they
may be separated by the Rosenbluth technique, making cross section 
measurements at
the same $Q^2$ but different $\epsilon$ to obtain different linear
combinations.  The apparent failure of this technique in extraction of \gep
%$G_E^p$
at $Q^2>1$ (GeV/c)$^2$ (as revealed by the recoil polarization 
method \cite{jones,gayou})
may indicate a failure of the one-photon exchange approximation \cite{afan}.
This does not invalidate the form given above, however.  It just
underscores the fact that the corrections may be non-negligible and
may become important particularly when trying to separate a small contribution from
a larger one.  This consideration does not present a great problem
when trying to extract the magnetic form factor of the neutron since
the electric form factor is generally much smaller, at least at low $Q^2$.
%Since the experimental situation at several (Gev/$c$)$^2$ is undetermined,
%it is quite possible that the neutron magnetic form factor is no longer dominant.


These Sachs form factors are trivially related to the Dirac and Pauli
form factors, F$_1$ and F$_2$, respectively,  which are
the coefficients of the 
helicity-conserving and -nonconserving currents to
which the photon can couple.  Non-relativistically the Sachs form
factors can be interpreted as the Fourier transforms of the charge and
current distributions to which the photon couples in the target.  No
such simple interpretation is available at higher $Q^2$.  The electric
form factor at any $Q^2$ can still be related to the Fourier transform
in the Breit frame.  But since the Breit frame is a different frame
for each $Q^2$, this relationship cannot be inverted to extract a
charge distribution without a prescription for boosting the nucleon. 
Recent work \cite{Miller} by Miller offers an interpretation of 
the infinite-momentum frame charge density of
the nucleon as a function of impact parameter in terms of the Dirac
form factor, $F_1=(G_e+\tau G_m)/(1+\tau)$.  Interestingly, this
implies that knowledge of the {\em magnetic} form factor of the
neutron is important to the understanding of its charge distribution.
Furthermore, Miller concludes that 
the neutron charge distribution is negative for small impact parameter,
which contrasts, at least na\"ively, with the long-standing 
belief that the neutron charge distribution is positive at the center.
A positive central charge distribution is in accordance with 
intuitive models --
for example, the neutron charge distribution reflects a virtual $p\pi^-$
pair, with the more massive proton closer to the center of mass.
The interest in understanding the charge distribution is
reflected by the appearance of a figure indicating 
the positive central charge distribution in the recent
Nuclear Physics long Range Plan.
%Thus, it is important to understand whether or not the central charge
%distribution of the neutron is indeed negative.
%The uncertainty in Miller's result arises mainly from the lack of high
%$Q^2$ measurements, necessitating measurements of the magnetic form factor,
%which we propose here.



In the approximation that the strange quark does not contribute to the
electromagnetic structure of the nucleon, the form factors can be
combined \cite{qf1,qf2,qf3}
to extract information about  the contributions of individual
quark flavors to the electromagnetic structure of the nucleon.
Assuming isospin symmetry, the up-quark distribution of the proton is
identical to the down-quark distribution of the neutron and {\it vice
  versa}. Since the electromagnetic couplings to the individual quarks
are known (and the coupling to gluons vanishes) the electric or
magnetic form factor of each nucleon can be written as a linear
combination of the electric and magnetic form factors of the two quark
flavors.  Combining measurements on the neutron and proton then allows
direct extraction of the ``up'' form factor (including contributions
from $u$ and $\bar u$ quarks from the sea, as well as valence quarks)
and the ``down'' form factor (also composed of all $d$ and $\bar d$ 
contributions).  In particular, improved measurements of the neutron's
magnetic form factor can be combined with existing measurements of the
proton's magnetic form factor to allow extraction of the ``up-magnetic
form factor'' and ``down-magnetic form factor''.  If, on the other
hand, the contribution of strange quarks is {\em not} negligible, then
the measurement of the neutron form factors would be critical to allowing
the strange contribution to be measured. (At present, however, there
are no plans to measure strange form factors in the $Q^2$ range in
which we propose to measure.)


%While the form factors may not lend themselves to simple interpretation
%in terms of the structure of the nucleons, they are still pivotal as
The form factors are pivotal as the meeting place between theory and experiment. 
% First-principles 
Calculations of nucleon structure (as opposed to parameterizations of form factors) can be
tested by their ability to predict the experimentally accessible
information on nucleon structure reflected in the form factors.  (Of
course, polarization observables and structure functions will also be
relevant.)  In particular, lattice QCD predictions will eventually
have the capability to make meaningful predictions of hadronic
structure.  Form factors of the proton and neutron will present
important tests of those predictions.



A great deal of experimental and theoretical effort
\cite{GPD1,GPD2,GPD3} is being expended
on an ambitious effort to greatly expand the knowledge of nucleon
structure by determining the generalized parton distributions.
Measurement of form factors plays an important role in that effort
since the form factors set the values of sum rules which the generalized parton
distributions must obey.


The neutron's form factors are more difficult to measure, of course, because there is no free-neutron target.  
Spin-asymmetry techniques have been used in extracting the tiny electric form factor of the
neutron \cite{pol1,pol2,pol3,pol4,pol5,pol6,pol7} 
and also in measuring the magnetic \cite{pol_gmn1,pol_gmn2,pol_gmn3} 
form factor, particularly at low $Q^2$.  
Generally at high $Q^2$, however, quasielastic
scattering from the deuteron has been used \cite{sub1,sub2,sub3,sub4,
SLAC_Rock,sub5,tag1,tag2,rat1,rat2,rat3,rat4,rat5,rat6,rat7,Will}
to extract \gmnc.  
This is based on the fact that the deuteron is a loosely coupled system, so
high-$Q^2$ quasi-elastic scattering can be viewed as the sum of
scattering from a proton target and scattering from a neutron target.
This simple picture is complicated only slightly by the fact that the
targets are not at rest but are moving with the ``Fermi motion''
intrinsic to the deuteron's wave-function.

Several techniques have been used to try to isolate the electron-neutron scattering of interest.  
In the ``proton-subtraction'' technique \cite{sub1,sub2,sub3,sub4,SLAC_Rock,sub5} 
single-arm quasi-elastic electron scattering from the deuteron is measured.  
This is combined with a measurement of single-arm elastic scattering from the proton.  
An attempt is made to fold in the expected effects of Fermi motion to
simulate the expected contribution of the proton in the measured quasi-elastic spectrum.  
This is then subtracted and the remainder is interpreted as a measure of quasi-elastic electron scattering off the
neutron from which the (almost purely magnetic) form factor can be determined.  
This technique tends to suffer from the error-propagation problems intrinsic to 
subtraction of two large numbers.  
At low $Q^2$ the proton electric form factor dominates (and the proton magnetic form factor is never
small compared to the neutron's).  
At high $Q^2$ inelastic background becomes a serious problem, to the extent that the quasi-elastic
``peak'' may not be visible, even as a shoulder on the background peak.  
Because these are single-arm measurements, no other information
is available with which to selectively reject background events.

The ``proton-tagging'' technique \cite{tag1,tag2} is a partial-coincidence method which takes
advantage of the fact that protons are easier to detect than neutrons.  
In that technique quasi-elastic electron scattering is
measured with an additional charged-particle detector 
%positioned where the recoiling nucleon is expected to pass (
centered around the direction of the momentum-transfer vector, $\vec{q}$.  
If no proton is detected, the event is ascribed to scattering from the neutron.  
This technique generally requires substantial theory-based corrections 
to account for the tail of the Fermi-motion which would cause a recoil proton to miss
the charged-particle detector (or cause a spectator proton to hit it).
Again, since the neutron is not detected, no cuts can be applied to
selectively reject inelastic background events.

We propose to use the ``ratio-method'' \cite{Durand} which is discussed in detail in the next section.  
It relies on measurement of both the recoil protons and recoil neutrons \cite{rat1,rat2,rat3,rat4,rat5,rat6,rat7,Will}.  
Inelastic background is substantially suppressed by even a crude nucleon-coincidence requirement.  
As will be seen in simulations
presented in section \ref{inelastic}, precise measurement of the
final-nucleon direction permits the use of cuts which further reduce
inelastic background down to manageable levels, even at the highest $Q^2$.
If the particle detection, particularly the neutron
detection is well understood, this technique is subject to the 
smallest systematic errors as it enjoys substantial cancellation of
many sources of systematic error which plague other techniques.

There are relatively few measurements of \gmn beyond $Q^2=1$ (GeV/c)$^2$.  
The few published measurements in the range $1<Q^2<4.5$
(GeV/c)$^2$ (shown in Fig. \ref{old_data}) have been eclipsed, both 
in number of points and in precision, by the recent CLAS data \cite{Will,Jeff} of
Lachniet.  
These (not yet published) data are shown in Fig. \ref{old_data} as the blue points.  
Several of the proposers of the present experiment played key roles in the CLAS experiment (Quinn,
Brooks and Gilfoyle).  
The ratio-method was used for those measurements and will be used in the proposed experiment.

In the figure, the value of \gmn is divided by the 'scaled dipole'.  The
dipole is a vector-meson-dominance-inspired empirical parameterization
of the proton's electric form factor: \gep $\approx G_D=(1+Q^2/.71$
(GeV/c)$^2$)$^{-2}$.  This appeared to be a good approximation for
\gep over a large $Q^2$ range until recent recoil-polarization 
measurements  \cite{jones,gayou}) showed that \gep actually 
fell rapidly below the
dipole form for $Q^2>1$ (GeV/c)$^2$.  The scaling approximation
hypothesizes that \gmp $\approx \mu_p G_D$ and \gmn $\approx \mu_n
G_D$.  The CLAS data show that the `scaled dipole' is a surprisingly
good approximation for \gmn out to $Q^2\approx 4.5$ (GeV/c)$^2$.

Beyond $Q^2=4.5$ (GeV/c)$^2$ there are only a 
few points, with large errors.  The
points plotted in green in Fig. \ref{old_data} are SLAC measurements
\cite{SLAC_Rock} made using the ``proton-subtraction'' technique.
While these points have relatively large
errors, they point to a trend which 
%appears  inconsistent with
is not seen in 
the CLAS data.  This makes it particularly interesting to investigate
the behavior of  \gmn  in the range $Q^2>4$ (GeV/c)$^2$  with a
measurement which is independent of either of those shown in Fig.
\ref{old_data}.  A similar plot is presented at the end of the
proposal, with the projected errors of the proposed measurement superimposed.





\begin{figure}
\includegraphics[width=6in]{old_dat.pdf}\\
\caption{\label{old_data}
Existing data on \gmn in $Q^2>1$ (GeV/c)$^2$ range are plotted  as ratio to
scaled dipole approximation.  Blue points are from CLAS e5 run
\cite{Will,Jeff}.  Dark blue lines show the statistical error while
light blue lines
are the quadrature sum of statistical and systematic errors.  Green circle
\cite{SLAC_Rock} and magenta circle \cite{sub5}
points are from SLAC. Older data are shown as yellow squared
\cite{rat2}, solid squares \cite{tag2}, and hollow triangles \cite{sub4}.
Some points have been slightly displaced horizontally to avoid overlap.}
\end{figure}


\section{Technique}
We propose to use the ``ratio method''\cite{Durand} to determine \gmn from
quasi-elastic electron scattering on the deuteron for $3.5 < Q^2 < 18$
(GeV/c)$^2$.  This method is far less sensitive to systematic errors than
the ``proton-subtraction'' or ``proton-tagging'' techniques. 

Use of the ``ratio method'' requires the measurement of both
neutron-tagged, d(e,e$'$n), and proton-tagged, d(e,e$'$p), quasi-elastic
scattering from the deuteron.  Simultaneous measurements of both
these reactions provides a substantial reduction of systematic error
because numerous experimental uncertainties 
cancel in forming the ratio:
\begin{equation}
R''=\frac{\left.\frac{d\sigma}{d\Omega}\right|_{\mbox{\small d(e,e$'$n)}}
}{\left.\frac{d\sigma}{d\Omega}\right|_{\mbox{\small d(e,e$'$p)}}}
\label{r-double-prime-eqn}
\end{equation}
This is insensitive, for example, to target thickness, beam intensity,
deadtime, electron trigger efficiency, electron acceptance, and the 
detection and reconstruction efficiency for the scattered electron track.

With a small and accurately-calculable nuclear correction,
$\epsilon_{\mbox{\scriptsize nuc}}$,  this measured ratio of quasi-elastic
cross sections can be used to determine the ratio of the elastic cross
sections:
$$ R'
=\frac{\left.\frac{d\sigma}{d\Omega}\right|_{\mbox{\small n(e,e$'$)}}
}{\left.\frac{d\sigma}{d\Omega}\right|_{\mbox{\small p(e,e$'$)}}}
=\frac{R''}{1+\epsilon_{\mbox{\scriptsize nuc}}}$$
Because of final-state interactions and other nuclear effects, there
would be substantial corrections to the na\"ive assumption that the 
coincident quasi-elastic cross section is equal to the cross section
for elastic scattering from the free nucleon.  Further, these
corrections would depend upon the fraction of the quasi-elastic peak
which is integrated.  A great advantage of the ratio method (with
a deuteron target) lies
in the fact that these corrections are almost identical for the case
of the neutron and the proton and so they cancel almost completely 
in the ratio.  The
small surviving correction, $\epsilon_{\mbox{\scriptsize nuc}}$, to the
ratio arises due to small effects such as the neutron-proton mass difference.
Figure \ref{aren_corr} shows detailed calculations\cite{Arenhovel} 
by Arenh\"ovel 
of the correction factor required in calculating the ratio of the nucleon 
elastic cross sections from the ratio of the integrated nucleon-tagged
quasi-elastic cross sections.  Here $\theta_{pq}$ is
the angle between the struck nucleon's final momentum vector
($\vec{p}$) and the momentum-transfer vector ($\vec{q}$). Final state
interaction effects are minimized by putting a tight cut on
$\theta_{pq}$ (i.e. requiring that the nucleon actually recoil in the
direction which would be expected in the absence of Fermi motion and
final state interactions).  It will be seen below that the region of
interest in the present proposal has $\theta_{pq}<3$ degrees.
Even at $Q^2=1.2$ (GeV/c)$^2$ the correction is seen to be less than
1\%.  For higher $Q^2$ up to 5 (GeV/c)$^2$ calculations of the nuclear
correction have been made \cite{Will,Jeff} using a model\cite{Jesch}
which applies Glauber theory to model the final-state interactions.  Again,
the corrections for the neutron and proton are almost identical and
cancel in the ratio.  The residual correction on the ratio 
$\epsilon_{\mbox{\scriptsize nuc}}$ was found to be under 0.1\%.  The
corrections are expected to be very small and calculable in the range
of interest here.  This correction is expected to contribute
negligibly to the systematic error of the measurement.


\begin{figure}
\begin{center}
\includegraphics[width=5 in]{aren_corr.pdf}\\
\end{center}
\caption{\label{aren_corr}
Arenh\"ovel predictions for (low $Q^2$) 
nuclear corrections (including FSI) as a function of the maximum
accepted value of $\theta_{pq}$.
%the angle between the struck nucleon's final momentum vector
%($\vec{p}$) and the momentum-transfer vector ($\vec{q}$). 
The required
correction is seen to be small for tight cuts on $\theta_{pq}$ and to
{\em decrease} with increasing $Q^2$.}
\end{figure}

Writing $R'$ in terms of neutron form factors,
$$ R'=\frac{\eta\frac{\sigma_{\mbox{\scriptsize Mott}}}
{1+\tau}\left((G_E^n)^2+\frac{\tau}{\epsilon}(G_M^n)^2 \right)}
{\left.\frac{d\sigma}{d\Omega}\right|_{\mbox{\small p(e,e$'$)}}}$$
where
$\eta$, $\epsilon$, and $\tau$ are defined above.  
%, as above,\\
%$\eta=\frac{1}{1+2\frac{E}{M_N}\sin^2(\theta/2)}$ \\
%$\epsilon^{-1}=1+\vec{q}^2/Q^2\tan^2(\theta/2)=1+2(1+\tau)\tan^2(\theta/2)$\\
%and $\tau=Q^2/4 M_N^2$

%Then, with a minor correction ($\approx1$ \%, using the Galster
%parameterization) for the electric form factor of the neutron,
%this in turn allows determination of the ratio of interest,
From this, then, can be extracted the ratio of interest,
\begin{equation}
R=R'-\frac{\eta\frac{\sigma_{\mbox{\scriptsize Mott}}}{1+\tau}(G_E^n)^2}{\left.\frac{d\sigma}{d\Omega}\right|_{\mbox{\small p(e,e$'$)}}}
=\frac{\eta\sigma_{\mbox{\scriptsize Mott}}\frac{\tau/\epsilon}{1+\tau}(G_M^n)^2}
{\left.\frac{d\sigma}{d\Omega}\right|_{\mbox{\small p(e,e$'$)}}}
\label{r_eqn}
\end{equation}
The term subtracted to extract $R$ from $R'$ will be small ($\approx
1$\% at most, and much less at high $Q^2$) if %$G_E^n$
\gen follows
the form of the Galster parameterization.  In section
\ref{systematic_errors} we will allow for an error of 100\% of Galster
(at low $Q^2$) up to 400\% of Galster (at high $Q^2$) and find that
this correction still does not cause unacceptable systematic errors.
A measurement of \gen up to $Q^2=10$ (GeV/c)$^2$ is planned\cite{new_GEN}
in a time-frame which will make it useful for analysis of results from
this measurement.

This measurement of $R$ then allows \gmn to be determined, given 
just the proton's
elastic cross-section at the corresponding kinematics.  
It may be noted that, because $R$ is proportional to the square of \gmnc, the
fractional error on \gmn will actually be only half of the fractional
error on $R$.  Since the quantity of greatest interest is \gmnc, it is
conventional to report the expected size of the errors on \gmnc.  
%The real result of the experiment, however, will be
%measurements of R.  
However, the experiment will actually be a direct measurement of 
$R''$ (from which $R$ is
inferred with small corrections, as described above).
This distinction is significant only in that
present uncertainties on the proton's form factors (and cross
section) do not actually imply systematic errors on the quantity
being measured, $R''$ (or $R$).  Subsequent improvements in the determination of the
proton cross section, at the kinematics of interest, can be combined
retrospectively with the results for $R$ from this measurement to
obtain improved values for \gmnc.  There would be no need to repeat the
analysis of this experiment to incorporate new proton measurements.

Similarly, since the proton cross section is dominated by \gmp
%{G$^{\mbox{\scriptsize p}}_{_{\mbox{\tiny M}}}~$}
for these kinematics, the ratio of formfactors,
\gmnc/\gmp
%{G$^{\mbox{\scriptsize p}}_{_{\mbox{\tiny M}}}~$}
can be cleanly extracted from the data.  In many ways this ratio is
more fundamental than \gmnc, lending itself to direct comparison to
theoretical predictions.  Extraction of this ratio does not suffer
from a systematic error due to uncertainties in proton cross section
measurements. Like \gmnc, this ratio enjoys a factor of two reduction
in the fractional error compared to $R$.

\section{Proposed Kinematics}
The kinematic points at which we propose to measure are shown in Table
\ref{Kin_table}.  The lowest-Q$^2$ points will overlap with existing
CLAS measurements while the highest-Q$^2$ points will greatly extend
the range in which \gmn is known with high precision

\begin{table}
\begin{center}
\caption{Kinematics of proposed measurements \label{Kin_table}}
\vspace{.2in}
{\begin{tabular}{|l|l|l|l|l|l|}
\hline
Q$^2$ & E$_{\mbox{beam}}$ & $\theta_e$ & $\theta_N$ & E$'$ & P$_N$ \\
(GeV/c)$^2$ & (GeV) & & & (GeV) & (GeV/c)\\
\hline
3.5 & 4.4& 32.5$^\circ$&31.1$^\circ$ & 2.5 & 2.6\\
4.5 & 4.4& 41.9$^\circ$&24.7$^\circ$ & 2.0 & 3.2\\
6. & 4.4& 64.3$^\circ$&15.6$^\circ$ & 1.2 & 4.0\\
8.5 & 6.6& 46.5$^\circ$&16.2$^\circ$ & 2.1 & 5.4\\
10. & 8.8& 33.3$^\circ$&17.9$^\circ$ & 3.5 & 6.2\\
12. & 8.8& 44.2$^\circ$&13.3$^\circ$ & 2.4 & 7.3\\
13.5 & 8.8& 58.5$^\circ$&9.8$^\circ$ & 1.6 & 8.1\\
16. & 11.& 45.1$^\circ$&10.7$^\circ$ & 2.5 & 9.4\\
18. & 11.& 65.2$^\circ$&7.0$^\circ$ & 1.4 & 10.5\\
\hline
\end{tabular}}
\end{center}
\end{table}


While the scattered electron energy is relatively constant (mostly near 1 to 2
GeV) across the kinematic points, the central nucleon momentum of
interest is seen to vary from 2.65 GeV/c to 10.5 GeV/c.  Individual
calibrations with 'tagged' protons and 'tagged' neutrons will be 
carried out at the three lowest-Q$^2$ kinematic points to ensure that
the neutron and proton detection efficiencies are well known.  As will
be seen below, the efficiencies are large and stable for higher Q$^2$.
%  Since
%the variation in efficiency is expected to be quite small, more
%extensive calibrations at lower momenta can be extrapolated to higher
%nucleon momenta with guidance from more restricted measurements at the
%higher momenta.

\section{Apparatus}
The use of the ratio method depends upon detection of both scattered
neutrons and protons.  Potential sources of systematic error arise in
%the understanding of the acceptance for these particles and in the
%calibration of the efficiencies of their detection.
determining the acceptance and detection efficiency of these particles.
  Errors associated
with nucleon acceptance can be reduced by matching the neutron and
proton acceptances so they cancel in the ratio (as does the electron
acceptance and efficiency). 
% It is critical that the detection
%efficiecies for neutrons and protons be well determined by calibration
%rections. 

We propose to use the existing BigBite spectrometer in Hall A to
measure the momentum and angle of the scattered electrons and the 
BigHAND detector  to detect both the scattered neutrons
and protons.  Nucleons scattered toward the BigHAND detector
will pass through the field of a large aperture dipole magnet which
will be positioned  along the nucleon flight path to vertically deflect protons
relative to neutrons.  The layout of the experiment is shown
schematically in Fig. \ref{layout}.


\begin{figure}
\includegraphics[width=6in]{layout.pdf}\\
\caption{\label{layout}
A schematic view of one possible configuration of the 
apparatus is shown. BigBite will detect
scattered electrons while BigHAND will detect the scattered nucleons.
The dipole magnet ``BigBen'' will deflect protons for the purpose of
particle identification.  The coil configuration shown is one option
for avoiding interference between the coils and the beamline.  A
magnetically-shielded hole in the return iron will allow the
unscattered beam to continue on to the beam dump.  
The corrector coils ``CC'' will compensate
for any effect of residual magnetic field on the beamline.  
Note: the 17 m flight path to BigHAND 
is not drawn to scale.
}
\end{figure}




The targets will be 10 cm long liquid deuterium (and liquid hydrogen
for calibration) cells with 100 $\mu$m aluminum windows.   This
gives about 1.7 g/cm$^2$ of target compared to about 0.054 g/cm$^2$ in
the windows.  As discussed below, selection cuts will reduce the
contribution of quasi-elastic events from aluminum below this 3.2\% ratio.
To obtain percent-level precision, however, it will be necessary to
subtract the contribution from the windows.  A dummy target cell will
be used, having windows at the same position as the real cell but with
windows thick enough to give the same luminosity as for a full cell.
Sufficient statistics for subtraction of the windows will be obtained
by running on the dummy cell for about four percent of the beam time
used for the full target.

As discussed below, past experience suggests the BigBite and 
BigHAND detector rates will
be reasonable  at a luminosity of $6.7\times10^{37}$/A /cm$^2$/s
where A is the number of nucleons in the target.  For a luminosity of
3.3 $\times 10^{37}$ on a 10 cm deuterium target, the beam current
would be 10.5 $\mu$A.

%{\bf XXX  Bogdan, please review and update. \\}
BigBite, shown in Fig. \ref{fig_BigBite}, is a large acceptance non-focusing
magnetic spectrometer.  It has a large 
acceptance (roughly 53 msr in the
intended configuration) and has been used successfully at high
luminosity ($\approx 10^{37}$/cm$^2$/s).  It will be configured for
high momentum measurements, with the entrance aperture of the 
dipole 1.55 m from the target and
%{\bf XXX} 
widely spaced coordinate-measuring detector  planes %spaced over 85 cm.
For the high luminosity of the experiment, the spectrometer will be
instrumented with GEM detector planes.  These detectors are planned
for use in the polarimeter of the new Super BigBite Spectrometer (SBS) and so
will be available for use as the tracking detectors for BigBite.
The GEM detectors are designed in a modular form specifically chosen
to allow them to be configured to instrument BigBite as well as the
new SBS spectrometer.  Figure \ref{GEM} shows how the modules will be
assembled to form four tracking planes for BigBite.
 In this configuration,
the expected momentum resolution will be $\sigma_p/p\approx$ 
.5\% because of the high resolution and small
multiple-scattering resulting from the relatively thin GEM detectors.
The angular resolution is
expected to be better than 1 mr in both horizontal and vertical angles 
\cite{bb_simulation}.

%The addition of a gas Cerenkov detector in BigBite is expected to
%eliminate the majority of the trigger rate which was seen in the GEn
%experiment.  Analysis of that data showed that most BigBite triggers
%were not associated with any charged tracks, indicating that the
%shower counters had been triggered by high energy photons from $\pi^0$
%decay.  Elimination of this source of false triggers is expected to
%reduce the trigger rate in BigBite to a few hundred Hz.  Therefore we
%intend to run using just BigBite (with the gas Cerenkov required) as
%a trigger

We intend to run with a single-arm trigger based only upon the
electron spectrometer.  This eliminates any neutron/proton bias from
the trigger and ensures that the trigger efficiency cancels completely
in the ratio of interest, R.
The lead-glass electromagnetic calorimeter will be used for the
trigger of the experiment.  
As will be discussed in section \ref{rates}, this is expected to allow
a modest single-arm trigger rate of less than 1 kHz.  
If the gas Cerenkov has been successfully commissioned, it may also be
used for redundant rejection of pions in BigBite.


\begin{figure}
%\includegraphics[width=5in]{bb_diagram.pdf}
\includegraphics[width=5in]{bb_gem_02.jpg}
\caption{\label{fig_BigBite}The BigBite spectrometer, 
configured for high momentum, high luminosity running.  
Tracking is performed with GEM detectors 
and a gas Cerenkov counter is located between the
detector packages. (The target label refers to another experiment.)}
\end{figure}

\begin{figure}
\includegraphics[width=5in]{BigBite-tracker.pdf}
\caption{\label{GEM}
The layout of a 40 cm X 50 cm GEM module is shown along with the
arrangements in which the modules would be used to form the four tracking
layers for BigBite.  The required active area for each layer is shown
as a blue dashed line.  One Amplification Unit is removed in the
diagram to show the U-V readout board. }
\end{figure}

The BigHAND (Hall A Nucleon Detector), shown in Fig.
\ref{fig_BigHAND}, 
is a large array of scintillators interspersed with seven half-inch
thick iron converters
which initiate hadronic showers.  A ``veto'' layer on the front face
was intended to distinguish neutrons from protons.  (An alternative
technique will be employed for this measurement, as will be discussed
below.)  Heavy shielding (2 inches of lead and 1 inch of iron) reduces
the electromagnetic rate in the scintillators.  

The observed r.m.s. 
spatial resolution
of reconstructed hadronic showers for the GEn experiment \cite{GEn-proposal} 
 is typically
4.3 cm. vertically and
7 cm. horizontally.  The detector will be located 17 meters from the
target so the corresponding angular resolution will be 2.5 mr and 4 mr
(0.15$^\circ$ and .24$^\circ$),
respectively.  With the higher nucleon momentum of the present
proposal, the resolution may be be somewhat improved, especially in
the horizontal direction.  This excellent resolution will permit
critical cuts on the direction of the recoil nucleon relative to the 
$\vec q$-vector direction.

\begin{figure}
\begin{center}
\includegraphics[width=4in]{figures/bighand.pdf}
\end{center}
\caption{\label{fig_BigHAND}The BigHAND (Hall A Nucleon Detector). 
Iron converters initiate hadronic showers which are detected by
the large scintillators. }
\end{figure}

The r.m.s. time-of-flight resolution of the overall detector array is
approximately 400 ps\cite{Rob_F}.  Since the momenta of the nucleons
of interest are sharply defined, this will allow tight timing cuts to
reject accidentals.With a 17 meter flight path, this
will allow clean rejection of low energy nucleons from break-up of nuclei.


%\begin{table}
%\begin{center}
%\caption{Flight time for nucleon vs. photon for 17 m flight path\label{ToFs}}
%\vspace{.2in}
%{\begin{tabular}{|l|l|l|l|l|}
%\hline
%Q$^2$  & $\theta_N$  & P$_N$ & T$_N$ & T$_N-$T$_\gamma$ \\
%(GeV/c)$^2$ & & (GeV/c) & (ns) & (ns)\\
%\hline
%3.5 &28.7$^\circ$ & 2.65&60.1&3.4\\
%4.5 &21.7$^\circ$ & 3.2 &59.0&2.4\\
%5.25  &22.7$^\circ$ & 3.6 &58.7&2.0\\
%6  &18.7$^\circ$ & 4.0 &58.2&1.54\\
%7  &18.7$^\circ$ & 4.6 &57.8&1.17\\
%8  &14.9$^\circ$ & 5.1 &57.6&0.95\\
%\hline
%\end{tabular}}
%\end{center}
%\end{table}




The efficiency predicted by simulation \cite{BH_simulation}  for 
conversion and detection of neutrons and protons
in the momentum range of interest is shown in Fig. \ref{np_eff}
for a 20 MeV (electron equivalent) threshold.
For neutrons, the predicted efficiency is seen to be over 70\% at the
lowest $Q^2$ of interest and to rise to over 90\% at the highest
$Q^2$.  The proton efficiencies are even higher.


As described in the next section, a magnetic 'kick' will be used to
distinguish protons from neutrons.
We have identified a large-aperture magnet at Brookhaven National
Laboratory which could be used for this purpose.  
This same dipole will serve as the
spectrometer magnet for the Super BigBite spectrometer.
In its present
configuration, this ``48D48'' magnet, shown in Fig. \ref{48D48} 
has a 120 cm $\times$ 120 cm
(48 in. $\times$ 48 in.) pole face with a 47 cm gap.

The magnet will be modified so it can be positioned near 
%$19^\circ$ with respect to 
the beam line.  This will involve machining a hole through
the return yoke to provide a low-field, iron-free region 
for passage of the outgoing
beam.  Asymmetric field coils will be needed to avoid interfering
with the outgoing beam,  One possible solution,
shown in Fig. \ref{layout}, involves the use of an existing
standard-coil/booster-coil pair originally designed to drive half of a
100 cm gap magnet.   Correcting coils would be used to compensate
for beam steering due to any residual fields.

The details of the yoke hole,
field coils, field clamps, and correcting coils will be finalized using a
magnetic field simulation program such as TOSCA.  The magnet
%will be constructed so that the it 
modifications will be designed such that the magnet 
can also be used for the
recently-approved high-Q$^2$ proton elastic cross section
experiment\cite{E12-07-108}.  


\section{Neutron/Proton Identification}

If neutron/proton identification were based
solely upon the response of the 'veto' layer, then
contamination by mis-identification would be a significant problem.
Experience from the GEn experiment\cite{GEn-proposal}  
indicates that about 2.5\% of
(independently identified) protons fail to fire the veto layer and would be
mis-identified as neutrons \cite{Seamus}.  More troublesome is the fact that a
significant fraction ($\approx 40\%$) of the detected neutrons
actually fire the veto layer (because the hadronic shower is initiated
in the front shielding).  Event topology could be used to more cleanly
identify a subset of more unambiguous neutrons or protons but at the
cost of a large reduction in detection efficiency.


\begin{figure}
\includegraphics[width=6. in]{np_eff.pdf}\\
\caption{\label{np_eff}
The predicted \cite{BH_simulation}  detection efficiency 
of BigHAND is shown for neutrons
(blue squares) and protons (red circles) as a function of the central nucleon
momentum associated with each $Q^2$.
}
\end{figure}

A much more clean separation of neutrons from protons can be made,
without loss of efficiency, by introducing a dipole magnet to
deflect the protons vertically.  If the initial direction of the
nucleon could be accurately predicted, then only a small deflection
would be needed to distinguish charged particles from neutral ones.
In the case of quasi-elastic scattering, the
measured $\vec q$-vector does not precisely predict the direction of the
struck nucleon's final momentum since the initial momentum of the
nucleon within the deuteron also contributes.  Using a reasonable
model of the deuteron's wave-function\cite{Lomon}, the momentum
distribution can be determined.  It is found that, with 95\% probability,
the component of the nucleon's momentum along any chosen direction is
less than 100 MeV/c.  A magnetic 'kick' of 200 MeV/c, then would
separate quasi-elastic protons from neutrons at the 95\% level.  
In the simplest analysis, a horizontal line could be defined across 
the face of BigHAND (for any given event detected in BigBite) such that the struck nucleon
would have a 95\% probability of falling below the line if the
particle were a neutron and a 95\% probability of falling above the
line if the particle were a proton.  

The remaining 5\%
mis-identification can still be accurately corrected by 
using the 'veto' layer (and possibly event topology) to determine the actual
%measuring the 
distribution of neutron and proton events relative to the ideal
positions they would have for elastic kinematics.  In particular the
spread of the neutron distribution relative to the $\vec q$-vector can
be investigated by
studying the sample which do not fire the veto (and making a modest
correction for protons).
Also, since
the initial momenta of the nucleons are vertically symmetric, the
actual distributions of neutron and proton events can be empirically
determined by observing the distributions of those neutrons which are
displaced downward from the point predicted for elastic kinematics and
those protons which are displaced upwards from the (magnetically
deflected)  point predicted for elastic kinematics.  
Either of these correction techniques 
should allow the systematic errors due to neutron/proton
mis-identification to be reduced to well below 1\%, and the
comparison of the two techniques should allow the confident
determination of this contribution to systematic error.

%The 200 MeV/c kick required to separate quasi-elastic protons from
%neutrons can be achieved by applying a dipole field, near the beginning
%of the flight path, having a field integral of $\int{B dl}\approx .85$ T$\cdot$m.
%We have identified a large-aperture magnet (a 48D48 spectrometer
%magnet from BNL) which could serve the purpose.  The recently-approved
%high-Q$^2$ proton elastic cross section experiment\cite{E12-07-108}
%will make use of the same magnet.  Some modification of the flux
%return yoke of the magnet will be needed to allow it to be placed
%close to the beam.


%48D48 
The 200 MeV/c kick required to separate quasi-elastic protons from
neutrons can be achieved by applying a dipole field, near the beginning
of the flight path, having a field integral of 
$\int{B dl}\approx 0.85$ T$\cdot$m.

For high Q$^2$, a 200 MeV/c kick would give a relatively small spatial
separation on the BigHAND detectors.  A proportionally larger field
will be used to ensure an adequate displacement. Table \ref{field_tab}
gives the field integrals assumed in the Monte Carlo simulations
presented below.  Also given are the resulting mean separations
between the undeflected neutrons and the deflected protons.  For low
Q$^2$ a larger deflection not be an advantage as it would deflect
protons out of BigHAND acceptance so they would be lost.  (See the
discussion of fiducial cuts below.)  

An alternate technique may be employed, at low Q$^2$, to use the 
deflection magnet to achieve neutron/proton identification.   This involves
making two sets of measurements, one with the field on and and one
with the field off.  The field-on measurements would be used only to
measure neutron-coincident events, 
indicated by the proximity of the BigHAND hit to the
position expected based on the $\vec{q}$-direction.  The field-off measurement
would be used to measure the sum of the  neutron-coincident and
proton-coincident events.
The number of proton coincidences would then be determined as the difference
between these measurements. There are two advantages of this technique over the
simultaneous measurements described above.  The BigHAND
acceptance is guaranteed to be identical for both types of event since
they are both detected in the same region of BigHAND, with no deflection.
Also there is no loss in acceptance resulting from the use of a 
strong dipole field so the separation can be improved by using as
strong a field as practical.  The apparent disadvantages of the technique are
that more beam time is needed to make the two separate measurements
and that there is a loss in statistical accuracy for the proton-tagged
events due to error propagation.  Both of these disadvantages are
negligible for the very high coincidence rates which will occur for
the low-Q$^2$ points.  The choice between these techniques for 
low-Q$^2$ measurements will be based on further simulation.  For the
higher-Q$^2$ points, simultaneous measurements present a clear
advantage in rates.  For the purpose of the discussion and 
simulations presented in section \ref{inelastic}
here, simultaneous measurement of proton-coincident and
neutron-coincident events have been assumed.


\begin{table}
\begin{center}
\caption{Field strength for deflection dipole is given for each
  of the proposed kinematics.  Also given are $\Delta_{\mbox{pn}}$,
  the separation on BigHAND between an undeflected neutron and a
  deflected proton having the same $\vec{q}$-vector, and
  $P_{\mbox{kick}}$, the effective vertical momentum 'kick' given by
  the dipole. \label{field_tab}}
\vspace{.2in}
{\begin{tabular}{|c|c|c|c|c|c|c|c|c|c|}
\hline
Q$^2$ (GeV/c)$^2$&3.5 & 4.5& 6.0& 8.5& 10.& 12.& 13.5& 16.& 18.\\
\hline
$\int B dl$ (T-m)&0.85&0.85&0.85&0.91&1.05&1.24&1.37&1.60&1.88\\
\hline
$\Delta_{\mbox{pn}}$(cm)&129&107&85&68&68&68&68&68&68\\
\hline
$p_{kick}$(MeV/c)&200&200&200&214&247&292&322&376&442\\
\hline
\end{tabular}}
\end{center}
\end{table}


\begin{figure}
% Hmmmm... 6.5 M file... got to find something smaller!
%\includegraphics[width=8.5 in,angle=90]{d12-m-2408-5-1.jpg}\\
%pdflatex doesn't know about tif
%\includegraphics[width=8.5 in,angle=90]{d12-m-2408-5-1.tif}\\
\includegraphics[width=6. in,angle=0]{48D48_photocopy.pdf}\\
\caption{\label{48D48}
Assembly diagram for generic 48D48 spectrometer magnet. Magnet is shown mounted
for horizontal bend-plane but will be used for vertical bend plane.
Coil configuration shown is that used at BNL.
}
\end{figure}

\section{Acceptance and Fiducial Cuts}



\label{fiducials}

Here we discuss event-selection cuts which will be applied to
ensure that the
systematic errors due to acceptance losses remain very small.


In the case of elastic kinematics (applicable for the calibration
reactions discussed in the next section) there is a direct mapping of
scattered-electron direction to recoiling-nucleon direction.  The
acceptance of both BigBite and BigHAND can be conveniently expressed
in terms of the electron-scattering angles.  The solid angle for
acceptance of coincidence events is then found from the overlap of
the acceptance of the two detector systems.  This is shown,
for the nine kinematic points, in 
Fig. \ref{Theta-Phi-accept} with the full geometric acceptance of
BigBite and a reduced acceptance of BigHAND, as discussed below.
At the lowest-Q$^2$ points, BigHAND is a limiting aperture and
fiducial cuts will be needed to restrict events for which the nucleon
misses it.  At large Q$^2$ the acceptance is limited by BigBite and
the coincident events fall in a small region at the center of BigHAND.


\begin{figure}
\includegraphics[width=2.in]{figures/acceptance_32_del_0.jpg}\hfill
\includegraphics[width=2.in]{figures/acceptance_30_del_0.jpg}\hfill
\includegraphics[width=2.in]{figures/acceptance_36_del_0.jpg}\\
\includegraphics[width=2.in]{figures/acceptance_42_del_0.jpg}\hfill
\includegraphics[width=2.in]{figures/acceptance_56_del_0.jpg}\hfill
\includegraphics[width=2.in]{figures/acceptance_52_del_0.jpg}\\
\includegraphics[width=2.in]{figures/acceptance_53_del_0.jpg}\hfill
\includegraphics[width=2.in]{figures/acceptance_66_del_0.jpg}\hfill
\includegraphics[width=2.in]{figures/acceptance_68_del_0.jpg}\\
\caption{\label{Theta-Phi-accept}
The acceptance of BigBite as a function of electron scattering angles
($\theta,\phi$) is shown in red, the acceptance of
BigHAND for elastically scattered neutrons {\em as a function of the
electron scattering angles} is shown in blue.  Overlap
(corresponding to coincident acceptance for elastic events) is shown
in green.  
The acceptance of BigHAND used for the plot is a subset for which
an elastic proton would also fall in the active area.
The plots 
correspond to the kinematics of the experiment (see Table
\ref{Kin_table}).  From left to right, the plots represent: Upper
row, $Q^2=3.5$, 4.5, 6.0; Middle row, $Q^2=8.5$, 10., 12.; 
lower row, $Q^2=13.5$, 16., 18. (GeV/c)$^2$.}
\end{figure}

%\subsection{Fiducial Cut on $\vec{q}}

For quasi-elastic events, the $\vec q$-vector can be reconstructed  
based on the scattered-electron momentum and direction measured by
BigBite. 
A fiducial cut can
be placed on the direction of $\vec q$ to choose the central direction
of the scattered nucleons.  While Fermi motion will widen the image,
the ideal case of elastic scattering can be used to map the acceptance
to a position distribution of neutrons on BigHAND (and a similar
proton distribution, taking into account the 
deflection by the dipole magnet).
The green overlap regions in
Fig. \ref{Theta-Phi-accept}
represent the angular region of the BigBite acceptance for
which scattering at elastic kinematics would send a nucleon into the
BigHAND acceptance regardless whether it was a neutron or proton.
%For the purpose of
%defining this fiducial region, the effective size of the BigHAND
%detector was reduced from the
%full geometric acceptance, as will now be described.  
Potential systematic errors are greatly
reduced by {\em using the same fiducial region for both neutron-coincident
and proton-coincident measurements}.  The `image' of the fiducial
region projected onto BigHAND will differ  for protons and neutrons
because of the vertical kick given to the protons by the dipole
magnet.  To first order, the effect of this offset can be prevented
from introducing a difference in acceptance by using a reduced
fiducial region. 
To determine whether a particular $(\theta,\phi)$ point is within the
fiducial, the direction and magnitude of the corresponding $\vec{q}$
for elastic scattering are determined.  The trajectories are evaluated
for both a neutron and a proton with that momentum.  Only if both such
particles would fall in the active region of BigHAND is the angular
point within the fiducial.

As a result of the fiducial cut, the neutron image for the accepted
elastic events would leave an empty strip at the top of the BigHAND 
acceptance.   The size of this strip is
determined by the shift in the proton image relative to the neutron
image.
Similarly the elastic protons occupy the top of BigHAND acceptance, 
leaving an unoccupied strip at the bottom of acceptance.  The size of the
strip follows from the vertical kick given to the 
protons by the dipole, the nucleon momentum, $p_N$, and the distance, $L$ from the
target to BigHAND.  While quasi-elastic protons (and neutrons) are not
guaranteed to remain within the acceptance, the matching of the 
acceptance losses is improved by this reduced fiducial cut.
Reducing the size of BigHAND by this strip of size
$D=\frac{\Delta p}{p_N}L$ at the top when calculating the elastic
neutron acceptance gives the BigHAND coverage shown in Fig. 
\ref{Theta-Phi-accept}.



%\begin{table}
%\begin{center}
%\caption{
%Top rows give the fraction, $f_{\mbox{\scriptsize fid}}$, of
%quasi-elastic electrons in BigBite which pass the fiducial cut.  
%Bottom rows give calculated probability, $P_{\mbox{\scriptsize coinc}}$
%for a quasi-elastic electron which passes 
%the fiducial cut, that the corresponding nucleon will
%fall within the BigHAND active area.   
%For the lowest $Q^2$ kinematic points, results for several
%fiducial cuts are presented with 'safety margin' set by $\delta$.
%For convenience, lower rows also give $\mathcal{F}_{\mbox{\scriptsize
%acc}}(\delta)=f_{\mbox{\scriptsize fid}}(\delta)/f_{\mbox{\scriptsize
%fid}}(0)$, the fraction of events surviving a cut given by $\delta$.
%\label{coincidence-prob}}
%\vspace{.2in}
%{\begin{tabular}{|ll|ll|l|l||l|}
%\hline
%$Q^2=$&3.5& &4.5& 6.0&8.0(GeV/c)$^2$&$\delta$(MeV/c)\\
%\hline
%$f_{\mbox{\scriptsize fid}}$(\%)&&$f_{\mbox{\scriptsize fid}}$(\%)&&$f_{\mbox{\scriptsize fid}}$(\%)&$f_{\mbox{\scriptsize fid}}$(\%)& \\
%67.2&&78.0&&87.8&98.8&0\\
%59.9&&74.0&& & &20\\
%50.2&&68.6&& & &40\\
%39.4&&61.7&& & &60\\
%28.5&&52.3&& & &80\\
%17.7&&41.0&& & &100\\
%8.1&&28.9&&  & &120\\
%\hline
%\hline
%$P_{\mbox{\scriptsize coinc}}$(\%)&$\mathcal{F}_{\mbox{\scriptsize acc}}(\delta)$(\%)&
%$P_{\mbox{\scriptsize coinc}}$(\%)&$\mathcal{F}_{\mbox{\scriptsize acc}}(\delta)$(\%)&
%$P_{\mbox{\scriptsize coinc}}$(\%)&$P_{\mbox{\scriptsize coinc}}$(\%)&\\
%\hline
%86.5&100&93.0&100&95.5&97.5&0\\
%90.4&89.&94.1&95.&&&20\\
%92.9&75.&95.3&88.&&&40\\
%94.7&59.&96.4&79.&&&60\\
%96.0&42.&97.1&67.&&&80\\
%96.5&27.&98.1&53.&&&100\\
%97.0&12.&98.4&37.&&&120\\
%\hline
%\end{tabular}}
%\end{center}
%\end{table}


A further refinement will be used 
for the two lowest-$Q^2$ kinematic points,
to reduce systematics.  For 
%those kinematics, especially 
the $Q^2$=3.5 and 4.5 (GeV/c)$^2$ points, the combined 
elastic acceptance would be largely determined by
BigHAND if the full BigBite acceptance shown in Fig.
\ref{Theta-Phi-accept} were used.   
So elastic events selected by such a fiducial would have
scattered nucleons extending to the edge of the BigHAND acceptance.
With the addition of Fermi motion, the quasi-elastic events may be
expected to have a larger loss, by falling outside the BigHAND
acceptance, than would be the case at the other kinematic points.
Monte Carlo simulation (described in section \ref{QE-simulation})
confirms that the acceptance losses are larger at these kinematics,
amounting to 14.4\% and 9.4\% for the lower and higher Q$^2$,
respectively.  To the extent that these acceptance
losses are equal for protons and neutrons, they cancel in the ratio
$R''$ (Eqn. \ref{r-double-prime-eqn}).  Systematic errors may arise,
however, due to difference in the acceptance-corrections resulting from
effects such as horizontal steering by the dipole's field, 
multiple-scattering, or edge effects in nucleon detection efficiency.
It is therefore prudent to reduce the acceptance corrections where
practical.  This can be effected by choosing a new smaller fiducial, 
based on BigBite measurements, by demanding that the $\vec q$ vector
point towards a 
further-reduced portion of the BigHAND face.   A margin of
safety, $d$, is excluded around all edges.
This defines a smaller 'active' area of BigHAND to be used in defining
the fiducial cut.
%This is represented graphically in
%Fig. \ref{BB-with-BH-superimposed} which shows the BigBite acceptance
%(in local coordinates) and superimposed images of the trajectories
%which would give elastic events at the edge of BigHAND (green) or at the edge
%of such a safety margin(red).  
If $d$ is chosen as $d=\frac{\delta}{p_N}L$
then, to first order, quasi-elastic coincidences with the electron
within the fiducial cut will be lost only if the struck nucleon had a
component of momentum of at least $\delta$ directed towards the edge
of BigHAND.  
%Table \ref{coincidence-prob}
%gives the resulting fractional acceptance, $P_{coinc}$, in BigHAND 
%for various choice of $\delta$.
In addition, to match neutron and proton acceptances, events are
 rejected if {\em either} a neutron or a proton with the corresponding 
$\vec{q}$ would pass near the top or bottom of BigHAND, within a
 distance corresponding to $\delta=100$ MeV/c.
 The cost of applying this tighter
fiducial cut is a smaller fractional acceptance for quasi-elastically
scattered electrons.
%, also summarized in Table \ref{coincidence-prob}.
The rate calculations in this proposal are based upon fiducial cuts
corresponding to $\delta=80$ MeV/c for $Q^2=$3.5 (GeV/c)$^2$ and 
$\delta=40$ MeV/c for the  $Q^2=$4.5.
With these cuts, the
estimated loss of acceptance due to Fermi motion will be less than 5\%
at all kinematic points.  This limits the potential for systematic errors
due to the difference in the loss of protons compared to neutrons.
The cost of these cuts is a reduction by 60\% in the counting rate at 
the lowest $Q^2$ point and by 17\% at  $Q^2=$4.5 (GeV/c)$^2$.
Since the count rates are very high at these kinematic points, the
loss of acceptance is justified by the decreased sensitivity to
systematic error.


%\begin{figure}
%%too many dots!!
%\includegraphics[width=3.in]{figures/BBwK1.jpg}
%\includegraphics[width=3.in]{figures/BBwK2.jpg}\\
%\includegraphics[width=3.in]{figures/BBwK3.jpg}
%\includegraphics[width=3.in]{figures/BBwK4.jpg}\\
%\caption{\label{BB-with-BH-superimposed}
%The acceptance of BigBite (rectangle) in terms of the horizontal ($y$)
%and vertical ($x$) position of a track at the entrance.  Also shown is
%the locus of elastically scattered electrons which would result in a
%nucleon a) at the edge of the BigHAND acceptance (green) and b) at the edge
%of the reduced acceptance corresponding to a safety margin at the edge
%of BigHAND (red).  The part of the BigBite acceptance within this region
%corresponds to the fiducial cut for elastic kinematics.  Dots show
%position at entrance of BigBite for simulated quasi-elastic electrons which
%satisfy the fiducial cut.
%The four plots
%correspond to four kinematic points (see Table \ref{Kin_table}): Upper
%left, $Q^2=3.5$, upper right, $Q^2=4.5$, lower left, $Q^2=6.0$, lower
%right, $Q^2=8.0$ (GeV/c)$^2$
%}
%\end{figure}



%Table \ref{final-acceptance} gives the combined solid angle
%$\Delta\Omega$ of BigBite
%and BigHAND for elastic events at several kinematic points.
%Also given (based on the quasi-elastic Monte Carlo described in section
%\ref{QE-simulation}) are  $f_1$, the fraction of events for which the electron
%and nucleon still fall into the combined acceptance once Fermi
%motion is included and $f_2$, the fraction of those events which would
%pass the fiducial cut.  The product $f_1 f_2 \Delta \Omega$ can be
%treated as an effective solid angle for accepting quasi-elastic events
%with the fiducial cut.  This is intended only to give an estimate of
%the size of these separate effects.  In final rate estimates, these
%effects are not treated as separate factors but are included
%event-by-event in the Monte Carlo integration of cross section over 
%acceptance.  For convenience, the last line of
%Table \ref{final-acceptance}  summarizes the
%motivation for the fiducial cuts by giving the 
%the fraction, $P_{coinc}$, of quasi-elastic events within the fiducial cut for 
%which the scattered nucleon also falls
%within the BigHAND acceptance.  The resulting acceptance fraction $P_{coinc}$
%are seen to be near unity and are expected to be nearly equal for neutrons
%and protons.  ($P_{coinc}$ is also given in Table
%\ref{coincidence-prob} but here it is given for the selected fiducial cuts.)


%\begin{table}
%\begin{center}
%\caption{Combined acceptance of BigBite and BigHAND for elastic
%events and for quasi-elastic events with fiducial cut.  The quantities
%$f_1$ and $f_2$ are defined in the text.
%\label{final-acceptance}}
%\vspace{.2in}
%{\begin{tabular}{|l|l|l|l|l|}
%\hline
%$Q^2$ (GeV/c)$^2$&3.5& 4.5& 6.0&8.0\\
%\hline
%$\Delta\Omega_{el}$ (msr)&39.6&53.6&53.6&53.2\\
%$f_1$ (\%)&78.6&87.2&92.0&96.3\\
%$f_2$ (\%)&28.5&65.7&87.8&98.8\\
%$\Delta\Omega_{eff}=f_1f_2\Delta\Omega_{el}$ (mSr)&8.86&32.1&43.1&50.6\\
%\hline
%\hline
%$P_{coinc} $ (\%)&96.0&95.3&95.5&97.5\\
%\hline
%\end{tabular}}
%\end{center}
%\end{table}


\section{Nucleon Detection Efficiency Calibration}
While the efficiency of electron-detection cancels in the ratio,
$R''$(equation \ref{r-double-prime-eqn}), that is not true for the neutron or proton detection
efficiencies.  The efficiency of BigHAND's detection of these
particles could be calculated in Monte Carlo, as has been done for
lower energies\cite{BH_simulation}, but a reliable determination of
the efficiency at the percent level will require calibration with
tagged sources of known protons and neutrons over part of the
kinematic range.  As shown in Fig. \ref{np_eff}, the efficiency is
large and quite constant for Q$^2 > 8$ (GeV/c)$^2$.  Below that,
however it changes rapidly as a function of nucleon momentum.  In that
region, 
the calculation of efficiency is likely to be more sensitive to the
details of the detector description.  It is important to have reliable
efficiency calibrations there.  

Fortunately the
efficiency may be expected to be quite stable since it is largely
determined by the mass distribution in the detector and the resulting
probability of hadronic shower initiation.  Factors, such as gain,
threshold, and light yield have a relatively minor effect since most showers
produce large numbers of secondaries and so 
their total light output is well above threshold.
This stability is demonstrated in Fig. \ref{np_ratio} which shows
the {\em ratio} of the efficiency found \cite{BH_simulation}  
for a threshold of 20 MeV (electron equivalent) to that found with a
threshold of only 5 MeV (electron equivalent).  Even at the lowest $Q^2$ of
interest, 3.5 (GeV/c)$^2$, such a huge change in threshold would result in only
$\approx 10$\% change in efficiency.
This is in contrast to detection of
low energy neutrons in scintillator, for which the detection
efficiency is a strong function of effective threshold.  Furthermore,
since BigHAND's detection is based purely on scintillation, it is
immune to the rapid changes in efficiency and background which can
more typically occur in wire chambers.
Because the efficiency will be stable, it is sufficient to have separate
calibration runs, rather than simultaneously taking calibration data
by having the calibration target in place
along with the deuterium target.  

\begin{figure}
\begin{center}
\includegraphics[width=3.5 in]{np_ratio.pdf}\\
\end{center}
\caption{\label{np_ratio}
The predicted \cite{BH_simulation}  {\em ratio} of the detection efficiency 
of BigHAND with a threshold of 20 MeV(ee) compared to a threshold of 5
MeV(ee) is shown for neutrons
(blue squares) and protons (red circles) as a function of the central nucleon
momentum associated with each $Q^2$.  (Displacement of individual
points from the overall curve is due to finite
Monte Carlo statistics in the simulation.)
}
\end{figure}

Efficiency measurements, as described below, are planned for the three
lowest kinematic settings, Q$^2 = 3.5$, 4.5 and 6.0 (GeV/c)$^2$. In
addition, the analysis of this data may benefit from calibration
measurements planned as part of the proposed GEN experiment
\cite{new_GEN} which plans to make similar efficiency measurements at 
Q$^2 = 5$, 7 and 9.5 (GeV/c)$^2$. 

For both protons and neutrons, the basis of the calibrations is a
cleanly identified reaction on the proton which produces a 'tagged'
nucleon with a known momentum vector, (p(e,e$'$)p for proton
calibration and p($\gamma$,$\pi^+$)n for neutron calibration).  
BigHAND is then searched for
the nucleon near the expected position and the detection efficiency is
determined as the ratio of the number of tagged events for which the
nucleon is found to the total number of tagged events.

\subsection{BigHAND Calibration Coverage}

%As will be seen below, the neutron calibration, like the proton 
%calibration will be done with essentially elastic-scattering kinematics.
The $p(e,e'p)$ proton-calibration reaction and the $p(\gamma,\pi^+n)$
neutron calibration reaction have essentially the same
(elastic-scattering) kinematics.
A disadvantage of the use of elastic kinematics for calibration is
that the acceptance of BigBite for electrons maps into a well-defined
corresponding angular range on the BigHAND detector face 
(see Fig. \ref{calzone1}).  Regions
beyond that, which would not be illuminated by calibration particles, may
still be of interest for quasi-elastic events since the particles are
then smeared beyond the strict elastic kinematic boundaries by the nucleon's
initial momentum.  To reduce the impact of the larger `footprint' of
quasi-elastic events, calibration data will
be taken at two positions of the BigBite spectrometer so the face of
BigHAND will be more widely illuminated by calibration nucleons.
This expanded calibration region is illustrated in Fig. \ref{calzone2}
which also shows the distribution of quasi-elastic nucleons coincident
with electrons within the fiducial region selected by  BigBite.
(The simulation is described in section \ref{QE-simulation}.) Table
\ref{calzone-table} shows the predicted fraction of quasi-elastic coincidence
events for which the detected nucleon will fall inside this
 calibration region. While the use of two calibration positions is not
 required at the Q$^2$=3.5 (GeV/c)$^2$ point, it affords an
 excellent opportunity to illuminate the entire BigHAND face with
 calibration events.  For the Q$^2$=6.0 (GeV/c)$^2$ point, further 
improvement could be made in the fraction of quasielastic events within
the calibrated region by using three BigBite positions (91.5\%
calibrated) or even four BigBite positions (93.1\% calibrated) but 
the results obtained with two positions (83.4\% calibrated) are
sufficient.  The extrapolation of the calibration at this setting will
be included in the systematic error analysis.
 As described below, extrapolation of lower-momentum 
calibration data will also be used to estimate efficiencies for those
events which are outside the elastic-calibration region at the
momentum of interest.

Calibration measurements will be made only at the three kinematic points
shown in Fig. \ref{calzone2} and listed in Table \ref{calzone-table}.
The systematic errors due to the extrapolation of the
nucleon detection efficiencies can be conservatively estimated to
be under 2\%.
 The hadronic cross sections change quite slowly in the
region in which these interpolations will be made (above the nucleon
momenta indicated by arrows in  Fig.\ref{pdg-sigmann}), as a result of which the estimated
efficiencies shown in Fig. \ref{np_eff} change only moderately in the
regions of interest.  The method of interpolation is described in the next paragraph.

\begin{table}
\begin{center}
\caption{
Fraction of detected quasi-elastic nucleons 
which fall inside the region calibrated
(at the same kinematic point) with a) a single BigBite position for
calibration or b) two off-set calibration positions.
\label{calzone-table}}
\vspace{.2in}
{\begin{tabular}{|l||l|l|}
\hline
$Q^2$&a) Fraction (\%) in&b) Fraction (\%) in\\
(GeV/c)$^2$&Single Cal. Zone&Double Cal. Zone\\
\hline
3.5& 97.7  &100.\\
4.5& 82.3  &99.7\\
6.0& 56.9  &83.4\\
\hline
\end{tabular}}
\end{center}
\end{table}


\begin{figure}
\begin{center}
\includegraphics[width=3.2in]{figures/bhek32cz.jpg}\hfill
\includegraphics[width=3.2in]{figures/bhek30cz.jpg}\\

\vspace{.35in}
\includegraphics[width=3.2in]{figures/bhek36cz.jpg}

\vspace{.35in}
\caption{\label{calzone1}
The face of the BigHAND detector is shown (rectangle) with a
superimposed blue outline of the region covered by elastically scattered
(calibration) nucleons for which the corresponding electron falls within the
acceptance of BigBite when BigBite is positioned at the
nominal scattering angle.  For clarity the image for neutrons and
protons are shown side-by-side rather than superimposed.  
Note that positive-x is downward, so the images are inverted.
 Dots indicate
positions of neutron (black) or proton (red) hits from simulated
quasi-elastic events subject to the fiducial cut.
The three plots
correspond to the kinematic points at which calibration measurements
will be made: 
Top left, $Q^2=3.5$; Top right, $Q^2=4.5$; Bottom, $Q^2=6.0$
(GeV/c)$^2$.}
\end{center}
\end{figure}


\begin{figure}
\begin{center}
\includegraphics[width=3.2in]{figures/bhek32cz2.jpg}\hfill
\includegraphics[width=3.2in]{figures/bhek30cz2.jpg}\\

\vspace{.35in}
\includegraphics[width=3.2in]{figures/bhek36cz2.jpg}

\vspace{.35in}
% too many dots!!
%\includegraphics[width=2.8in]{figures/bhek32cz2.jpg}
%\includegraphics[width=2.8in]{figures/BHwK2.jpg}\\
%\includegraphics[width=2.8in]{figures/BHwK3.jpg}
%\includegraphics[width=2.8in]{figures/BHwK4.jpg}\\
%\includegraphics[width=6in,angle=-90]{figures/framescalzones.eps}
\caption{\label{calzone2}
The face of the BigHAND detector is shown (rectangle) with a
superimposed blue outline of the region covered by elastically scattered
(calibration) nucleons for which the corresponding electron falls within the
acceptance of BigBite when BigBite is positioned at two planned
calibration angles.  For clarity the images for neutrons and for
protons are shown side-by-side rather than superimposed.  
Note that positive-x is downward, so the images are inverted.
 Dots indicate
positions of neutron (black) or proton (red) hits from simulated
quasi-elastic events subject to the fiducial cut.
The three plots
correspond to the kinematic points at which calibration measurements
will be made: 
Top left, $Q^2=3.5$; Top right, $Q^2=4.5$; Bottom, $Q^2=6.0$
(GeV/c)$^2$.}
\end{center}
\end{figure}



Those 
regions of BigHAND which are not directly calibrated at a particular nucleon
momentum will be illuminated by the lower momentum
calibration nucleons at other kinematic points.  In this case
Monte Carlo simulation, tuned using the lower energy measurements, will
be used to extrapolate the efficiency calibration to the momentum of
interest.  Essentially, the observed efficiency at lower momentum can
be used to determine an effective number of interaction lengths in
sub-regions of the detector.  The effective thickness can then be
scaled to higher momentum by taking into account the known variation
of hadronic cross-sections with momentum and a new efficiency can be 
predicted.  This is a modest extrapolation
since the hadronic cross sections change slowly as the nucleon
momentum changes over the range from 2.6 to 10.5 GeV/c (see Fig.
\ref{pdg-sigmann}).
This results in the slow variation in BigHAND's efficiency seen in
Fig. \ref{np_eff}.  The efficiency variation is expected to be quite modest
over the entire range of interest ($3.5\le Q^2\le 18.0$ (GeV/c)$^2$)
and to be almost constant at the highest $Q^2$.
%Table \ref{calzone-table} shows that 
%$\approx$95\% or more of the quasielastic events are expected to 
%fall within the calibrated region except at the $Q^2=$8.0 (GeV/c)$^2$
%point for which 11.4\% of events will require the use of an
%extrapolated calibration.  
Since Fig. \ref{np_eff} shows the
efficiency to be changing very slowly in this kinematic range
(Fig.  \ref{pdg-sigmann} shows the nucleon cross-section 
to be essentially flat for the corresponding
momentum ($p_N>$4.3 GeV/c).), this extrapolation is expected to have
small uncertainty. 
%The same technique will be used to interpolate
%efficiencies for the two kinematic points for which calibration data
%will not be taken.  The effective thickness will be determined for
%both the higher and lower calibration points, and will be
%interpolated. 
Despite the large size of BigHAND, a portion (one tenth)
of its structure could be brought to the tagged neutron
beam at Dubna for calibration.  This could  provide a cross check of the 
Monte Carlo simulation of momentum-dependence of efficiency at high
nucleon momentum.

%This is confirmed in Fig. \ref{np_eff} which
%shows simulated neutron and proton efficiencies to change slowly and
%to be almost constant at $Q^2=8$ (GeV/c)$^2$.

\begin{figure}
\begin{center}
\includegraphics[width=4in]{figures/pdg.pdf}
\end{center}
\caption{\label{pdg-sigmann}
Total cross section \cite{PDG} for proton-proton (red) 
and proton-neutron(blue).  Some
points with large errors have been suppressed.  Green arrows indicate
the central nucleon momenta for the three kinematic points at which
calibration data will be taken.  }
\end{figure}

%\begin{figure}
%\includegraphics[height=6in,angle=-90]{np_eff.pdf}
%\caption{\label{np_eff}
%Simulated efficiency of BigHAND for thresholds of 5 or 20 MeV
%(electron equivalent).}
%\end{figure}

\subsection{Proton Calibration}

Calibration of the detection efficiency for protons can be done
straightforwardly, using a hydrogen target in place of the deuterium
target and using BigBite to select elastic-scattering events.  The
recoil elastic protons have the same momentum as the protons at the
center of the quasi-elastic peak, and so are ideal for calibrating the
efficiency.  
%This kinematic match is shown in Fig. \ref{momdist_a} and
%\ref{momdist_b} where
%BigHAND is divided into strips of polar angle and the momentum spectrum of
%quasi-elastic nucleons in each strip is overlaid with the spectrum of
%calibration nucleons.

%\begin{figure}
%\begin{center}
%\includegraphics[width=5in]{figures/momdistr1.pdf}
%\end{center}
%\caption{\label{momdist_a}
%For the lowest $Q^2$ kinematic point, the kinematic overlap of the
%(elastic-kinematics) calibration nucleons with the quasi-elastic
%nucleons of interest is shown.  The BigHAND acceptance is divided into
%ten strips in polar angle.  Each plot represents the nucleon momentum spectrum
%falling on a strip (blue) and the momentum spectrum of calibration
%nucleons reaching that strip (red).  Note that zero is suppressed on
%the horizontal scale.  Horizontal scale on each plot is 2.4 to 3.0
%GeV/c. The width of the quasi-elastic nucleon momentum
%distribution is seen to be modest and reasonably 
%centered on the calibration data.
%}
%\end{figure}


%\begin{figure}
%\begin{center}
%\includegraphics[width=6in]{figures/momdistr4.pdf}
%\end{center}
%\caption{\label{momdist_b}
%Same as \ref{momdist_a}, but for the highest $Q^2$ kinematic point.
%Horizontal scale on each plot is 4.6 to 5.6 GeV/c.
%}
%\end{figure}


Clean selection of tagged protons requires that BigBite
be able to distinguish elastic events with negligible
contamination from inelastic events, for which the 'tagged' nucleon
might be absent or at a significantly different angle than
predicted.  Figure \ref{fig-elastic-inelastic-kin} shows the
kinematic locus for elastic scattering kinematics (in the region of each of the
highest-Q$^2$ kinematic point for which calibration measurements are
planned) and the locus of the scattered-electron
kinematics for pion-production threshold.  The two are seen to be
cleanly separated in both angle and energy.  The bars superimposed on
the plots indicate the anticipated resolution of BigBite in angle and energy.

As discussed above, the coverage of the calibration will be
extended across a wider region of BigHAND by taking calibration data
in two BigBite positions, as represented in Fig. \ref{calzone2}.


\begin{figure}
\begin{center}
%\includegraphics[width=2.8in]{figures/single_arm_1.jpg}
%\includegraphics[width=2.8in]{figures/single_arm_2.jpg}\\
%\includegraphics[width=2.8in]{figures/single_arm_3.jpg}
%\includegraphics[width=3.4in]{figures/single_arm_4.jpg}\\
\includegraphics[width=3.4in]{figures/single_arm_36.jpg}\\
\end{center}
\caption{\label{fig-elastic-inelastic-kin}
%\vspace{.25 in}
Scattered electron energy versus angle for elastic scattering from the
nucleon (green) and for pion-production threshold (red).  The bars
indicate the BigBite resolution in momentum and angle.  
The plot shows the clean kinematic separation at the $Q^2=6.0$
(GeV/c)$^2$ kinematic point.  The separation is even cleaner at the other
kinematics for which calibration measurements are planned.}
\end{figure}



\subsection{Neutron Calibration}

\label{n-cal}
A clean {\it in situ} source of tagged neutrons at essentially 
elastic-scattering kinematics 
 can be generated using a radiator to produce real photons
which, in turn,  produce neutrons through the p($\gamma,\pi^+$)n reaction on a
Hydrogen target.  The calibration for each kinematic setting 
will be done at the same two BigBite
settings discussed above for proton efficiency calibration.
Since the kinematics of the reaction are essentially those of elastic
scattering, the overlap of the calibration neutron spectrum with the
quasielastic spectrum is the same.
% as that shown in  Fig. \ref{momdist_a} and \ref{momdist_b}.
  
The end-point method can be used to select the
reaction of interest and exclude multi-body final states.  In
particular, for each beam energy, the maximum possible $\pi^+$
momentum for three-body p($\gamma,\pi^+$)$\pi$N reaction 
sets the scale of the lowest $\pi^+$
momentum which can safely be used to select the desired two-body channel.
Clear identification of the exclusive p($\gamma,\pi^+$)n reaction will
be ensured by requiring that the $\pi^+$ momentum be at least 1.5\%
above the kinematic limit for the three-body channel. 
With 0.5\% momentum resolution in BigBite, the 1.5\% gap between the
lowest $\pi^+$ momentum used and the highest momentum from
$2\pi$-production provides an adequate safety margin. Production of
$2\pi$ background at this kinematic limit will be negligible both
because it requires photons at the bremsstrahlung end point and
because the phase-space vanishes.  This minimum
acceptable $\pi^+$ momentum, in turn sets the limit for the minimum
usable photon energy and the part of the bremsstrahlung end-point
region which will actually be used for calibration events.  Table \ref{brem-end-point} shows
the region of the photon end point which can be used for neutron
calibration at each kinematic setting.  For the lowest $Q^2$ point,
for example, the three-body reaction cannot produce pions above 2.453
GeV.  For safety, pions will be identified as coming from the two-body
calibration reaction only if they lie between 2.494 GeV and the
kinematic limit of 2.54 GeV.


\begin{table}
\begin{center}
\caption{Kinematics limiting the part of bremsstrahlung spectrum which
can be used for neutron efficiency calibration. \label{brem-end-point}}
\vspace{.2in}
{\begin{tabular}{|l|l|l|l|l|l|l|l|}
\hline
Q$^2$ & E$_{\mbox{beam}}$ & 
$\theta_e $&
$E^{max}_{\pi}$&
$E^{max}_{\pi}$&
$E^{limit}_{\pi}$&
$E^{min}_\gamma$ &
$\int ~\Gamma ~dk$\\
 & & & $(\gamma,\pi)$ &$(\gamma,2\pi)$ &$(\gamma,\pi)$ & & \\
(GeV/c)$^2$ & (GeV) & &
 (GeV) &
 (GeV) &
 (GeV) &
 (GeV) &
\\
\hline
3.5 & 4.4& 32.5$^\circ$& 2.54 & 2.453 & 2.494 & 4.25 & 0.0023\\
4.5 & 4.4& 41.9$^\circ$& 2.00 & 1.928 & 1.96 & 4.21 & 0.0030\\
%6.0 & 4.4& 77.8$^\circ$ &  0.936& 0.904 & 0.92 & 4.02 & 0.00641\\
6.0 & 4.4& 64.3$^\circ$ &  1.20& 1.16 & 1.18 & 4.12 & 0.0046\\
\hline
\end{tabular}}\\
\end{center}
{For each kinematic point, the values given are: 
$E^{max}_{\pi}(\gamma,\pi)$, the end-point energy for $\pi$ production; 
$E^{max}_{\pi}(\gamma,2\pi)$, the $\pi$ end-point energy for $2\pi$ production; 
$E^{limit}_{\pi}(\gamma,\pi)$, the minimum $\pi$ energy to be used
for calibration (to exclude  $E^{max}_{\pi}(\gamma,2\pi)$ by 1.5\%);
$E^{min}_\gamma$, minimum photon energy giving $E_{\pi}(\gamma,\pi)$
above $E^{limit}_{\pi}(\gamma,\pi)$;
$\int ~\Gamma ~dk$, the photon flux (per electron) integrated
\cite{matthews}  from  
$E^{min}_\gamma$ to the photon end point (assuming $6\%$ Cu radiator).
}
\end{table}



The real photons used for the p($\gamma,\pi^+$)n calibration reaction
will be generated using a copper 6\%  radiator upstream of the LH$_2$
target.  Maintaining the same limit on rate in BigBite then requires that
the beam current be reduced by a factor of four for these
measurements.  While real photons interacting with well-defined 2-body
kinematics will dominate, roughly one quarter of the rate is expected to
originate from virtual-photon interactions.  The virtual-photon
contribution can be studied (along with the corresponding distribution
of neutrons in BigHAND) by running without a radiator and 
measuring neutrons associated with the
p(e,$\pi^+$) reaction.  Because this is a relatively small correction,
allocation of beam time would be optimized to yield the minimal error
on the real-photon calibration by spending only a fraction as much
beam time on running without a radiator.  (Taking the virtual and real
cross sections to be equal and the neutron acceptance and
efficiencies to be as large for virtual production implies that
the error bar would be minimized by dedicating 20$\%$ of the
neutron-calibration time to running without a radiator.)  
%Since the
%same running conditions, LH$_2$ target without a radiator, are used
%for the proton efficiency calibration, the data needed to subtract the
%real photon contribution to the neutron calibration is partly available 
%parasitically from those efficiency calibrations.

\section{Simulations}

At high $Q^2$ the kinematic separation of quasi-elastic and inelastic
events becomes more washed-out by the kinematic-broadening effects of 
Fermi motion in the deuteron.  Also the size of the inelastic
cross section relative to the quasi-elastic grows rapidly with
increasing $Q^2$.  A simulation is needed to determine whether there
is a serious problem with contamination of the quasi-elastic coincidence
signal by inelastic events for which a nucleon accidentally hits
the BigHAND near where a quasi-elastic nucleon would be expected.

This section will present the technical details of the implementation
of the simulations, including the normalization of the inelastic
spectrum relative to the quasi-elastic.  The next section will present
the results of the simulations.

\subsection{Quasi-elastic}

\label{QE-simulation}
Simulation of the quasi-elastic signal was carried out in a spectator model
in which the virtual photon was assumed to interact with only one nucleon
while the other simply escaped.  This implies that the spectator
nucleon is projected into `on-shell' kinematics by the interaction
(with whatever initial momentum it has) and so the initial off-shell
mass of the struck nucleon is determined by the requirements of energy
conservation.  While off-shell effects were included at the kinematic
level, no attempt was made to modify the electron-nucleon scattering
cross section to reflect the off-shell nature of the struck nucleon.
For these simple signal-to-noise estimates, the scaled dipole
form factors were assumed for the nucleons (except the Galster
parameterization was used for the neutron electric form factor).
As described below, kinematic effects of the initial motion of 
the nucleon were reflected by
calculating the cross section based on the electron energy and
scattering angle as determined in the rest frame of the scattered
nucleon. 

The momentum distribution of nucleons was taken from the
momentum-space wave-function (non-relativistic Fourier transform of
spatial wave-function) for a Lomon and Feshbach deuteron
potential. The particular model used (\#10 from reference
\cite{Lomon}) gave 5.79\% D-state and included a hard core. The hard
core is reflected in a high-momentum tail in the momentum-space
wave-function making this a somewhat 'worst case' simulation.

In brief, the steps of each quasi-elastic event simulation are summarized
here.  A Fermi momentum was chosen for the struck nucleon based on the
probability distribution for magnitude of $p$ derived from the
deuteron wave-function.  A direction was chosen isotropically.
  The corresponding kinetic energy of the the on-shell
spectator was subtracted from the deuteron mass to find the
kinematically-consistent initial energy of the struck nucleon from
which the (off-shell) invariant mass of the struck nucleon was found.
The beam electron was then rotated and boosted to a frame in which the
nucleon was at rest (and the electron was rotated back onto the
z-axis).  A scattering angle was then chosen isotropically (flat in
$\phi$ and in $\cos\theta$) and the elastic scattering cross section
was evaluated for these rest-frame initial kinematics.  The scattered
energy was calculated with the added requirement that the off-shell
initial-state nucleon be promoted to an on-shell nucleon in the final
state. Some 'sanity cuts'
were applied to eliminate extreme cases in which the quasi-elastic
model was clearly pushed beyond the range of applicability, such as
cases with off-shell invariant mass of the struck nucleon being less
than 10\% of the nucleon mass or final electron energy being unphysical.
All boosts and
rotations were then inverted on the final-state particles to return
them to the lab frame.  No kinematic weighting was done on the 
distributions of Fermi momentum or scattering angle to reflect the
greater probability of scattering at small angles and at kinematics
which lead to lower electron energy in the nucleon rest frame.  The
higher weighting of these events was reflected in the calculated cross
section, which was then used to weight the entries made to the
final-state distributions.  Thus the simulation not only properly
accounted for these kinematic effects, it also resulted in properly
normalized cross sections for the simulated reaction. To select the
quasi-elastic events of interest, an acceptance cut was imposed to
require that the electron fell into the BigBite acceptance while the
scattered nucleon fell into the BigHAND acceptance.  The effects of
finite-resolution were then incorporated by smearing the final
electron-energy angle and energy by Gaussian distributions to simulate
the BigBite resolution and similarly smearing the detected nucleon
angle to reflect the BigHAND resolution.  
 The resulting quantities were
then used to calculate $W^2$ and $\theta_{pq}$
and the  simulated distributions
of the quantities of interest were incremented, weighted by cross section.
Here $\theta_{pq}$, introduced above, is the 
angle between the calculated $\vec{q}$ 
direction and the observed scattered nucleon direction while $W^2$ is
the squared missing-mass of the hadronic system as calculated assuming
a stationary proton target (i.e. $W^2=(m_p+\omega)^2-(\vec{q})^2=
m_p^2+2m_p\omega-Q^2$).


\subsection{Inelastic}

The term ``inelastic'' is used here to imply particle production
and is exclusive of quasi-elastic events.
Simulation of inelastic events required a more sophisticated model for
the basic interaction on the nucleon.  This was done with the use of
the  Genev physics Monte-Carlo \cite{genev} written by the Genoa
group and used extensively in simulations for CLAS.  This program is
designed to simulate with, reasonable empirical distributions,
production of multi-pion final states and production and decay of
Delta's, rho mesons, and omega meson (phi meson production was not
enabled when the simulations were run).  It can simulate 
neutron or proton targets and both were used in simulating
inelastic events from the deuteron.  


The smearing effects of Fermi motion for quasi-free inelastic
production from the nucleons in the deuteron were included in a
similar way to that described above for the quasi-elastic production.  
There was, however, no mechanism to
put the initial-state nucleon off-shell for the initial state used by
Genev.  The spectator model was therefore implemented by treating the
initial state as two on-shell nucleons with equal and opposite
Fermi-momentum (in the deuteron rest frame).  The effective violation
of conservation of energy implied by this approximation is modest
(a few tens of MeV) and is expected to have the effect of
widening tails and so causing backgrounds to be over-estimated if anything.


Final-state distributions were simulated separately for
electro-production off the neutron and proton.
The same momentum-state wave-function was used to generate the initial
momentum distribution of the target nucleons.  After rotating and
boosting to the nucleon rest frame the energy of the incident electron
was passed to a Genev-based subroutine which simulated a single
inelastic event for the chosen effective beam energy.  The scattered
electron direction was selected randomly (both $\theta$ and azimuthal
angle, $\phi$) 
by Genev based on cross-section-weighting subject to constraints on
$W^2$ and $Q^2$, discussed below.  The predicted final-state particles
were then boosted and rotated back to the lab frame by reversing all
boosts and rotations done to the initial-state particles.  In order to
make effective use of simulated events without biasing 
distributions, those events which had
a final electron azimuthal angle outside the range of
$-20^\circ$ to $+20^\circ$ were rotated about the beam direction by an
angle chosen to give a final azimuthal angle chosen randomly within
that range.  This enhanced the yield of events within the BigBite
acceptance but didn't affect distributions which had, at minimum, a
requirement of a hit in BigBite.

The range of $Q^2$ and $W^2$ to be generated by Genev was selected
empirically since the effects of Fermi motion made it difficult to
predict the significant range {\it a priori}.  A low-statistics run of
the simulation with a broad range was subjected to the acceptance cut
of BigBite for the angle(s) of interest for the beam energy being
simulated.  The resulting $Q^2$ and $W^2$ distribution showed clear
peaks in the regions which were relevant for scattering into BigBite.
High statistics runs were then done with those ranges selected for generation
of Genev events.

Full kinematic information was written out for each event (including
particle identification for each four-vector).  These were then
selected to produce samples of interest for each kinematic point which
would have an electron within the BigBite acceptance.  The effects of
finite detector resolution were folded in (by smearing of
each four-vector) before calculation of kinematic quantities of
interest such as $W^2$ and $\theta_{pq}$. 




\subsection{Inelastic Background Normalization}

A fundamental difference between the quasi-elastic and inelastic
simulations is that the inelastic simulation produced simulated events
without a corresponding cross section by which to weight them.  While
the relative cross sections were accounted for in the probability of
generation of different types and topologies of events, an overall
normalization is needed to allow comparison of the inelastic events
(from each target nucleon) with the quasi-elastic results.  

Normalization of inelastic to quasi-elastic cross sections was done
empirically, using SLAC spectra for single-arm electron scattering
from the deuteron. Figure \ref{SLAC_data} shows spectra from
\cite{SLAC_Stuart} and \cite{SLAC_Rock} used for the normalization.
The kinematic coverage of those measurements (truncated to the $W^2$
range of relevance) is shown in Fig. \ref{SLAC_kin} with the same
colors as in Fig. \ref{SLAC_data} to distinguish the two data sets.
Two conveniently chosen ranges were used to characterize the cross
sections in the quasi-elastic and inelastic regions.  These are
indicated in Fig. \ref{SLAC_data}  as green bars indicating the limits selected
for the 
``Quasi-elastic region'' ($0.5<W^2<0.88$ GeV$^2$) and red bars
indicating the limits selected to define the ``Inelastic region''  ($1.3<W^2<1.7$ GeV$^2$).
As can be seen from the figures, the regions were chosen to give
samples which were almost purely representative of the indicated final
state, without significant contamination of inelastic events in the
Quasi-elastic region or {\it vice versa}.  The Inelastic region was
also chosen close to the quasi-elastic peak so it would be
representative of the events which would be likely to cause
background.

\begin{figure}
\includegraphics[width=6in]{figures/fig_slac.pdf}
\caption{\label{SLAC_data}Measured single-arm spectra from SLAC
covering the quasi-elastic and inelastic regions.  Blue points were
taken at E=5.507 GeV and indicated angle (in degrees).  Magenta points were taken
at indicated energy and $\theta=10^\circ$.}
\end{figure}

\begin{figure}
\includegraphics[width=6in]{figures/fig_slac_kin.pdf}
\caption{\label{SLAC_kin}The $Q^2$ vs. $W^2$ coverage corresponding to
the spectra of the previous figure.
}
\end{figure}

Within the quasi-elastic model, the kinematic variation of
cross-section within the Quasi-elastic region would be expected to follow
the sum of the elastic cross sections for scattering from the proton 
and neutron.  The numerically summed cross section in the Quasi-elastic
region of each spectrum was divided by the predicted sum of
proton-elastic and neutron-elastic (based on scaled-dipole and
Galster) to obtain a measured strength (which was quite stable at a
value of $\approx 0.35$).  Similarly, based on
$$\frac{d^2\sigma}{d\Omega dE'}=\frac{\alpha E' (W^2-m_p^2)/(2m_p)}
{4\pi^2 Q^2E}\frac{2}{1-\epsilon}(\sigma_T(W^2,Q^2)+\epsilon\sigma_L(W^2,Q^2))$$
the kinematic factors were divided
out of each bin of the double-differential cross section in the Inelastic region to
yield the corresponding value of ($\sigma_T+\epsilon\sigma_L$).  Since 
the non-resonant background
dominates over $\Delta$ production in these spectra, a non-resonant
empirical scaling of $\sigma_L/\sigma_T\approx 0.25/\sqrt{Q^2}$ was used
to allow $\sigma_T$ (summed over the Inelastic region) to be
extracted for each spectrum.  (An alternate extreme would be to treat
the inelastic cross section as purely transverse, as it might be if
the $\Delta$ dominated.  This was tried and resulted in only a modest
change in the predicted inelastic strength at the the kinematics of
interest.)  The inelastic/quasi elastic strength can then be
characterized as the ratio of the extracted $\sigma_T$ from the
Inelastic region divided by the
scaled cross-section from the Quasi-elastic region.  This ratio (for
both sets of SLAC kinematics)  was
found to be reasonably well parameterized as a simple parabolic function of 
$Q^2$ ($r=0.015 \frac{Q^2}{(\mbox{GeV/c})^2}$).  This form was then used to predict the
inelastic cross section within the Inelastic region for the $Q^2$
applicable for the beam energy and scattering angle of the kinematics
of interest.  These were found as multiples of the summed simulated
cross section in the Quasi-elastic region (divided by the sum of
proton and neutron elastic cross sections).  

This gave the total normalization of the inelastic cross section.
It remained to find individual scaling factors for the proton-target
and neutron-target inelastic cross section to simulate the deuteron
cross section.  The inelastic cross section on the neutron was taken
to be half of the inelastic cross section on the proton.  Since the
final state distributions (including individual measure of 
proton-coincidences or neutron-coincidences) were almost identical for
the two assumed targets, the final results are almost insensitive to
this choice of relative strength.  This then allowed scaling factors
to be determined to scale the number of simulated events to a
double-differential cross section.  These normalized results are shown
in the next section.



%Figure \ref{seamus_nocoinc} compares one example predicted deuterium
%spectrum to our actual spectra \cite{GEn-proposal,Seamus}  on Hydrogen and 
%$^3$He at similar kinematics taken with BigBite.  The beam energy and
%scattering angle of
%the measurements are different (see caption)  but they are at 
%very similar $Q^2$ to
%the lowest proposed kinematic point.  Quantitative comparisons are not possible,
%but the Hydrogen peak is seen to be much narrower than the predicted
%deuterium peak (as expected) while the $^3$He quasi-elastic peak is so
%wide it is not resolved from the inelastic background.  The height of
%the inelastic background (relative to the height of the quasi-elastic
%peak) can be seen to be qualitatively similar indicating, at least,
%that the background is not grossly underestimated in the simulation.

%Similarly, Fig. \ref{seamus_coinc} shows an example predicted deuterium
%spectrum, with a proton coincidence in BigHAND.  This is compared 
%to our actual 
%spectra \cite{GEn-proposal,Seamus}  on Hydrogen and 
%$^3$He at similar kinematics taken with BigBite and with a coincidence
%(neutron or proton) demanded in BigHAND.   The Hydrogen spectrum is
%now seen to be much narrower than the deuterium quasi-elastic peak,
%confirming that instrumental resolution is a small effect compared to
%Fermi smearing.  The inelastic background at $W^2\approx2$ GeV$^2$ in $^3$He
%is seen to be similar to the height of the quasi-elastic peak. 
%Quantitative evaluation of the inelastic-to-quasielastic ratio of the 
%simulation is not possible since the
%conditions are not identical.  The narrower deuterium peak would be
%expected stand out more than the more Fermi-smeared $^3$He peak.
%Also, random-time coincidence have not been subtracted from the $^3$He
%data and, as can be seen from Fig. \ref{seamus_nocoinc}, are expected
%to preferentially fall in the inelastic region.
%The estimated background (relative to quasi-elastic) in the simulation 
%is seen to be comparable to that observed.

A test of the Monte-Carlo at a previously calculated kinematic point 
($Q^2=8$ (GeV/c)$^2$) is shown in Fig. \ref{slac_overlay}.
SLAC data \cite{SLAC_Rock} taken at a similar $Q^2$, but different
beam energy and angle is compared to the Monte-Carlo prediction
for single-arm electron
scattering from  deuterium.  The SLAC data was taken at  E=18.5 GeV
and $\theta=10^\circ$.  It  is compared
to the Monte-Carlo prediction at the same momentum-transfer at E=6.0
GeV and
$\theta=52.0^\circ$.  {\em\bf There are no adjustable parameters.}  The
Monte-Carlo spectrum has simply been scaled up by the ratio of
prediction for the ratio of deuteron quasi-elastic cross sections.
This cross section was predicted (at each set of kinematics) as the
sum of the dipole-prediction cross-section on the proton plus that on
the neutron.  The agreement is seen to be excellent and the inelastic
background is, if anything, overestimated.


%\begin{figure}
%\begin{center}
%\includegraphics[width=3in]{figures/fig_sum1a.jpg}\\
%\includegraphics[width=3.4in]{seamus_H.pdf}\\
%\includegraphics[width=3.4in]{seamus_3He.pdf}
%\end{center}
%\caption{\label{seamus_nocoinc}
%Predicted spectrum on Deuterium (Top) at $Q^2=3.5$ GeV$^2$ (E=4 GeV,
%$\theta=37.5^\circ$) is compared to measured spectra taken with BigBite
%on Hydrogen (Middle) and $^3$He (Bottom) targets 
%at $Q^2=3.4$ GeV$^2$ (E=3.29 GeV,
%$\theta=51.6^\circ$).  All spectra are single-arm spectra with no
%coincidence requirement.
%}
%\end{figure}

%\begin{figure}
%\begin{center}
%\includegraphics[width=3in]{figures/fig_sum1b.jpg}\\
%\includegraphics[width=3.4in]{seamus_H_nucleon.pdf}\\
%\includegraphics[width=3.4in]{seamus_3He_nucleon.pdf}
%\end{center}
%\caption{\label{seamus_coinc}
%Predicted spectrum on Deuterium (Top) at $Q^2=3.5$ GeV$^2$ (E=4 GeV,
%$\theta=37.5^\circ$) is compared to measured spectra taken with BigBite
%on Hydrogen (Middle) and $^3$He (Bottom)  targets 
%at $Q^2=3.4$ GeV$^2$ (E=3.29 GeV,
%$\theta=51.6^\circ$).  The simulated spectrum demands a proton in
%coincidence anywhere in BigHAND.  The real spectra demand a hit in
%BigHAND but do not distinguish between proton and neutron coincidences.
%Accidental coincidences have not been subtracted in the measured spectra.
%}
%\end{figure}


\begin{figure}
\begin{center}
\includegraphics[width=4in]{figures/slac_overlay.pdf}\\
\end{center}
\caption{\label{slac_overlay}
A measured spectrum \cite{SLAC_Rock} for single-arm electron
scattering from  deuterium (E=18.5 GeV, $\theta=10^\circ$) is compared
to the Monte-Carlo prediction at the same momentum-transfer (the
$Q^2=8$ point in the present proposal, at E=6.0 GeV,
$\theta=52.0^\circ$).  The data points are shown in magenta, the
Monte-Carlo prediction in black.  (The quasi-elastic contribution is
shown in red, while the inelastic prediction is in blue.) 
{\em\bf There are no adjustable parameters.}  The
Monte-Carlo spectrum has simply been scaled up by the ratio of
(dipole-approximation) prediction for the ratio of deuteron
quasi-elastic cross sections.
}
\end{figure}




\section{Inelastic Background}

\label{inelastic}
Results from the simulations of inelastic contributions are
shown in Figures \ref{kin1_sim} through \ref{kin4_sim} for a selection
of kinematic points across the range of interest.  Each figure
represents one of the kinematics shown in Table \ref{Kin_table}.  
The simulation was run for each kinematic point but typical example 
results are shown here for the 
for the $Q^2=3.5$, 8.5, 13.5 and 18
(GeV/c)$^2$ kinematic points. 


%In each figure the upper left plot gives the missing-mass spectrum 
%integrated over the acceptance of the experiment. These are single-arm
%spectra such as would be obtained using BigBite alone.    These 
%figures are not of direct relevance
%to the coincidence experiment proposed here.  They are intended only to
%set the scale of how difficult single-arm quasi-elastic measurements
%would be and to demonstrate that the simulated single-arm spectra are
%qualitatively similar to the measurements show in
%Fig. \ref{SLAC_data}.  

In all simulated spectra, the statistical
fluctuations reflect the statistics of the Monte-Carlo simulations
and are not intended to simulate the statistics acquired by the
proposed experiment.

\begin{figure}
\includegraphics[width=3.1in]{figures/fig_sum1a.jpg}\hfill
\includegraphics[width=3.1in]{figures/fig_sum1b.jpg}\\
\vfill
\includegraphics[width=3.1in]{figures/fig_sum1c.jpg}\hfill
\includegraphics[width=3.1in]{figures/fig_sum1d.jpg}\\
\vfill
\caption{\label{kin1_sim} Projections onto $W^2$ and $\theta_{pq}$ 
for simulations of inelastic
background (blue) and quasi-elastic (red) cross section for the $Q^2=3.5$
(GeV/c)$^2$ kinematic point.   Vertical axes are efficiency- and
acceptance-weighted cross section integrated over the combined
spectrometer acceptance, in fb/bin.  The effect of applying the tight fiducial
cuts on the inelastic background is shown by the green curves.
}
\end{figure}


\begin{figure}
\includegraphics[width=3.1in]{figures/fig_sum2a.jpg}\hfill
\includegraphics[width=3.1in]{figures/fig_sum2b.jpg}\\
\vfill
\includegraphics[width=3.1in]{figures/fig_sum2c.jpg}\hfill
\includegraphics[width=3.1in]{figures/fig_sum2d.jpg}\\
\vfill
\caption{\label{kin2_sim}  Projections onto $W^2$ and $\theta_{pq}$ 
for simulations of inelastic
background (blue) and quasi-elastic (red) cross section for the $Q^2=8.5$
(GeV/c)$^2$ kinematic point.   Vertical axes are efficiency- and
acceptance-weighted cross section integrated over the combined
spectrometer acceptance, in fb/bin.
}
\end{figure}


\begin{figure}
\includegraphics[width=3.1in]{figures/fig_sum3a.jpg}\hfill
\includegraphics[width=3.1in]{figures/fig_sum3b.jpg}\\
\vfill
\includegraphics[width=3.1in]{figures/fig_sum3c.jpg}\hfill
\includegraphics[width=3.1in]{figures/fig_sum3d.jpg}\\
\vfill
\caption{\label{kin3_sim} Projections onto $W^2$ and $\theta_{pq}$ 
for simulations of inelastic
background (blue) and quasi-elastic (red) cross section for the $Q^2=13.5$
(GeV/c)$^2$ kinematic point.   Vertical axes are efficiency- and
acceptance-weighted cross section integrated over the combined
spectrometer acceptance, in fb/bin.
}
\end{figure}


\begin{figure}
\includegraphics[width=3.1in]{figures/fig_sum4a.jpg}\hfill
\includegraphics[width=3.1in]{figures/fig_sum4b.jpg}\\
\vfill
\includegraphics[width=3.1in]{figures/fig_sum4c.jpg}\hfill
\includegraphics[width=3.1in]{figures/fig_sum4d.jpg}\\
\vfill
\caption{\label{kin4_sim} Projections onto $W^2$ and $\theta_{pq}$ 
for simulations of inelastic
background (blue) and quasi-elastic (red) cross section for the $Q^2=18$
(GeV/c)$^2$ kinematic point.   Vertical axes are efficiency- and
acceptance-weighted cross section integrated over the combined
spectrometer acceptance, in fb/bin.
}
\end{figure}


The plots on the left in each figure show the neutron-coincident
quasi-elastic (red) and inelastic (blue) spectra and their sum
(black), integrated over the acceptance of the experiment.
The plots
on the right show the equivalent proton-coincident spectra.  
Proton- and neutron- coincidence here, as in the proposed experiment,
are defined based on proximity of the simulated BigHAND hit to the
position which would be predicted based on the $\vec{q}$-vector
constructed based on the resolution-smeared information from BigBite.

The upper plots show the W$^2$ spectra   Kinematic
broadening and the large inelastic cross sections are seen to result
in a large contribution of inelastic events under the quasi-elastic
peak, particularly at high $Q^2$.  (At the lowest Q$^2$ the
apparently-large inelastic background is greatly reduced by the use of
the fiducial cuts discussed above, leaving only the background spectra
shown in green.)  Without additional cuts to reduce
this contamination, this would present a significant problem for these
measurements.

The lower plots show  the
distributions of $\theta_{pq}$.  As mentioned above, $\theta_{pq}$ is
the angle between direction of the nucleon's momentum ($\vec{p}$),
reconstructed from the position of the hit on BigHAND, and the 
momentum-transfer vector ($\vec{q}$), as reconstructed based on the
scattered-electron's energy and direction.  For elastic scattering
from a nucleon at rest, $\theta_{pq}$ would peak sharply at zero,
having a finite width only due to measurement resolution.  For
quasi-elastic scattering, $\theta_{pq}$ is broadened by
the unknown initial momentum of the struck nucleon.  However, it is
seen to still be sharply peaked.  At the $Q^2=4.5$ (GeV/c)$^2$ point
shown in Fig. \ref{kin2_sim}, the quasi-elastic $\theta_{pq}$
distribution is seen to be almost entirely contained below 
$\theta_{pq}=2^\circ$ while at the $Q^2=18$ (GeV/c)$^2$ point
shown in Fig. \ref{kin4_sim}, the quasi-elastic $\theta_{pq}$
distribution is even sharper, being almost entirely below
$\theta_{pq}=1^\circ$.  The distribution for inelastic events is seen
to be much wider. This provides an additional cut which can be used to
select the quasi-elastic events of interest and reject the background
from inelastic events.  Furthermore, the linear rise of the inelastic
distribution (which is a geometric effect, reflecting a roughly constant
density of nucleon hits per unit area in the region pointed to by the
$\vec{q}$ vector) suggests the possibility of correcting for residual
inelastic contamination by extrapolating the large-angle $\theta_{pq}$
spectrum into the region of the cut and predicting the contribution of 
inelastic events surviving the cut.


\begin{figure}
\vspace{-.25in}
\includegraphics[width=3.1in]{figures/fig_pqa.jpg}\hfill
\includegraphics[width=3.1in]{figures/fig_pqb.jpg}\\
\vfill
\includegraphics[width=3.1in]{figures/fig_pqc.jpg}\hfill
\includegraphics[width=3.1in]{figures/fig_pqd.jpg}\\
\vfill
\includegraphics[width=3.1in]{figures/fig_pqe.jpg}\hfill
\includegraphics[width=3.1in]{figures/fig_pqf.jpg}\\
\vfill

\vspace{-.2in}
\caption{\label{kin4_additional}
Quasi-elastic (red) and inelastic (blue)  cross-section vs. $W^2$ with coincident
neutron detection with different cuts on $\theta_{pq}$.  The cuts
applied to $\theta_{pq}$ are indicated on each plot.  The vertical axis
for each plot is the efficiency- and acceptance-weighted integrated
cross section in fb/bin.
   }
\end{figure}

From the figures presented above, it is clear that the 
inelastic background is largest in the
case of neutron-coincident measurements at the highest $Q^2$.  For
this case, Fig. \ref{kin4_additional} presents additional results 
from the simulation demonstrating the effect of cuts on $\theta_{pq}$.
The signal-to-noise ratio is seen to improve as the cut is tightened.
For a cut tighter than $\theta_{pq}<1^\circ$ the accepted
quasi-elastic coincidence cross section is seen to decrease (as expected from the 
$\theta_{pq}$ plot shown in Fig. \ref{kin4_sim}). 

The optimal choice of cuts on $W^2$ and $\theta_{pq}$ involves a
trade-off of statistics against signal purity.
Integration of the spectra for the
$\theta_{pq}<.5^\circ$ cut, for example, indicates that the 
inelastic background can be reduced to a 23\% contamination by a cut selecting
$-0.95<W^2<0.5$ GeV$^2$ at a cost of
77\% of the quasi-elastic acceptance.  The signal could be about
doubled by increasing the upper $W^2$ cut-off to 1.0 GeV$^2$ while
only increasing the background to 26\%.  Similarly  increasing the
cut-off to 1.5 GeV$^2$ would add another 50\% to the integrated signal
but would increase the background contamination to 37\%. (We refer to
background contamination at background/signal.  The fraction of
background relative to total counts would be a smaller number.
e.g. A 50\% background, meaning that background is 50\%
of signal, is equivalent to 33.3\% of total events being background events.)

The best cuts will have to be chosen based on the observed data,
estimated background contamination and estimated systematic error on
the determination of the background.  For the simulated data presented
here, a set of cuts was chosen to minimize the error on the extracted
value of R.  These were based on the simulated spectra, the
anticipated luminosity and running time requested in the present
proposal, and an assumed 20\% systematic error (in addition to
$\sqrt{n}$ statistical error) on the background contamination.  The
fractional systematic error was assumed to be common in the neutron
and proton contamination and so partly canceled in the evaluation of
R.  For the actual analysis, a locus in $W^2-\theta_{pq}$ space might
be used to select events.  For the purpose of rate estimates, a simple
rectangular region was selected by optimizing separate cuts on $W^2$
and $\theta_{pq}$.

The resulting optimal cuts and anticipated contamination are
presented in Table \ref{inelastic_contam}.  
These results are used
below in estimation of systematic errors.  The contamination
fractions listed in  Table \ref{inelastic_contam} are the values
before correction.  Given a 20\% systematic error, the majority of the
contamination would be subtracted off.  Since the inelastic
backgrounds in the neutron and proton spectra are expected to be of
similar shape (and will be extrapolated into the quasi-elastic region 
using similar shapes) if one background is underestimated the other may
also be expected to be underestimated.  Similarly, if the neutron
background is oversubtracted, the proton background will also be
oversubtracted.   As mentioned above, this common error in the
numerator and denominator partly cancels in evaluation of the ratio,
$R''$.  The cancellation is not complete because the fractional errors
in the numerator and denominator will differ. But the partial
cancellation results in a fractional error on the ratio, due to
systematic background error, which is
smaller than the fractional error contributed to the neutron or proton counts.



\begin{table}
\begin{center}
\caption{
Estimated fractional contamination of inelastic events in the
quasi-elastic sample after $W^2$ and $\theta_{pq}$ cuts but {\it before any correction
  is applied}
\label{inelastic_contam}}
\vspace{.2in}
{\begin{tabular}{|c|c|c|c|c|c|c|c|c|c|}
\hline
$Q^2$ (GeV/c)$^2$&3.5& 4.5&6.0& 8.5&10.&12.&13.5&16.&18.\\
\hline
\hline
Max. $\theta_{pq}$ (deg.)&2.5&2.3&1.9&1.1&0.9&0.8&0.7&0.6&0.6\\
\hline
Max. $W^2$ (GeV$^2$)&1.1&1.2&1.3&1.3&1.4&1.4&1.6&1.6&1.7\\
\hline
\hline
Proton contamination (\%)&5.3&5.5&8.8&13.&15.&21.&26.&28&33.\\
\hline
Neutron contamination (\%)&13.5&11.&15.4&26.&30.&34.&43.&50.&51.\\
\hline
\end{tabular}}
\end{center}
\end{table}




\section{Rates and Trigger}

\label{rates}
In this section we review the inputs used in the rate calculations and
give the expected rates for the quasi-elastic coincidence measurements
and for the calibration reactions.

Since background rates are roughly proportional to the number of
target nucleons, rate estimates are based on a luminosity of
$\mathcal{L}=6.7\times 10^{37}/A$ cm$^{-2}$s$^{-1}$, where $A$ is the number of
nucleons in the target.  While this exceeds our experience in the
GEn experiment \cite{GEn-proposal}  using the same equipment, there
are several factors which decrease the susceptibility of the present
experiment to accidental background. 
The observed rate of
accidental events above threshold in BigHAND was $\approx 2$ MHz for that
measurement.  BigHAND will be significantly further back from
the target (17 m compared to 6.5 to 12 m for GEn running) so a much
smaller solid angle is subtended. Further, the deflector magnet will
reduce rates due to low-energy particles.  BigBite will be
instrumented with GEM detectors which should be able to handle a
significantly higher luminosity than proposed here. 

% The gas Cerenkov
%detector will eliminate pion-induced triggers, bringing the trigger
%rate down to an acceptable level.

A Monte Carlo\cite{Seamus} simulation has been done to predict the background trigger rates
which will may be expected as a function of the threshold applied to
the electromagnetic calorimeter.  The simulation included a
parameterization of expected light-generation by hadrons.
Pion rates were based on an empirical fit to
observed charged pion production rates measured at SLAC.  A
parameterization of electron-scattering rates from deep
inelastic scattering was also included.
The results of the simulations at the kinematic points of this
experiment are shown in Fig. \ref{trig_rates}. 
For each kinematic point, the 
threshold must be chosen chosen low enough so that quasielastic events
are accepted with reasonable efficiency.  An advantage of the ratio
method is that there is no systematic error introduced by such an
inefficiency since it would be independent of the type of recoiling
nucleon.  Table \ref{thresholds} lists, for each kinematic point, the
lower limit for scattered-electron energy for quasielastic events, as
determined from the quasi-elastic Monte Carlo.  The table also lists
suitable threshold values which are $\approx$10\% below this minimum.
Comparing those thresholds to Fig. \ref{trig_rates} shows that the
expected background trigger rates are comfortably low, generally well
below 1 kHz.  

The threshold on the electromagnetic calorimeter will serve to reduce
the trigger rate due to low energy scattered 
electrons from inelastic events.  While suppression of
inelastic events is generally advantageous, it is necessary to have a
measure of the inelastic rate near the quasi-elastic peak in order to
test and calibrate inelastic predictions which can then be used to
correct for contamination under the quasielastic peak.  Monte Carlo
simulations of the effect of a threshold-cut on scattered electron
energy have determined that these thresholds will cause negligible
distortion of the inelastic background for $W^2<$ 2.5 (GeV/c)$^2$.

%The pion rates are based on a parameterization of charged pion
%production measured at SLAC found in D. E. Wiser, Ph.D. Thesis,
%University of Wisconsin-Madison (1977) (unpublished).  Charged pions are
%convoluted with an energy deposition spectrum based on a MC Bogdan had
%done for the first GEn experiment.  This supresses them considerably.
%pi0 rates we say are half pi+ rate plus half pi- rate.  I will also say
%that the pi0s are not decayed and are treated as something that will
%entire deposit it's energy in the calorimeter.  This is likely a reason
%that this overpredicts the rates.  Electron rates are included from DIS
%cross sections (but don't contribute very much).
%Seamus



\begin{table}
\begin{center}
\caption{Suitable thresholds for the electromagnetic calorimeter
  trigger are listed.  For comparison, the lower limit for
  quasi-elastically scattered electrons is also given.
}
\label{thresholds}
\vspace{.2in}
{\begin{tabular}{|c|c|c|c|c|c|c|c|c|c|}
\hline
$Q^2$&3.5& 4.5&6.0& 8.5&10.&12.&13.5&16.&18.\\
\hline
\hline
Quasi-elastic E'$_{\mbox{min}}$ (GeV)&2.4&1.8&1.1&1.8&3.0&2.0&1.4&2.1&1.2\\
\hline
Threshold (GeV)&2.1&1.6&1.0&1.6&2.7&1.8&1.2&1.9&1.0\\
\hline
\end{tabular}}
\end{center}
\end{table}




\begin{figure}
\includegraphics[width=6in]{gmn_trigrates.pdf}\\
\caption{\label{trig_rates}
The predicted trigger rates from background particles as a function of
the threshold applied to the electromagnetic calorimeter.  Each curve
represents one of the proposed kinematic points.
}
\end{figure}




For all rate calculations  the 
%neutron detection efficiency was taken
%to be 75\% and the proton detection efficiency to be 80\%.  These
%are the calculated efficiencies \cite{BH_simulation} for a 25 MeV
%threshold with relatively low energy (1.7 GeV kinetic energy)
%nucleons and so should be a lower limit on the efficiency.  
%Figure \ref{np_eff} suggests that the actual efficiencies will be
%significantly better at high $Q^2$, but that simulation was done with a
%lower threshold than we intend to use. The
track-reconstruction efficiency in BigBite was taken to be 75\% and
the trigger live time was taken to be 80\%.  
%The fraction of data
%surviving geometric cuts to reject target windows was taken to be 75\%.
The estimated efficiencies for neutron and proton detection (shown in
Fig. \ref{np_eff}) are given in Table \ref{rates_input}.  

The cross section for quasi-elastic scattering was numerically
integrated over the combined acceptance of BigBite and BigHAND,
subject to the fiducial cut, as described in section \ref{QE-simulation}.
The integrals, labeled as ``$\int \frac{d\sigma}{d\Omega} d\Omega$'', 
 are given in Table \ref{rates_input}.  
The line labeled ``$W^2/\theta_{pq}$ cut'' gives the fraction of quasi-elastic
events surviving the cuts, described in the previous section, used  to reject
inelastic events.
 Defining 
$\mathcal{L}_0=6.7\times 10^{37}$  cm$^{-2}$s$^{-1}=240$/fb/hr, the luminosity
for the quasi-elastic measurements on deuterium will be
$\mathcal{L}_0/2=120$/fb/hr.
The resulting predicted rates of electron-nucleon coincidences are
given in the first two lines of Table \ref{count_rates}.
\begin{table}
\begin{center}
\caption{
Values used in calculating count rates.
\label{rates_input}}
\vspace{.2in}
{\begin{tabular}{|l|l|l|l|l|l|l|l|l|l|}
\hline
$Q^2$ (GeV/c)$^2$&3.5& 4.5&6.0&8.5&10.&12.&13.5&16.&18.\\
$E$ (GeV)&4.4&4.4&4.4&6.6&8.8&8.8&8.8&11.&11.\\
$\theta_e$&32.5$^\circ$&
41.9$^\circ$&64.3$^\circ$&46.5$^\circ$&33.3$^\circ$&44.2$^\circ$
&58.5$^\circ$&45.1$^\circ$&65.2$^\circ$\\
\hline
\hline
p efficiency (\%)&78.4&86.0&93.8&98.2&96.5&99.0&99.0&99.0&99.0\\
\hline
n efficiency (\%)&73.0&80.9&86.6&86.6&91.3&91.1&91.1&91.1&91.1\\
\hline
\hline
Quasi-elastic\\
\hline
p-coinc. $\int \frac{d\sigma}{d\Omega} d\Omega$ (fb)&
890&570&69&30.4&32.&5.9&1.35&1.21&.20\\
\hline
n-coinc. $\int \frac{d\sigma}{d\Omega} d\Omega$ (fb)&
380&250&31&13.9&15.&2.7&0.63&0.56&.093\\
\hline
$W^2/\theta_{pq}$ cut (\%)&80&85&74&67&64&61&63&60&63\\
\hline
\hline
Proton elastic (calibration)\\
\hline
Full $\Delta\Omega$ (msr)&22.6&38.4&53.6&---&---&---&---&---&---\\
\hline
$\frac{d\sigma}{d\Omega}_{p(e,e')}$ (pb/sr)&107&17&1.3&---&---&---&---&---&---\\
\hline
\hline
$p(\gamma,\pi^+)n$ (calibration)\\
\hline
$\int \Gamma dk$&0.0023&0.0030&.0046&---&---&---&---&---&---\\
\hline
$\theta^*_{\gamma\pi}$&86.5$^\circ$&102$^\circ$&128$^\circ$&---&---&---&---&---&---\\
\hline
$\frac{d\sigma}{d\Omega}_{p(\gamma,\pi^+n)}$ (pb/sr)&2100&1190&1770&---&---&---&---&---&---\\
\hline
\end{tabular}}
\end{center}
\end{table}

For proton efficiency measurements using elastic scattering on Hydrogen, 
the full combined solid angle of
BigBite and BigHAND can be used, without a fiducial cut.  This 
is given as ``Full
$\Delta\Omega$'' in Table \ref{rates_input}.  The (scaled dipole
approximation) cross section is also given.  For this calibration on
an LH2 target, the full luminosity $\mathcal{L}_0$ can be used.  The 
resulting rates are given in the third line of Table
\ref{count_rates}.
Dashes indicate the kinematic points for which proton efficiency is
high and will be estimated rather than based directly on calibration runs.

For the neutron efficiency measurements using the $p(\gamma,\pi^+n)$
reaction the full combined solid angle, ``Full $\Delta\Omega$''  can 
again be used.   Because
the 6\% radiator will approximately quadruple the reaction rates in
the target, the luminosity must be reduced to $\mathcal{L}_0/4$.
Table \ref{rates_input} gives the number of useful end-point photons
per incident electron ($\int \Gamma dk$), from Table
\ref{brem-end-point}. 
Dashes indicate the kinematic points for which neutron efficiency is
high and will
be estimated rather than being based directly on calibration runs.
Table \ref{rates_input} also
lists the center of mass angle, $\theta^*_{\gamma\pi}$,  for 
the $(\gamma,\pi^+)$ reaction and
the estimated cross section for the reaction.  The latter is
calculated using the scaling predicted by the constituent counting
rule:
$$s^7\frac{d\sigma}{dt}=\mbox{ constant}$$
where $s$ and $t$ are the Mandelstam variables (invariant mass squared
and four-momentum transfer squared).  The constant value was
conservatively taken to
be $0.5\times 10^7$ GeV$^{14}$ nb/GeV$^2$ at
$\theta^*_{\gamma\pi}=90^\circ $ based on 
measurements \cite{E94-104} made in the
range of $s$ of interest.  The cross section for  the
actual $\theta^*_{\gamma\pi}$ was then found based upon an
empirical fit \cite{Anderson} 
to the angular distribution:
$$s^7\frac{d\sigma}{dt}\propto(1-\cos\theta^*_{\gamma\pi})^{-5}
(1+\cos\theta^*_{\gamma\pi})^{-4}$$
Finally $\frac{dt}{d\cos\theta}$ was numerically evaluated to convert
from$ \frac{d\sigma}{dt}$ to $\frac{d\sigma}{d\Omega}$.
The resulting estimated count rates are given in the last line of Table
\ref{count_rates}.  % {\bf XXX  Better check the neutron rates!!!} 



\begin{table}
\begin{center}
\caption{
Predicted coincidence rates (counts per hour)
\label{count_rates}}
\vspace{.2in}
{\begin{tabular}{|l|l|l|l|l|l|l|l|l|l|}
\hline
$Q^2$ (GeV/c)$^2$&3.5& 4.5&6.0& 8.5&10.&12.&13.5&16.&18.\\
\hline
\hline
$d(e,e'p)$&40700&26600&3110&1345&1240&244&56.7&47.0&7.9\\
\hline
$d(e,e'n)$&17600&12000&1600&627&580&114&26.5&22.0&3.72\\
\hline
$p(e,e'p)$&273000&82000&9300&4400&5000&850&200&175&30\\
\hline
$p(\gamma,\pi^+n)$&2920&4030&13500&---&---&---&---&---&---\\
\hline
\end{tabular}}
\end{center}
\end{table}




\section{Systematic Errors}

\label{systematic_errors}
In this section we will estimate (or set upper limits on)
contributions to the systematic error on the ratio $R$ determined by
this experiment.  This sets the scale for the statistical accuracy
for which we should aim since there would be little gain in achieving
statistical errors which are much smaller than the systematic errors.

The use of the ratio method eliminates many potential sources of
systematic error.  Because $d(e,e'p)$ and $d(e,e'n)$ are measured
simultaneously, their ratio is insensitive to target thickness, target
density, beam current, beam structure, live time, trigger efficiency, 
electron track reconstruction efficiency, etc.
So the fractional error achieved on the ratio can be much smaller that
which could be achieved on the measurement of
either cross section in itself.

Table \ref{systematic_summary} lists the estimated contributions to
systematic errors, which are discussed in detail below.

The corrections needed to convert from the measured value $R''$
(Equation \ref{r-double-prime-eqn}) to the quantity of interest, $R$
(Equation \ref{r_eqn}) are only of the order of 1\%, so systematic
errors on them will be neglected.  


The errors on the proton elastic
cross section (needed to extract \gmn from $R$) do not contribute to
the error on the ratio \gmnc/{G$^{\mbox{\scriptsize p}}_{_{\mbox{\tiny M}}}~$}.
This is a more fundamental result than \gmnc, itself, or the ratio of
\gmn to the scaled dipole.  The ratio 
\gmnc/{G$^{\mbox{\scriptsize p}}_{_{\mbox{\tiny M}}}~$} is generally
more directly calculable within a given theoretical model.
While subsequent
improvements in proton cross section measurements 
can be combined retrospectively with the $R$ values from this
experiment to improve the extraction of \gmnc, we include an estimated
contribution to the error based on projected errors on 
\gmnc/{G$^{\mbox{\scriptsize p}}_{_{\mbox{\tiny M}}}~$} from 
an approved 12 GeV experiment\cite{E12-07-108}.  We assume a 1.7\%
fractional error on the proton cross section across the region of
interest except 4\% at the highest Q$^2$, which is beyond the region
to be 
covered in that measurement.  
This error should be included as a systematic in extraction
of (\gmnc)$^2$, but not of the ratio 
\gmnc/{G$^{\mbox{\scriptsize p}}_{_{\mbox{\tiny M}}}~$}.


If {G$^{\mbox{\scriptsize n}}_{_{\mbox{\tiny E}}}~$} behaves like the
Galster parameterization at high Q$^2$, then its contribution 
to the neutron cross section is small.  Thus, even large fractional
errors in {G$^{\mbox{\scriptsize n}}_{_{\mbox{\tiny E}}}~$} would
produce only small errors in (\gmnc)$^2$.  The contribution listed as 
{G$^{\mbox{\scriptsize n}}_{_{\mbox{\tiny E}}}~$} in Table 
\ref{systematic_summary} are based on an assumed error of 100\% of the
Galster parameterization, growing to 200\% for Q$^2=10$ (GeV/c)$^2$
and growing to 400\% for the two highest-Q$^2$ points.




\begin{table}
\begin{center}
\caption{
Estimated contributions (in percent) to systematic errors on R.
\label{systematic_summary}}
\vspace{.2in}
{\begin{tabular}{|l|l|l|l|l|l|l|l|l|l|}
\hline
$Q^2$ (GeV/c)$^2$&3.5& 4.5&6.0& 8.5&10.&12.&13.5&16.&18.\\
\hline
\hline
 proton cross-section&1.7&1.7&1.7&1.7&1.7&1.7&1.7&1.7&4.\\
\hline
\hline
%$G_E^n$
\gen&1.7&1.1&0.49&0.51&0.56&0.69&0.38&.89&.39\\
\hline
Nuclear correction,&-&-&-&-&-&-&-&-&-\\
\hline
Accidentals&-&-&-&-&-&-&-&-&-\\
\hline
Target windows&.2&.2&.2&.2&.2&.2&.2&.2&.2\\
\hline
Acceptance losses&0.45&0.5&0.3&0.4&0.16&0.18&0.1&0.1&.08\\
\hline
Inelastic contamination&1.5&1.14&1.32&2.64&2.94&2.62&3.24&4.4&3.64\\
\hline
Nucleon mis-identification&0.6&0.6&0.6&0.6&0.6&0.6&0.5&0.5&0.5\\
\hline
BigHAND calibration&0.3&0.3&0.5&2&2&2&2&2&2\\
\hline
\hline
Without proton err.\\
%\hline
%Total (quadrature sum)&
%1.72&1.43&1.59&3.4&3.63&3.36&3.85&4.86&4.19\\
\hline
Syst. error on \gmnc/{G$^{\mbox{\scriptsize p}}_{_{\mbox{\tiny M}}}~$}&
1.21&0.9&0.81&1.72&1.83&1.72&1.93&2.47&2.1\\
%&1.2&0.89&1.01&1.72&1.81&1.72&1.93&2.47&2.1\\
%%%%&1.2&0.9&1.2&1.85&2.23&2.09&2.68&3&3.62\\ & oops with stat
\hline
\hline
With proton err.\\
\hline
Syst. error on \gmnc&
1.48&1.24&1.17&1.92&2.02&1.91&2.11&2.61&2.9\\
%&1.47&1.23&1.32&1.92&2&1.91&2.11&2.61&2.9\\
%%%1.47&1.24&1.47&2.04&2.39&2.26&2.81&3.12&4.14\\
\hline
\end{tabular}}
\end{center}
\end{table}





Accidental coincidences of background events in BigHAND are not
expected to cause significant systematic errors.  Previous experience
in the GEn experiment\cite{GEn-proposal}  
showed a background rate of $\approx$2 MHz
across the entire detector at a similar luminosity (BigHAND saw about 
half the proposed luminosity, although BigBite viewed only a third of
that).  BigHAND will be
significantly further from the target for this measurement so lower
rates may be expected.  Even with a 1 MHz rate, the probability of an
accidental coincidence within a 5 ns timing gate will be only 0.5\%
across the entire detector.  The area searched for coincident protons
or neutrons is limited by the $\theta_{pq}<2^\circ$ cut to about 1.1
m$^2$ of the 8.4 m$^2$ face.  So the rate of accidental coincidences
is expected to be less than 0.07\% for each nucleon species.
Furthermore, this accidental contamination can be accurately estimated
and subtracted by measuring the accidental coincidence rate in other
parts of BigHAND, where true coincidences are not expected.


The target windows contain about 3.4\% as many nucleons as the LD2
target.  We use this as an upper limit on the background contribution
since many of the events from aluminum will be rejected by the
$\theta_{pq}<2^\circ$ cut.  This contribution will be subtracted off
by running on a dummy target with thick aluminum windows.  It should
be possible to apply this correction with at least 5\% accuracy (allowing for
uncertainty in window thickness).  So the systematic error on the
correction should be under 0.17\%.


\subsection{Acceptance Losses}

As discussed above, loss of nucleons from the acceptance of BigHAND
can cause a systematic error, to the extent that the losses differ for
neutrons and protons.  At the higher $Q^2$ points, the distribution of
nucleons (shown in Fig. \ref{calzone2}) are centered in BigHAND,
with only modest acceptance losses.  At the lower $Q^2$ points a
fiducial cut will be applied to select events which are similarly
centered in BigHAND (as discussed in section \ref{fiducials}) so the
acceptance losses are kept small. 
% As shown in Table \ref{final-acceptance}, 
The acceptance corrections are less than 5\%
in all cases, and considerably smaller at the highest $Q^2$ points.  
These losses
result from the high momentum tail of the deuteron wave-function, which
is identical for the neutron and proton.  The cause of a difference in
the corrections for the two nucleon species would be a difference in
placement of the proton and neutron 'images' relative to the edges of
BigHAND.  It very conservative to allow for a possible 10\%
uncorrected difference in these corrections.  This is entered as the
systematic error contribution in Table \ref{systematic_summary}.


This systematic error can be investigated (and perhaps even reduced)
by examining the stability of the extracted ratio, $R$, as the
fiducial cut is tightened or loosened from its normal value.
Additionally, choosing a ``window-frame'' fiducial which
preferentially selects events scattered near the edge of the BigHAND
acceptance would allow investigation of the actual loss of protons and
neutrons, which can be compared to Monte-Carlo estimates.  Perhaps
more importantly, such a study can be used to determine how closely
the fractional losses of neutrons and protons are matched.

\subsection{Inelastic Contamination}

With the $W^2$ cuts described at the end of  section \ref{inelastic}
the predicted
residual contamination of inelastic events in the quasi-elastic sample
{\em before any corrections are made}
are given in Table \ref{inelastic_contam}.
These may be overestimated since the kinematics used for the inelastic
simulation tend to exaggerate the effects of Fermi broadening.
In the ratio $R$ the contamination tends to cancel.  More importantly,
it should be possible to accurately estimate and subtract the
inelastic contamination.  The $\theta_{pq}$ distribution
(such as that shown in Fig. \ref{kin4_sim}) beyond
2$^\circ$ gives a measure of the amount of inelastic
contamination.  The measured distribution can be
extrapolated to small angle to estimate the residual
contamination.   It should certainly be possible
to estimate the inelastic contamination at the 20\% level.  Assuming
the numbers in Table \ref{inelastic_contam} are reduced by such a
factor, gives the contributions listed in 
Table \ref{systematic_summary}.   This is generally the largest contribution to
the systematic error.



\subsection{Nucleon mis-identification}
Because of the long tail of the momentum distribution of the deuteron
wave-function, some nucleons will be displaced far from the position
predicted based on their $\vec q$ vector and charge.  This will result
in mis-identification if a neutron is displaced sufficiently far
upwards or a proton sufficiently far downwards.  In the
model used for the deuteron wave-function, about 5\% of the nucleons
have a component of momentum exceeding 100 MeV/c in any chosen
direction.  With a 200 MeV/c 'kick' being given to protons by the
dipole, this would result in 5\% mis-identification rates.  The cut on
$W^2$ preferentially rejects events with large Fermi momentum, however
and so the mis-identification rate is reduced to 3\% or less for the
cuts considered here.  The misidentification of one species as the
other and {\it vice versa} do not cancel because the proton rate is
higher and protons have a higher efficiency of being detected.   If the
mis-identification went uncorrected, the number of detected 'neutrons'
would be increased by proton contamination while the number of detected
'protons' would be decreased by a loss of protons which would not be
offset by misidentified neutrons.  Thus the effect would be an
overestimate of $R$.  The fractional overestimate of $R$ depends on
the actual ratio of proton/neutron cross sections and efficiencies.
Taking the ratio of cross sections to be 
$\left.\frac{d\sigma}{d\Omega}\right|_p/\left.\frac{d\sigma}{d\Omega}\right|_n
\approx 2.4$ and the detection efficiencies to be
$\epsilon_p=80$\% and $\epsilon_n=75$\%, this 3\% kinematic spread of
nucleons would cause the neutron rate to be overestimated by 4.7\% and
the protons to be underestimated by 1.8\% and so $R$ would be overestimated by 
6.6\%.  

The contamination will not go uncorrected, however, and at least two
techniques will be used to measure this kinematic tail.  The neutron
tail below the predicted point on the face of BigHAND 
can be measured without contamination
from protons. Similarly the proton tail above the predicted point on
BigHAND will
be free of neutrons.  Symmetry can then be used to predict the
contamination of neutrons in the proton peak and, with minor kinematic
corrections because of the deflection magnet, the proton contamination
of the neutron peak.  
Since these tails originate from the same Fermi motion, the neutron and
proton tails should be almost identical apart from minor distortion
due to the deflection magnet.  Another technique for determining the
tails in the 'contamination region' will be to use the vetos to
preferentially select neutrons or protons.  In particular a clean
sample of neutrons can be selected (with only a few percent proton
contamination, which can be subtracted off) to separately determine 
the kinematic spread.  It should be possible to measure at the 10\%
level, the 'leakage' of neutrons/protons into the regions in which the
other species is expected.   A 10\% error in the measurement of such a
3\% tail would cause a systematic error of $\approx$0.6\% in the extracted
value of $R$.  In fact, stronger field settings used for higher-Q$^2$
points should reduce the misidentification, so the error estimates in 
Table \ref{systematic_summary} are probably significantly overestimated.


\subsection{Nucleon Detection Calibration}

There are several potential issues involved in determining the systematic errors
associated with the calibration of the BigHAND efficiencies for
detection of neutrons and protons.  A useful attribute of the 
$p(\gamma,\pi^+)n$ reaction is that the neutrons used for calibration
have essentially the same energy as those in the middle of the
quasielastic peak.  This is a significant advantage relative to our
previous CLAS experiment \cite{Will,Jeff} at lower $Q^2$ in which $p(e,e'\pi^+)n$ was used for 
calibration.  There, the three-body final state gave lower neutron
energies than those of interest so it was necessary to parameterize
the efficiency as a function of energy to estimate the efficiency at
the energy of interest.  Thus, the largest source of systematic
errors in the earlier experiment is avoided here.

In principle the statistical error on the calibration of the detection
efficiencies represents a systematic error on the measurements of the
quasi-elastic reaction of interest.  In fact it is not prohibitive to
obtain sufficient statistics on the calibration reaction so that the
error on the efficiencies are comparable to the statistical error on
the quasi-elastic measurements.  We treat this as a statistical error
in subsequent discussions, and so don't include it in 
Table \ref{systematic_summary}.

As shown in Table \ref{calzone-table}, up to 17\% of the
quasi-elastic nucleons (in the $Q^2=6.0$ (GeV/c)$^2$ case) will fall
outside the region which can be calibrated at elastic kinematics at
the same beam energy and BigHAND position.  The ``uncalibrated'' regions
of the detector could reasonably be assigned the average efficiency
obtained from the calibration, since they share the same
shower-generation geometry and
are, in fact, mostly just different positions on the same physical
detector bars.  A further refinement can be made, as has been
described in section \ref{n-cal}, by extrapolating calibrations made
at lower energies to take into account the slightly different
interaction probability.  Since the efficiency is high, and unlikely
to change rapidly, it is unreasonable to expect the efficiency in
the ``uncalibrated'' region to differ by more than 2\% from the
efficiency predicted by these methods.
%, as the 
%actual change in efficiency from one calibration point to the next
%(shown in Table \ref{rates_input} and Fig. \ref{np_eff} does not
%exceed 8\% from one calibration point to the next.  
%the nucleon cross section
%shown in Fig. \ref{pdg-sigmann} doesn't change by more 
%than 4\%. 
(Copious calibration data will
be taken at the low $Q^2$ point, so any anomalous regions, such as
dead PMT's will be identified and either repaired or corrected for.)
We assign an upper-limit systematic error of 2\% of the fraction of
the events which fall outside the calibrated region.  Since this error
appears in both the numerator and denominator of $R$, it is multiplied
by $\sqrt{2}$.
% to give an error of 2.8\% of the fraction of
%nucleons outside the calibrated region (found from the last column of
%Table \ref{calzone-table}).

%For the $Q^2=5.25$ and 7 (GeV/c)$^2$ points, the BigHAND efficiency will
%be estimated by interpolation of calibration points at higher and
%lower $Q^2$.  (As described above, the interpolation will be guided by
%the expected efficiency variation so an 'effective thickness' will be
%interpolated.)  Table \ref{rates_input} shows that the total change in
%efficiency between the points at $Q^2=4.5$ and 6.0(GeV/c)$^2$ is only
%5.7\% for neutrons and 7.8\% for protons. Taking one quarter of this
%range as a generous estimate of the error bar on the interpolation 
%of the efficiency to $Q^2=5.25$ (GeV/c)$^2$ for
%each species gives a systematic error on the efficiency-corrected 
%ratio of 2.8\%.  Similarly interpolation to $Q^2=7$ (GeV/c)$^2$ would
%be assigned errors of one quarter of 4\% and of 3.8\% for neutrons and
%protons respectively, to give an error of 1.5\% on the
%efficiency-corrected ratio.

For the highest-Q$^2$ points, no direct calibrations will be made.
The proton and neutron
efficiencies there are both large and stable, we assign a systematic error
2\% in the efficiency correction for those points.

The final line of Table \ref{systematic_summary} shows the estimated
total systematic error on $R$, found by adding the individual
contributions in quadrature.  

\section{Installation}

All the equipment used for this experiment is either already existing
or planned
for use in the Super-BigBite-Spectrometer (SBS) in Hall A.
\begin{itemize}
\item The BigBite spectrometer exists, but with different 
  tracking instrumentation than is planned for this measurement.
\item The GEM detector planes are planned as instrumentation of the
  polarimeter in the SBS
\item The BigHAND detector array exists  and has been used in Hall A.
  It is presently in storage at JLab
\item The BigBen magnet is needed for an already-approved
\cite{FPP} 12 GeV experiment.  Although we are using it only as a
deflector magnet, it will eventually be the heart of the SBS.
\end{itemize}

Two major pieces of equipment will need to be installed in the Hall
for this experiment.  

Since the BigHAND nucleon detector has been used in
the past \cite{GEn-proposal}, there is operational experience
on rigging it in and out of the Hall.  In preparation for the GEn
experiment, it was rigged into the Hall in six weeks, with no previous
operational experience and with the BigBite spectrometer being rigged
in parallel.

The 48D48 (BigBen) 'spectrometer' magnet (here being used as a
particle-identification magnet) is available from BNL but 
will required major modifications in
advance of installation to allow it to be placed close to the beam line
without mechanical or magnetic interference with the beam on its way
to the beam dump.  We expect to play a leading role in the magnetic 
and mechanical
design for modifications to the magnet and for beam line shielding
and/or correctors.  We will also participate in the design of
mechanical systems to allow the magnet to be re-positioned for
different scattering angles.
These
modifications are needed independently for the approved GEP
\cite{FPP} experiment.

The same equipment is needed for the proposed high-Q$^2$ measurement of 
\genc\cite{new_GEN} and the BigBen magnet would be used for the
proposed transversity experiment and as part of the SBS facility.
Installation time could be economized by scheduling the \gen
experiment to follow the presently proposed experiment.  Then only the
polarized $^3$He target would need to be installed for the change-over
to the next experiment. Similarly the BigBen magnet could then be
instrumented to form the SBS spectrometer for the transversity
measurement.  This would obviate the need for rigging the magnet back
out of the Hall.

A significant upgrade of the
BigHAND electronics is planned, to reduce the dead time of the veto
detectors.  That can be carried out independently of Hall operations.

Based on past experience, including the installation of the GEn
experiment, we estimate that six to eight weeks will be required to
install the equipment needed for this experiment.  Ideally, of course,
this will be carried out during an accelerator down-time, or when beam
is unavailable because the three other halls are running.



\section{Beam Time Request}

Given the simulation results for rate estimates and background
contamination, an evaluation was made of the trade-off of statistical
and systematic errors.  Since the dominant systematic error is due to
background contamination, that was the error considered in choosing
running times which allowed a good trade-off of cuts which were tight
enough to reduce systematic errors but not so tight as to require
prohibitive running times.  Table \ref{R_stat_syst} gives the
resulting contributions of statistical errors and background-related
systematic errors to the determination of R for the  requested
beam times listed below, in Table \ref{beam_request}.  

\begin{table}
\begin{center}
\caption{
Estimated statistical and systematic (from background subtraction
only) errors on R for $W^2-\theta_{pq}$ cuts optimized for the requested
running times.
\label{R_stat_syst}}
\vspace{.2in}
{\begin{tabular}{|l|l|l|l|l|l|l|l|l|l|}
\hline
$Q^2$ (GeV/c)$^2$&3.5& 4.5&6.0& 8.5&10.&12.&13.5&16.&18.\\
\hline
\hline
Statistical (\%)&0.3&0.3&0.8&1.4&1.3&2.4&3.2&3.4&5.9\\
\hline
Systematic (\%)&1.5&1.2&1.3&2.6&2.9&2.6&3.2&4.4&3.6\\
\hline
\end{tabular}}
\end{center}
\end{table}

Table \ref{R_stat_syst} demonstrates that the beam hour requests have
been chosen to give a reasonable match of systematic and statistical
errors.  The four lowest-Q$^2$ points appear to be exceptions to this,
with statistical errors which are significantly smaller than the 
systematic errors.  Decreasing the running times at those kinematics
to match the statistical and systematic errors would have only a very
minor impact on the overall beam time request.  The run times have
been chosen to take advantage of the high rates and allow copious
statistics to be accumulated to enable careful studies of small
systematic effects using the data.


Because binomial statistics apply for the efficiency measurements,
the number of required events is considerably smaller than might be
expected for Poisson statistics.  If $N_{in}$
particles are incident on the detector, each with probability $p$ of
being observed, then the variance on the number of observed particles
$N_{obs}$ is
$$\sigma^2=N_{in}~p(1-p)\approx N_{obs}~(1-p)$$
So the fractional error in the efficiency $\eta=N_{obs}/N_{in}$ is
$$\frac{\sigma_\eta}{\eta}=\frac{\sigma}{N_{obs}}=\frac{\sqrt{1-p}}{\sqrt{N_{obs}}}$$
The required statistics for a given fractional error is therefore
reduced by a factor of $(1-p)$ compared to counting statistics.  With 
conservatively estimated neutron and proton efficiencies of at 
least $p_n=0.70$ 
and $p_p=0.75$ this reduces the required number of calibration
coincidences by factors of 0.3 and 0.25, respectively.


\begin{table}
\begin{center}
\caption{
Beam Time Request (beam hours).  ``Normal $\mathcal{L}$'' refers to
running at the standard luminosity of 6.7 $\times 10^{37}$ /A/cm$^2$/sec.
Reduced luminosity running is indicated as ``Half $\mathcal{L}$'' or
``10\% $\mathcal{L}$''.
\label{beam_request}}
\vspace{.2in}
{\begin{tabular}{|l|l|l|l|l|l|l|l|l|l|l|}
\hline
$Q^2$ (GeV/c)$^2$&3.5& 4.5&6.0& 8.5&10.&12.&13.5&16.&18.&\\
\hline
$E$ (GeV)&4.4& 4.4& 4.4&6.6&8.8&8.8&8.8&11.&11.&\\
\hline
$\theta_e$&32.5$^\circ$&
41.9$^\circ$&64.3$^\circ$&46.5$^\circ$&33.3$^\circ$&44.2$^\circ$
&58.5$^\circ$&45.1$^\circ$&65.2$^\circ$&\\
\hline
$\theta_{N}$&31.1$^\circ$&
24.7$^\circ$&15.6$^\circ$&16.1$^\circ$&17.9$^\circ$&13.3$^\circ$
&9.8$^\circ$&10.7$^\circ$&7.0$^\circ$&\\
\hline
\hline
d$(e,e')$\\
\hline
Normal $\mathcal{L}$&12&12&18&18&24&36&96&108&216&\\
\hline
Dummy target&2&2&2&2&3&4&8&8&16&\\
\hline
Half  $\mathcal{L}$&12&12&12&12&&&&&&\\
\hline
Dummy half $\mathcal{L}$&2&2&2&2&&&&&&\\
\hline
 10\% $\mathcal{L}$&12&12&&&&&&&&\\
\hline
Dummy 10\% $\mathcal{L}$&2&2&&&&&&&&\\
\hline
\hline
H$(e,e')$\\
\hline
Normal $\mathcal{L}$&3&3&24&4&4&4&5&6&24&\\
\hline
Half  $\mathcal{L}$&3&6&2&2&2&2&2&2&2&\\
\hline
10\%  $\mathcal{L}$&18&18&&&&&&&&\\
\hline
BigBen off&6&6&6&2&2&3&6&6&24&\\
\hline
Dummy target&2&2&2&&&&&&&\\
\hline
\hline
H$(\gamma,\pi^+)$\\
\hline
Radiator&24&24&12&&&&&&&\\
\hline
Dummy target&2&2&2&&&&&&&\\
\hline
No radiator&6&6&3&&&&&&&\\
\hline
\hline


Total&
106&106&89&42&35&49&118&130&282&
%106&106&89&50&56&56&120&132&248&
$\Rightarrow$ 957\\
\hline
\hline
Commissioning&&&&&&&&&&96\\
\hline
3 Energy changes&&&&&&&&&&124\\
\hline
%15 angle changes&&&&&&&&&&60\\
6 BigBite angle changes&&&&&&&&&&24\\
\hline
8 BigHand angle changes&&&&&&&&&&48\\
\hline
6 polarity changes&&&&&&&&&&24\\
\hline
\hline
Beam request&&&&&&&&&&1173\\
&&&&&&&&&&$\approx$49 days\\
\hline
\end{tabular}}
\end{center}
\end{table}

Requested beam times are tabulated in 
Table \ref{beam_request}. 
The first line lists the times, discussed above, found to give a
suitable trade-off of statistical and systematic errors.  Additional
time is requested at lower luminosity to ensure that rate-related
effects, such as accidentals, are well understood.  At each
luminosity, running time on the dummy target is included to ensure
that accidental background can be distinguished from target-window
background.
Time is also requested for calibration of the BigHAND detector with
tagged protons and tagged neutrons.
 In some cases the requested time greatly
exceeds the minimum to allow us to take advantage 
of data which is readily
available.  The proton calibration rate is so high at the lowest
$Q^2$ points, for example, that adequate statistics could be
acquired in a few minutes.  This high rate provides an excellent
opportunity to study efficiency variation across the face of BigHAND,
so far more data taking time is scheduled.  Just a few hours of running
will give a million events, allowing the BigHAND face to be finely
subdivided and precisely calibrated in many separate regions.  For several
calibration points or high-rate quasi-elastic points we also allow
time for running at reduced luminosity to ensure that accidental rates
are well understood. 
Elastic scattering measurements with the 48D48 deflector dipole turned
off will be useful for checking the alignment of the $\vec{q}$
inferred from BigBite measurements with the actual hit positions in BigHAND.
For the
neutron calibration, time is included for running without a radiator
to measure the virtual photon contribution. 
 Time is also allowed for running on dummy
targets for subtraction of target window contributions.  
 Eight hours are allowed for each beam energy change. 
In addition to the time required to position the spectrometers for
each measurement, each elastic calibration requires two
spectrometer-moves and two reversals of the BigBite polarity.  Four
hours are allowed for each operation.
Finally, 72 beam hours are requested for commissioning of the system
including the new 48D48 deflector magnet.  In total, 28 days are requested.



\begin{table}
\begin{center}
\caption{
Estimated contributions (in percent) to errors on R, and resultant
errors on \gmnc/{G$^{\mbox{\scriptsize p}}_{_{\mbox{\tiny M}}}~$}.
\label{error_summary}}
\vspace{.2in}
{\begin{tabular}{|l|l|l|l|l|l|l|l|l|l|}
\hline
$Q^2$ (GeV/c)$^2$&3.5& 4.5&6.0& 8.5&10.&12.&13.5&16.&18.\\
\hline
\hline
 proton cross-section&1.7&1.7&1.7&1.7&1.7&1.7&1.7&1.7&4.\\
\hline
\hline
%$G_E^n$
\gen&1.7&1.1&0.49&0.51&0.56&0.69&0.38&.89&.39\\
\hline
quadrature sum of other syst.&
1.72&1.43&1.59&3.4&3.63&3.36&3.85&4.86&4.19\\
Statistical error&
0.3&0.3&0.8&1.4&1.3&2.4&3.2&3.4&5.9\\
\hline
\hline
Without proton err.\\
\hline
Total error on R&
2.43&1.83&1.81&3.71&3.89&4.19&5.02&6&7.25\\
\hline
Error on \gmnc/{G$^{\mbox{\scriptsize p}}_{_{\mbox{\tiny M}}}~$}&
1.22&0.91&0.9&1.85&1.94&2.09&2.51&3&3.62\\
\hline
\hline
With proton err.\\
\hline
Error on \gmnc&
1.48&1.25&1.24&2.04&2.12&2.26&2.65&3.12&4.14\\
\hline
\end{tabular}}
\end{center}
\end{table}






Figure \ref{projected_gmn} shows the size of the errors on the
extracted values of \gmn which would be obtained with the desired statistics
and the systematic errors given in Table \ref{systematic_summary} 
(added in quadrature), see Table \ref{error_summary}.  The value is
arbitrarily plotted at unity. 
The fractional error
on \gmn has been calculated using the projected error on the proton
cross section. 
Because \gmn is proportional to $\sqrt{R}$, the error is scaled 
down by a factor of two to give the fractional error
on \gmnc.   The errors are seen to be far smaller than those on the few
existing data points covering part of this kinematic range.

\begin{figure}
\includegraphics[width=6in]{projected_err.pdf}\\
\caption{\label{projected_gmn}
Existing data on \gmn in the $Q^2$ range of the proposed measurement
are plotted as ratio to
scaled dipole approximation. (See caption of Fig. \ref{old_data}.)
Red points (arbitrarily plotted at unity) show projected size of error
bars for this experiment.  Thick error bars include projected
statistical and systematic errors of the proposed experiment but do not
include errors on other measurements.  Thin error bars include
statistical and systematic errors and also estimated errors on \gen %G$_E^n$
and on the proton elastic cross section.  Solid light-blue circles 
with error bars
(arbitrarily plotted at 0.8 for clarity) indicate the position and
projected total errors\cite{Jerry} of the CLAS12 experiment. 
}
\end{figure}






\section{Relation to Other Experiments}

This experiment uses much of the same detector equipment as the GEn experiment
(E02-013)\cite{GEn-proposal}  and many of the 
collaborators most involved in that
experiment are also involved here.
Technical expertise in tracking in BigBite and in calibration of BigHAND 
will be available for this analysis.
%%%%
The large dipole magnet, which is proposed for deflection of protons,
is a part of an approved 12 GeV experiment which will measure Electric
Form Factor of the proton at very large $Q^2$ \cite{FPP}.

High precision measurements of the neutron magnetic form factor were
made at lower $Q^2$ in the CLAS e5 measurement (E94-017).
Many of the principle people involved in the analysis \cite{Jeff} of 
that data set are involved in the present proposal.
This measurement will complement the CLAS measurement by 
extending the precision measurements to a far higher $Q^2$.  
We will draw on much expertise and experience in
controlling systematic errors in such a ratio measurement.

A 12 GeV experiment (E12-07-104) has been approved to extend
the high-precision measurements of \gmn out to beyond $Q^2=13$
(GeV/c)$^2$ using the CLAS12 detector.
Some of the spokespersons of that experiment are also involved in the
present proposal.
The results from the two experiments will be complementary in that 
the CLAS12 data will have generally larger systematic 
errors (3\% estimated systematic error on \gmnc)\cite{Jerry} 
but will cover a subset of the  range of $Q^2$ (up to about 14
(GeV/c)$^2$) with contiguous coverage which
can be subdivided into many data points.  
The proposed experiment would measure nine discreet  $Q^2$
values and extend the coverage $Q^2=18$ (GeV/c)$^2$). The present
proposal aims to have significantly
smaller systematic errors (and for high Q$^2$ points, smaller
statistical errors) in the 
region of overlap.  
CLAS12 is a large-acceptance device which, by its nature, collects data
simultaneously at many scattering angles but with low luminosity.  
The proposed experiment collects data only at predetermined angles but can
run at far higher luminosity (the planned luminosity for this proposal is 
over 600 times the design luminosity for CLAS12).  
Systematic errors are more easily controlled for single-position detectors 
than for large acceptance detectors.
The use of the same detector for both nucleon species also
helps control the systematic errors in the proposed measurement.
%%%
In the present proposal, the large baseline between the target
and the hadron detector (17 m) also provides a major advantage in 
selection of quasi-elastic events
due to superior angular and momentum resolution.

Since neutron-detection statistics dominate the statistical errors,
Table \ref{CLAS_vs_us} compares the expected statistics of 
Quasi-elastic neutrons
detected in each experiment.  Because of the CLAS12 measurements are all
made simultaneously, the statistics fall off rapidly at high Q$^2$.  In
the present experiment the high luminosity can be combined with beam
times tailored to each kinematic point to make the statistics more
even.  It is seen that the present experiment not only extends the
coverage to higher Q$^2$ than the CLAS12 experiment, but also beyond
$Q^2=10$ (GeV/c)$^2$ it has significantly higher statistics.

\begin{table}
\begin{center}
\caption{\label{CLAS_vs_us}
Comparison of expected statistics of detected neutrons between CLAS12
experiment and present proposal.  Dashes indicate points beyond the
kinematic range of an experiment, asterisks indicate points for which
no measurement is made.  The last column lists the time allocated for
data-taking in the present experiment.  This may be compared to the
total or 56 days (1344 hours) over which the CLAS12 experiment would
acquire the statistics given in column 2.}
\vspace{.2in}
{\begin{tabular}{|l|r|r|l|}
\hline
Q$^2$ &CLAS12&Present&Hours\\
(GeV/c)$^2$&&proposal&\\
\hline
2.5&$1.6\times 10^5$&---&\\
3.5&$2.3\times 10^6$&$2.1\times 10^5$&12\\
4.5&$6.6\times 10^5$&$1.4\times 10^5$&12\\
5.5&89000&*&\\
6.0&*&28000&18\\
6.5&35000&*&\\
7.5&16000&*&\\
8.5&7700&11300&18\\
9.5&4000&*&\\
10&*&13900&24\\
10.5&2200&*&\\
11.5&1300&*&\\
12&*&4100&36\\
12.5&800&*&\\
13.5&500&2550&96\\
16&---&2380&108\\
18&---&800&216\\
\hline
\end{tabular}}
\end{center}
\end{table}

Our experience on the CLAS e5 experiment showed it to be very valuable
to have overlapping measurements.  
There, two beam energies and two neutron detection systems allowed up 
to four redundant measurements in some regions.  
This was beneficial as a way to protect against
unsuspected systematic problems in each of the measurements.
Ultimately each of the experiments will benefit from the partial
overlap of the kinematic coverage.  
The CLAS12 data will serve as a useful check 
of the lower-Q$^2$ results of the present proposal.  
Inspection of the existing data (Fig. \ref{projected_gmn}) suggests 
that the proposed measurement must either see an 
abrupt change in the behavior of the form
factor in this kinematic range or  
be in conflict with either the CLAS \cite{Jeff} or SLAC 
\cite{SLAC_Rock} data sets.
In any of those cases, the CLAS12 data would then be an important confirmation.


\section{Group Contributions to 12 GeV Upgrade}
%{\bf XXX ...This needs to be written before submitting the proposal to the PAC}
The following is a list of personnel from the institutions and their intended
contribution to the proposed experiment:

\begin{itemize}


\item 
The CMU spokesperson and collaborators have long experience with neutron
  detectors and recent experience with the BigHAND detector (which
  includes 100 scintillator detectors provided by CMU).  This group
  will perform the work necessary to prepare the BigHAND detector and
  to ensure that its elements and readout are ready for operation.


\item 
The Hall A spokesperson and collaborators will be 
responsible for infrastructure of 
the 48D48 magnet, which is a part of three Form Factor experiments.

\item 
The Rutgers spokesperson and collaborators will be responsible
  for preparation of the data acquisition components including
  front-end electronics and customized software.



\end{itemize} 

In addition to equipment specific to this experiment, many
collaborators are involved in projects in support of equipment for use
in Hall A in the 12 GeV era.   Many of these components are parts of
the Super BigBite spectrometer (SBS) which will use the BigBen
spectrometer and GEM detectors, among other components.


\begin{itemize}
\item 
The INFN collaborators of are committed 
to provide a reconfigured tracker in BigBite and its operational support. 
They will also have a major role in implementation of the RICH 
in Hall A.  The source of funding for this group is INFN.
INFN has approved about \$150k for prototyping of the large GEM chambers
and expected to support whole front tracker in GEP5 experiment including
electronics for high resolution operation in SBS and BB.

\item 
The UVA collaborators are responsible for construction and operation
of high polarization high luminosity He target, which is a major part
of the new GEN proposal to this PAC.
The UVA collaborators are also responsible for reconfiguration of the tracker
in the Super BigBite Spectrometer and its operation. 
The source of funding for this group is DOE.
The University of Virginia group, which recently developed a major new tracker
for the BigBite spectrometer, will  submit an MRI proposal to NSF 
for construction of a GEM-based tracker for SBS. 


\item 
The Glasgow group intends to work on GEM-based detectors for the 
PANDA experiment and will share their results in hardware design and readout 
software with this effort, effectively contributing several FTE's.

\item 
The Florida International University also intends to contribute in
the development of a GEM-based tracker at least 1 FTE and put
a graduate PhD thesis student in this experiment.

\item 
The CMU group
will use their expertise  to implement the hadron calorimeter
and the beam line magnetic shielding, both of which also required in
the GEP5 experiment E12-07-109. 
This group has also become involved in the development of Compton
polarimeter equipment for Hall A, and will continue that development
into the 12 GeV era.
The source of funding for this group is DOE.

\item 
Along with the CMU group, College of William and Mary group, together with Dubna collaborators, 
intends to prepare the hadron calorimeter elements and their 
implementation in time for the experiment around 2014. 

\end{itemize}




\newpage\noindent
\begin{thebibliography}{}
\newcommand {\etal} {{\it et al.}}
\newcommand {\PRL}  {Phys. Rev. Lett.}

\bibitem{rosenbluth} M.N. Rosenbluth, Phys. Rev. {\bf 79}  615 (1950).

\bibitem{jones} M.K.~Jones \etal, \PRL {\bf 84} 1398 (2000).

\bibitem{gayou} O.~Gayou \etal, \PRL  {\bf 88}  092301 (2002).

\bibitem {afan} A.V. Afanansev \etal, Phys. Rev. D {\bf 72} 013008 (2005).


\bibitem{Miller}
G. Miller, Phys. Rev. Lett. {\bf 99}  112001 (2007).

\bibitem{qf1}
D. Kaplan and A. V. Manohar, Nucl. Phys. B {\bf 310}, 527 (1988).

\bibitem{qf2}
R. D. McKeown, Phys. Lett. B {\bf 219}, 140 (1989).

\bibitem{qf3}
D.H. Beck, Phys. Rev. D {\bf 39}, 3248 (1989).

\bibitem{GPD1}
D. M\"uller, D.~Robaschik, B.~Geyer, F.M.~Dittes, J.~Horejsi,
Fortsch.\ Phys.\  42 (1994) 101.

\bibitem{GPD2}
X. Ji, Phys. Rev. Lett. {\bf 78}, 610 (1997); 
Phys. Rev. D {\bf 55}, 7114 (1997).

\bibitem{GPD3}
A.V. Radyushkin, Phys. Lett. B {\bf 380}, 417 (1996); Phys. Lett. B {\bf 385}, 333 (1996); 
Phys. Rev. D {\bf 56}, 5524 (1997).


\bibitem{pol1}
T. Eden {\it et al.}, Phys. Rev. C {\bf 50}, R1749 (1994).

\bibitem{pol2}
M. Ostrick {\it et al.}, Phys. Rev. Lett. {\bf 83}, 276 (1999).

\bibitem{pol3}
C. Herberg {\it et al.}, Eur. Phys. Jour. A {\bf 5}, 131 (1999).

\bibitem{pol4}
Jefferson Lab experiment E93-038, spokespersons: R. Madey, S. Kowalski.

\bibitem{pol5}
I. Passchier {\it et al.}, Phys. Rev. Lett. {\bf 82},  4988 (1999).

\bibitem{pol6}
H. Zhu {\it et al.}, Phys. Rev. Lett. {\bf 87}, 081801, 2001.

\bibitem{pol7}
M. Meyerhoff {\it et al.}, Phys. Lett. B {\bf 327}, 201 (1994).

\bibitem{pol_gmn1}H.~Gao {\it et al.}, Phys. Rev. C {\bf 50}, R546 (1994);
H.~Gao, Nucl. Phys. A {\bf 631}, 170c (1998).

\bibitem{pol_gmn2}
W. Xu {\it et al.}, Phys. Rev. Lett. {\bf 85}, 2900 (2000).

\bibitem{pol_gmn3}
W. Xu {\it et al.}, Phys. Rev. C. {\bf 67}, R012201 (2003). 


\bibitem{sub1}
E.B. Hughes {\it et al.}, Phys. Rev. {\bf 139},  B458 (1965); {\it ibid.} 
{\bf 146}, 973 (1966).

\bibitem{sub2}
B. Grosset\^{e}te, S. Jullian, and P. Lehmann, Phys. Rev. {\bf 141}, 
1435 (1966). 

\bibitem{sub3}
A.S. Esaulov {\it et al.}, Sov. J. Nucl. Phys. {\bf 45}, 258 (1987).

\bibitem{sub4}
R.G. Arnold {\it et al.}, Phys. Rev. Lett. {\bf 61}, 806 (1988).

\bibitem{SLAC_Rock}
S. Rock {\it et. al.}, Phys. Rev. D {\bf 46} 24 (1992).

\bibitem{sub5}
A. Lung {\it et al.}, Phys. Rev. Lett. {\bf 70}, 718 (1993).



\bibitem{tag1}
R.J. Budnitz {\it et al.}, Phys. Rev. {\bf 173}, 1357 (1968).

\bibitem{tag2}
K.M. Hanson {\it et al.}, Phys. Rev. D {\bf 8}, 753 (1973).



\bibitem{rat1}
P. Stein {\it et al.}, Phys. Rev. Lett. {\bf 16}, 592 (1966).

\bibitem{rat2}
W. Bartel {\it et al.}, Phys. Lett. B {\bf 30}, 285 (1969); {\it ibid.} 
{\bf 39}, 407 (1972); Nucl. Phys. B {\bf 58}, 429 (1973).

\bibitem{rat3} P.~Markowitz {\it et al.}, Phys. Rev. C {\bf 48}, R5 (1993).

\bibitem{rat4}H.~Anklin {\it et al.}, Phys. Lett. B {\bf 336}, 313 (1994).

\bibitem{rat5}E.E.W.~Bruins {\it et al.}, Phys. Rev. Lett. {\bf 75}, 21 
(1995).

\bibitem{rat6} H.~Anklin {\it et al.}, Phys. Lett. {\bf B428}, 248 (1998).

\bibitem{rat7} 
G. Kubon {\it et al.}, Phys. Lett. {\bf B524} 26 (2002).

\bibitem{Will}
J. Lachniet et al, nucl-ex 0811.1716, submitted to Phys. Rev. Lett.\\
JLab Experiment E94-071\\
``The Neutron Magnetic Form Factor from Precision Measurements
of the Ratio of Quasielastic Electron-Neutron to
Electron-Proton Scattering in Deuterium'', W. Brooks and M. Vineyard
{spokespersons}.

\bibitem{Durand}
L. Durand, Phys. Rev. {\bf 115}  1020 (1959).

\bibitem{Jeff}
J, Lachniet thesis, Carnegie Mellon University, unpublished, June,
2005\\
http://www-meg.phys.cmu.edu/~bquinn/jeff\_thesis.pdf


\bibitem{new_GEN}
``Measurement of the Neutron Electromagnetic Form Factor Ratio GEn/GMn at
High Q2'' proposal to PAC34, S. Riordan, G. Cates and 
B.Wojtsekhowski (spokespersons).



\bibitem{Arenhovel}
H. Arenh\"ovel, private communication.

\bibitem{Jesch}
S. Jeschonnek and J.W. Van Orden, Phys. Rev. C, {\bf 62} 044613, 2000.

\bibitem{GEn-proposal}
JLab experiment E02-013,
G.Cates, N.Liyanage and B.Wojtsekhowski (spokespersons).


\bibitem{Seamus}
S.Riordan, private communication.

\bibitem{bb_simulation}
See proposal of JLab Experiment E12-06-122
"Measurement of neutron asymmetry $A_1^n$ in the valence quark
region using BigBite spectrometer."
G.Cates, N.Liyanage , Z.Meziani, G.Rosner, X.Zheng and B.Wojtsekhowski
(spokespersons).


\bibitem{Rob_F}
Rob Feuerbach, private communication.

\bibitem{BH_simulation}
S.Abrahamyan, 
GEANT-4 simulation with saturation effect in light output.

\bibitem{matthews}
J.L. Matthews and R.O. Owens, Nucl. Instr. and Meth. {\bf 111}, 157 (1973).

\bibitem{Lomon}
E.L. Lomon and H. Feshbach, Annals of Phys. {\bf 48} 94 (1968).

\bibitem{E12-07-108}
E12-07-108 proposal.

\bibitem{PDG}
Data from PDG, http://pdg.lbl.gov/xsect/contents.html and references therein.

\bibitem{genev}
M.Ripani and E.M.Golovach based on 
P.Corvisiero, {\it et al.}, Nucl. Instr. and Meth. A, {\bf 3464} 33 (1994).


\bibitem{SLAC_Stuart}
L.M. Stuart, {\it et. al.}, Phys. Rev. D {\bf 58} 032003 (1998).

\bibitem{E94-104}
L.Y. Zhu, {\it et. al.}, Phys. Rev. C {\bf 71} 044603 (2005).


\bibitem{Anderson} 
R.L. Anderson, {\it et al}, Phys. Rev. D {\bf 14}  679 (1976).

\bibitem{FPP}
JLab E12-07-109, 
 Ch.~Perdrisat {\it et al}, Large Acceptance Proton
Form Factor Ratio Measurements at 12 and 15 (GeV/$c$)$^2$
Using Recoil Polarization Method.

\bibitem{Jerry}
G. Gilfoyle, private communication.

% \bibitem[label]{cite_key}
% literature citation ...
% ....
\end{thebibliography}






\end{document}




