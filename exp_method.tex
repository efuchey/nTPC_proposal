\section{Technique}
%
The neutron form factors are challenging to be determind experimentally especially because there is no free neutron target. 
However, since the deuterium is a loosely coupled system, it can be viewed as the sum of a proton target and a neutron target. 
In fact, quasi-elastic scattering from deuterium has been used to extract the neutron magnetic form factor, \gmn, at modestly high \qsq~for decades
~\cite{Hughes:1965zza, Arnold:1988us} in the single arm (e,e') experiments. 
However, the proton cross section needs to be subtracted by applying a single-arm quasi-elastic electron-proton scattering. 
This ``proton-subtraction" technique suffers from a number systematic uncertainties e.g. contributions from inelastic and secondary scattering processes. 

Many year ago, L.~Durand~\cite{Durand:1959zz} proposed the so-called ``ratio-method" based on the measurement of both D(e,e'n) and D(e,e'p) reactions. 
In this method, many of the systematic errors are cancel out. 
Several experiments \cite{Bruins:1995ns, Kubon:2001rj, Lachniet:2008qf} have applied the ratio-method to determine the neutron magnetic form factor. 
We propose to use this method to measure Rosenbluth slope and extract (in OPE approximation) the neutron electric form factor, \gen.

Data will be collected for quasi-elastic electron scattering from deuteron in process $D(e,e'n)p$. 
A complementary $D(e,e'p)n$ data will be taken to calibrate the experiment apparatus.
The current knowledge of the $e-p$ elastic scattering cross section (obtained in the single arm H(e,e')p and H(e,p)e' experiments) will be also used
for precision determination the experiment kinematics.

Applying Rosenbluth technique to measure \gen~requires accurate measurement of the cross section  and suffers from large uncertainties. 
To overcome this issue, we propose to extract the value of \gen~from the measured the ratio of quasi-elastic yields, $R_{n/p}$, in scattering from a deuteron target as follows: 

\begin{equation}
R_{n/p} \equiv R_{observed} = \frac{N_{e,e'n}}{N_{e,e'p}}
\label{eq:1}
\end{equation}
$R_{observed}$ needs to be corrected to extract the ratio of e-n/e-p scattering from nucleons:

\begin{equation}
R_{corrected} = f_{corr} \times R_{observed} \;\; ,
\label{eq:2}
\end{equation}
where the correction factor $f_{correction}$ takes into account the variation in the hadron efficiencies due to changes of $e-N$ Jacobian, the radiative corrections, and absorption in path
from the target to the detector, and small re-scattering correction.

In one-photon approximation, $R_{corrected}$ can be presented as: 

\begin{equation}
R_{corrected} = \frac {\sigma_{_{_{Mott}}}^n \cdot (1+\tau_p)}{\sigma_{_{_{Mott}}}^p \cdot (1+\tau_n)} \times \frac{\epsilon \sigma_{_L}^n + \sigma_{_T}^n}{\epsilon \sigma_{_L}^p + \sigma_{_T}^p}
\end{equation}

It is important that the ratio $R_M = \frac {\sigma_{_{_{Mott}}}^n \cdot (1+\tau_p)}{\sigma_{_{_{Mott}}}^p \cdot (1+\tau_n)}$ could be determine with very high accuracy even
with modest precision for the beam energy, electron scattering angle, and detector solid angle.


Now, let us write the $R_{corrected}$ at two values of $\epsilon$ using $R_s^{n(p)} = \sigma_{_L}^{n(p)}/ \sigma_{_T}^{n(p)}$ as

\begin{equation*}
R_{{corrected},\epsilon_1} = R_{M,\epsilon_1} \times \frac{\epsilon_1 \sigma_{_L}^n + \sigma_{_T}^n}{\epsilon_1 \sigma_{_L}^p + \sigma_{_T}^p}
\end{equation*}
\begin{equation*}
R_{{corrected},\epsilon_2} = R_{M,\epsilon_2} \times \frac{\epsilon_2 \sigma_{_L}^n + \sigma_{_T}^n}{\epsilon_2 \sigma_{_L}^p + \sigma_{_T}^p}
\end{equation*}

In this two equation there are two unknown variables: $\sigma_{_L}^n$ and $\sigma_{_T}^n$.

The dominant contribution to the uncertainty of $R_s^{n} = \sigma_{_L}^{n}/ \sigma_{_T}^{n}$ is coming the uncertainty in $R_s^{p}$ which is at \qsq=4.5 \gevcsq~is of 0.05 
according to the global analysis~\cite{Christy2020ab}.

This resulting in the equation with just one unknown variable $R_s^n$:

\begin{equation*}
A = B \times \frac{\epsilon_1 R_s^n + 1}{\epsilon_2 R_s^n  + 1} \approx B \times (1 + \Delta \epsilon \cdot R_s^n)
\end{equation*}

Where the variable $A = R_{{corrected},\epsilon_1}/R_{{corrected},\epsilon_2}$ will be measured with precision of 0.1\% and the variable 
\mbox{$ B = {R_{M,\epsilon_1}}/{R_{M,\epsilon_2}} \times (\epsilon_1 \sigma_{_L}^p + \sigma_{_T}^p)/(\epsilon_2 \sigma_{_L}^p + \sigma_{_T}^p) \approx (1 + \Delta \epsilon \cdot R_s^p) $}
which will be determined to precision $0.25 \times 0.05$.
The final uncertainty of $R_s^n$ is 0.05 which dominated by current knowledge of the same proton parameter 1.91*(sqrt(0.057*2) -sqrt(0.057*2+0.05)).
 
At \qsq=4.5 \gevcsq~the combination $\mu_n$\gen/\gmn~has the value of 0.5 (see the review~\cite{Punjabi:2015bba}), 
so the accuracy of our experiment (assuming the TPE portion of $R_s^n$ as large as for the proton) for will be $\delta (\mu_n \sqrt{ \tau \sigma_{_L}/\sigma{_{_T}}}$) = $\pm$0.08. 


