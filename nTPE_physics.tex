% V3.8        do not change or remove this line
%
\section{Physics Motivation}

\indent
The nucleon plays the same central role in hadronic physics that the hydrogen atom does in atomic physics and the deuteron in the physics of nuclei. 
The structure of the nucleon and its general properties, such as (static) charge, magnetic moment, size, mass, and the form factors and structure functions,
are of fundamental scientific interest.  
The nucleon is a laboratory for the study of the quark-gluon interaction and both nucleons, the proton and the neutron, need to be explored.  
At present the proton has been more thoroughly studied at large \qsq~than the neutron.
More data on the neutron is essential if we are to make real progress in obtaining a complete description of the quark structure of the nucleon~\cite{Cates:2011pz}.

Considerable information on the structure of the nucleon has been obtained by using electromagnetic probes via electron scattering.
Inclusive deep inelastic scattering (DIS) has been a classical tool with which the partonic structure of the nucleon has been probed.
At high \qsq, DIS yields information on the light-cone momentum space distributions of quarks and gluons in the nucleon when viewed
through the infinite momentum frame.
Many of the experimental foundations of QCD are in fact derived from investigations of various aspects of DIS.

Exclusive processes, on the other hand, such as elastic electron and photon scattering, can provide information on the spatial distribution
of the nucleon's constituents, which is parameterized through the elastic nucleon form factors.
%
Experimental studies of elastic electron scattering from both the proton and the neutron were initiated at SLAC and are now being thoroughly
performed at Jefferson Lab and other facilities world-wide.

The Dirac form factor, $F_1$, describes the distribution of the electric charge, while the helicity non-conserving Pauli
form factor, $F_2$, describes the distribution of the magnetic moment; these two form factors are the ingredients of the hadronic current.  
These currents contain information on the transverse charge distribution for an unpolarized and transversely polarized nucleon, respectively, 
in the infinite momentum frame~\cite{Miller:2007uy, Carlson:2007xd}.

The Sachs form factors, $G_E$ and $G_M$, the ratio of which will be extracted directly from our data for the neutron, are related to $F_1$ and $F_2$ by
%
\begin{equation}
F_{1} = \frac{G_{E} + \tau G_{M}}{1+\tau} \mbox{  and  }
F_{2} = \frac{G_{M} - G_{E}}{\kappa (1+\tau)},
\label{eq:f1f2}
\end{equation}
%
where $\kappa$ is the nucleon anomalous magnetic moment.

At asymptotically high \qsq, one can apply perturbative QCD (pQCD) to describe the \qsq~dependence of exclusive electron scattering.
Early attempts to determine the scaling behavior for $F_1$ were performed  by using a simple dimensional counting
rule justified by the inclusion of two gluon exchange processes~\cite{brodsky,brodlep}.

A recent calculation by Belitski \etal~\cite{belitsky} was performed where quark orbital angular momentum was included to determine 
the behavior for the non-helicity conserving form factor $F_2$.  
It was found to logarithmic accuracy that the ratio $F_2/F_1$ should, at high \qsq, follow the form

\begin{equation}
\frac{F_2}{F_1} \propto \frac{ \log^2\left(Q^2/\Lambda^2\right) }{Q^2},
\end{equation} 

This behavior was found to set in surprisingly early for the proton data for \qsq $> 2.0~\mathrm{GeV}^2$ with $\Lambda \approx 300~\mathrm{MeV}$.  
Using preliminary \gen~data from E02-013 up to $3.5~\mathrm{GeV}^2$, this scaling had not yet been observed~\cite{riordan} suggesting pQCD has not yet set in at this range in $Q^2$.  
A calculation from ANL utilizing a Poincare invariant truncated Faddeev equation for a quark-diquark system~\cite{bhagwat} suggests this type of behavior for each of the two nucleons may be expected.  
For high \qsq, this experiment in conjunction with high \qsq~\gmn~data may be able to observe the onset of this pQCD behavior in the neutron form factors.
%

Over the years many QCD-inspired models have been developed to describe nucleon electromagnetic form factors at small and intermediate $Q^2$
values ($Q^2 < 1$--2~$\mathrm{GeV}^2$).
While these have provided some insights into the possible origin of the nonperturbative quark structure of the nucleon, ultimately one would like
to use experimental form factor data to test the workings of QCD itself.
%

Recently, important developments in QCD phenomenology has been the exploration of the generalized parton distribution (GPD)
formalism \cite{Ji97,rad96,rad97}, which provides relations between inclusive and exclusive observables.
The nucleon elastic form factors $F_1$ and $F_2$ are given by the first moments of the GPDs
%
\begin{equation}
F_1(t) = \sum_q \int^1_0 H^q (x,\xi,t,\mu) dx\ 
 \mbox{  and\  }
F_2(t) = \sum_q \int^1_0 E^q (x,\xi,t,\mu) dx,
\label{eq:F1/2}
\end{equation}
%
where $H^q$ and $E^q$ are two of the generalized parton distributions, $x$ is the standard Bjorken $x$, $\xi$ is is the ``skewdness'' of the reaction (Fig.~\ref{fig:dvcs}), $t$ is the four-momentum transferred by the electron, $\mu$ is a scale parameter necessary from the evolution over $Q^2$, analogous to DIS parton distributions, and the sum is over all quarks and anti-quarks.  
These may be accessed through processes such as deeply virtual compton scattering, where the interaction is factorized into a hard part with the virtual photon/photon interactions with an individual quark and a soft part of the residual system where the GPD information is contained, Fig.~\ref{fig:dvcs}.%

Furthermore, as shown earlier by Ji \cite{Ji97}, the moments of GPDs can yield information, according to the Angular Momentum Sum Rule, 
on the contribution to the nucleon spin from quarks and gluons, including both the quark spin and orbital angular momentum.

At present, experimental measurements of GPDs are scarce.  
Until such data becomes available, work has been done to attempt to parameterize these GPDs,  which rely heavily on data from electromagnetic form factors and parton distributions from DIS as constraints~\cite{dfjk04, kroll06, guidal08}.  
Data at high \qsq~for \gen~would contribute significantly in the development of these models.

The isovector and isoscalar form factors constructed from the proton and neutron form factors have different sensitivity to higher Fock components of the light cone quark wave function.
This difference can be an important handle to test the valence quark dominance in exclusive reactions in the few \gevcsq~range.
%
Data on $F_{1p}$ and $F_{1n}$ will allow the extraction of information related to the {\it (u-d)} distribution, which was
calculated recently using the GPD approach by K.~Goeke, M.~Polyakov, and M.~Vanderhaeghen \cite{polyakov}.

Recent theoretical developments also indicate that measurements of the individual elastic form factors of the nucleon up to high $Q^2$ may shed light on the problem of nucleon spin \cite{ralston}.
%

As an incidental benefit of the proposed experiment, a better determination of the neutron electric form factor will be very important for calculations of nuclear form factors, such as those of the deuteron.
Even though \gen $\ll$ \gep at $Q^2 \approx 0$, at larger \qsq ($Q^2 \sim 3~\mathrm{GeV}^2$) the ratio \gen/\gep can be as large as
$\approx 0.4$, so that accurate information on \gen at large \qsq is essential for a reliable description of the deuteron form factors.


%%%%%%%%%%%%%%%%%%%%%%%%%%%%%%%%%%%%%%%%%%%%%%%%%%%%%%%%%%%%%%%%%%%%%%%%%%%%%%%%%%%%%%%%%%%%%%%%
