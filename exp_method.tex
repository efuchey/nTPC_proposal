\section{Technique}
%
The neutron form factors are challenging to be determine experimentally especially because there is no free neutron target. 
However, since the deuterium is a loosely coupled system, it can be viewed as the sum of a proton target and a neutron target. 
In fact, quasi-elastic scattering from deuterium has been used to extract the neutron magnetic form factor, \gmn, at modestly high \qsq~for decades~\cite{Hughes:1965zza, Arnold:1988us} in the single arm (e,e') experiments. 
However, the proton cross section needs to be subtracted by applying a single-arm quasi-elastic electron-proton scattering. 
This ``proton-subtraction" technique suffers from a number systematic uncertainties e.g. contributions from inelastic and secondary scattering processes. 

Many year ago, L.~Durand~\cite{Durand:1959zz} proposed the so-called ``ratio-method" based on the measurement of both D(e,e'n) and D(e,e'p) reactions. 
In this method, many of the systematic errors are cancel out. 
Several experiments \cite{Bruins:1995ns, Kubon:2001rj, Lachniet:2008qf} have applied the ratio-method to determine the neutron magnetic form factor. 
We propose to use this method to measure Rosenbluth slope and extract (in OPE approximation) the neutron electric form factor, \gen.

Data will be collected for quasi-elastic electron scattering from deuteron in process $D(e,e'n)p$. 
A complementary $D(e,e'p)n$ data will be taken to calibrate the experiment apparatus.
The current knowledge of the $e-p$ elastic scattering cross section (obtained in the single arm H(e,e')p and H(e,p)e' experiments) will be also used
for precision determination the experiment kinematics.

Applying Rosenbluth technique to measure \gen~requires accurate measurement of the cross section  and suffers from large uncertainties. 
To overcome this issue, we propose to extract the value of \gen~from the measured the ratio of quasi-elastic yields, $R_{n/p}$, in scattering from a deuteron target as follows: 

\begin{equation}
R_{n/p} \equiv R_{observed} = \frac{N_{e,e'n}}{N_{e,e'p}}
\label{eq:1}
\end{equation}
$R_{observed}$ needs to be corrected to extract the ratio of e-n/e-p scattering from nucleons:

\begin{equation}
R_{corrected} = f_{corr} \times R_{observed} \;\; ,
\label{eq:2}
\end{equation}
where the correction factor $f_{correction}$ takes into account the variation in the hadron efficiencies due to changes of $e-N$ Jacobian, the radiative corrections, and absorption in path
from the target to the detector, and small re-scattering correction.

In one-photon approximation, $R_{corrected}$ can be presented as: 

\begin{equation}
R_{corrected} = \frac {\sigma_{_{_{Mott}}}^n \cdot (1+\tau_p)}{\sigma_{_{_{Mott}}}^p \cdot (1+\tau_n)} \times \frac{\epsilon \sigma_{_L}^n + \sigma_{_T}^n}{\epsilon \sigma_{_L}^p + \sigma_{_T}^p}
\end{equation}

It is important that the ratio $R_{Mott} = \frac {\sigma_{_{_{Mott}}}^n \cdot (1+\tau_p)}{\sigma_{_{_{Mott}}}^p \cdot (1+\tau_n)}$ could be determine with very high relative accuracy even with modest precision for the beam energy, electron scattering angle, and detector solid angle. 
Now, let us write the $R_{corrected}$ at two values of $\epsilon$ using $R_c^{n(p)} = \sigma_{_L}^{n(p)}/ \sigma_{_T}^{n(p)}$ as:
\begin{equation*}
R_{{corrected},\epsilon_1} = R_{Mott,\epsilon_1} \times \frac{\epsilon_1 \sigma_{_L}^n + \sigma_{_T}^n}{\epsilon_1 \sigma_{_L}^p + \sigma_{_T}^p}
\hskip .5 in
R_{{corrected},\epsilon_2} = R_{Mott,\epsilon_2} \times \frac{\epsilon_2 \sigma_{_L}^n + \sigma_{_T}^n}{\epsilon_2 \sigma_{_L}^p + \sigma_{_T}^p}
\end{equation*}

In these two equations there are two unknown variables: $\sigma_{_L}^n$ and $\sigma_{_T}^n$.
The dominant contribution to the uncertainty of the slope of the cross section vs. $\epsilon$,  
$S_c^n = \sigma_{_L}^n/ \sigma_{_T}^n$, will come from the uncertainty of $S_c^p$.
At \qsq=4.5 \gevcsq, according to the global analysis of $e-p$ cross section~\cite{Christy2020ab}, the value of $S_c^p$ is close to $1/(\tau \, \mu_p^2) = 0.107$ with uncertainty of 0.01.
The resulting equation for $S_c^n$ is:
\begin{equation*}
A = B \times \frac{1 + \epsilon_1 S_c^n}{1 + \epsilon_2 S_c^n} \approx B \times (1 +  \Delta \epsilon \cdot S_c^n),
\end{equation*}

where the variable $A = R_{{corrected},\epsilon_1}/R_{{corrected},\epsilon_2}$ will be measured with relative precision of 0.1\%.  
Assuming, for this estimate, equal values of \qsq~for two kinematics, the  $\tau$ and $\sigma_{_T}$ for two kinematics are canceled out, and the variable
\mbox{$ B = {R_{M,\epsilon_1}}/{R_{M,\epsilon_2}} \times (1+ \epsilon_2 \, S_c^p)/( 1 + \epsilon_1 \, S_c^p)$}.
For actual small range of $\epsilon$ and small value of the slope, the $B \approx (1 - \Delta \epsilon \cdot S_c^p)$.
The value of B will be determined from global proton $e-p$ data to a precision of $0.25 \times 0.01$.
 
At \qsq=4.5 \gevcsq~the ratio $\mu_n$\gen/\gmn~is of $0.55 \pm 0.05$, see the review~\cite{Punjabi:2015bba}.
%
In a simplest model, the slope $S_c^n$ is a sum of the slope due to \gen/\gmn~and the TPE contribution.
If we use for TPE the prediction~\cite{Blunden:2005ew}, shown in Fig.~2, the TPE leads to increase of $S_c^n$ by a factor of 2,
so the result of this experiment for TPE will be $0.069 \pm 0.012 \pm 0.01$, where the first uncertainty is due to accuracy 
of \gen/\gmn~and the second one due to projected precision of this experiment. It would be a 4-4.5 sigma observation of the neutron TPE.
