\section{Introduction}

\label{sec:sec1}

In 1950's, a series of experiments performed by R.~Hofstadter~\cite{Hofstadter:1956qs} revealed that the nucleons have a substructure 
(would be called later the quarks and gluons).
The experiment confirmed M.~Rosenbluth's theory~\cite{Rosenbluth:1950yq} based on one-photon exchange approximation.
In the Born approximation, where the interaction between the electron and the nucleon occurs $via$ an exchange of a one virtual photon (OPE), 
the unpolarized $e-N$ elastic cross section can be expressed in terms of a nucleon magnetic, \gmf, and electric, \gmf, form factors. 
These form factors describe the deviation from a point-like scattering cross section:  

\begin{equation}\label{eq:Ros}
\bigg(\frac{d\sigma}{ d\Omega}\bigg)_{eN\rightarrow eN}  = \frac{\sigma_{_{Mott}}}{\epsilon(1+\tau)}\;\; 
\big[ \tau \cdot G^{2}_{_M}(Q^2) + \epsilon \cdot G^{2}_{_E}(Q^2)\big] \;=\; \sigma_{_T} + \epsilon \cdot \sigma_{_L},
\end{equation}
\vskip .25 in
 where $E$ and $E'$ are the incident and scattered electron energies, respectively, $\theta$ is the electron scattering angle, 
$\tau \equiv -q^{2}/4M^{2}$,  with $-q^2 \equiv Q^2 = 4EE'\sin{(\theta/2)}$ being the negative four momentum transfer square, 
$M$ is the nucleon mass, and $\epsilon = \big[ 1 + 2(1+\tau) \tan^2{(\theta/2)} \big]^{-1}$ is the longitudinal polarization 
of the virtual photon, $\sigma_{_L}$ and $\sigma_{_T}$ are the cross sections for longitudinally and transversely polarized virtual photons, respectively.\\

\begin{figure}[h]
\includegraphics[trim = 0mm 0mm 0mm 0mm, width = 0.75\textwidth]{Plots/Fig1-b.png}
\caption{The square root of Rosenbluth slope, corrected for kinematical factor $\sqrt {\tau}$ and $\mu_p$, observed in elastic electron-proton scattering,
adopted from Ref.~\cite{Christy2020ab}.}
\label{pic:Fig1}
\end{figure}

The linear $\epsilon$ dependence of the cross section is due to $\sigma_{_L}$ term, see Eq.~\ref{eq:Ros}.
The ratio $\sigma_{_L}/\sigma_{_T}$ is a Rosenbluth slope related to \gef/\gmf (in OPE), see Fig.~\ref{pic:Fig1}.
The data show that at \qsq~of 4-5 \gevcsq~the Rosenbluth slope is three-four times larger than it suppose to be (in OPE) for
the observed values of the \gep/\gmp~ratio.

%
The nucleon electromagnetic form factors can reveal a lot of information about the nucleon internal structure, as well as the quark distribution. 
The form factors depend only on one variable the negative square of the four-momentum transfer carried by the photon, \qsq. 
In the limit of large \qsq, pQCD provides well-motivated predictions for the \qsq-dependance of the form factors and their ratio. 
However, it was never predicted at what \qsq~range the pQCD prediction (scaling) will be valid.
Studies of GPDs show that pQCD validity will require a very large \qsq~of 100~\gevcsq. 
It was discovered at JLab, using the double polarization methods, that the proton electric and magnetic form factors behave differently starting at \qsq~ $\approx$ 1~\gevcsq.
 
\begin{figure}[th]
\includegraphics[width = 0.75\textwidth]{Plots/nTPE-BMT.pdf}
\caption{Projected impact of TPE on \gen/\gmn~using LT separation, according to Ref.~\cite{Blunden:2005ew}.}
\label{pic:Fig2}
\end{figure}
 
Experimentally, the nucleon form factors can be measured using one of two techniques: polarization transfer technique and Rosenbluth technique. 
The polarization method examines the polarization transfer from longitudinally polarized electron to the recoiling nucleon and 
determine the resulting azimuthal asymmetry distribution using a polarimeter. 
Alternatively, one can use the polarized  electron beam and a polarized target. 
While in the Rosenbluth method, the electric and magnetic from factors can be separated by making two or more measurements with 
different $\epsilon$ values ($i.e.$ different beam energies and angles), but with same \qsq~value. 
Rosenbluth technique requires an accurate measurement of the cross section and suffers from large systematic uncertainties arising from several factors. 
For instance, an accurate knowledge of the neutron detector efficiency is required.

When comparing the values of \gep/\gmp~obtained from both techniques, a significant discrepancy was observed (see Fig.~\ref{pic:Fig1}). 
Such discrepancy implies a potential problem in our understanding of the nucleon substructure. 
Many efforts were made in order to provide legitimate explanation, and it is believed that the inconsistency is due to contribution of two-photon exchange
in $e-N$ elastic scattering process, see Refs.~\cite{Arrington:2011dn, Afanasev:2017gsk}.
Predictions made for the neutron case are shown in Fig.~\ref{pic:Fig2}, adopted from~\cite{Blunden:2005ew}.
The contribution of TPE could reach about 30\% of Rosenbluth slope value at 5 \gevcsq.

In the following we propose to make precision L/T separation of the elastic electron-neutron cross section and first experimental assessment 
of the two-photon exchange contribution on the neutron magnetic form factor measurements (see also Ref.~\cite{Wojtsekhowski:2017kti}).
The result of the nTPE experiment will likely add a new component to our understanding of the elastic electron-nucleon process.

