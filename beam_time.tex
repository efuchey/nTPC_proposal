%v1.9
\section{Beam time request}

{\bf We request 48 hours total time (32 hours of beam-on target)} to measure the two-photon effect (and \gen~in one-photon approximation) 
at \qsq~= 4.5 \gevcsq~through a measurement of the cross sections of the reaction D(e,e'N) at a large value of the virtual photon polarization $\epsilon$=0.84.

We plan to take 12 hours of data at a full luminosity of $2.86~\times~10^{38}~\mathrm{cm}^{-2}\mathrm{s}^{-1}$, which corresponds to a beam intensity of $I_{exp} =~30~\mu$A on a liquid deuterium target with length $l_{tgt}~=~15$~cm and density $d_{tgt}~=~0.169~\mathrm{g.cm}^{-3}$. 
To have a better handle on our backgrounds, we also plan to take 12 hours of data at half luminosity (basically by lowering the beam intensity by a factor 2).
In each of these configurations, we also need to take data on a ``dummy'' target ({\it i.e.} on a target cell identical to the one used for production, but empty) to understand the contamination of our data from the target walls. 
The projected use of this time is summarized in Table.~\ref{tab:beamtime}.
%
\begin{center}
\begin{table}[h]
\begin{tabular}{|l|c|c|c|}
\hline
Task & Target & $I_{exp}$ & time requested \\
\hline
Data taking (Prod.) & 15~cm LD$_2$ & $30~\mu$A & 12 hours \\ 
\hline
Data taking (Syst.) & 15~cm ``Dummy'' & $30~\mu$A & 4 hours \\ 
\hline
Data taking (Prod.) & 15~cm LD$_2$ & $15~\mu$A & 12 hours \\ 
\hline
Data taking (Syst.) & 15~cm ``Dummy'' & $15~\mu$A & 4 hours \\ 
\hline
Setting change & - & - & 16 hours \\
\hline
\hline
{\bf Total} & \multicolumn{3}{r|}{\bf 48 hours} \\
\hline
\end{tabular} 
\caption{Summary table for the beam time request.}%, decomposed for the different tasks that need to be accomplished for this experiment
\label{tab:beamtime}
\end{table}
\end{center}
%
This experiment will take place in Hall A, utilizing the BigBite spectrometer to detect electrons scattered off 
the liquid deuterium target, and HCal calorimeter to detect the recoiling neutron and proton.

Data taking (if approved by PAC48) will take place in summer 2021 during the approved and scheduled run of the GMn, E12-09-019, experiment,
which is going to measure the $e-n$ elastic scattering cross section at \qsq~= 4.5 \gevcsq~at $\epsilon$=0.60.

The set of instrumentation and required beam current for proposed measurement is identical to one in the GMn experiment.
The beam energy of 6.6 GeV will be used.
One of two data points required for the cross section LT separation is already in the data taking plan of GMn.

There are no other measurements of TPE in the $e-n$ elastic scattering and knowledge of the TPE is essential for the understanding 
of the elastic electron scattering from neutron (and proton) and hadron structure.  
Furthermore, it is a necessary input in the analysis and interpretation of a wide range of electron scattering processes. 

The kinematics of our measurements emphasize the same \qsq~range where TPE in $e-p$ elastic scattering was observed to dominate in Rosenbluth slope.
Measuring at this high momentum transfers will provide unique input for testing TPE calculations~\cite{Blunden:2005ew}.

We propose to measure the Rosenbluth slope and extract (in one-photon approximation) $\delta$\gen/\gmn~to an accuracy of 0.15, which would bring its precision to a level comparable with that of the double polarization experiments GEN-RP and GEN-He3 at such value of \qsq.
Such precision should be sufficient to detect the TPE contribution to the $e-n$ Rosenbluth slope on the three sigma level.
