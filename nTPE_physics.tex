% June 10 9 am        do not change or remove this line
%
\section{Physics Motivation}
\label{sec:sec2}
%\iffalse
\subsection{Form factor measurements at high \qsq}
\indent
The nucleon plays the same central role in hadronic physics that the hydrogen atom does in atomic physics and the deuteron in the physics of nuclei.
The structure of the nucleon and its specific properties, such as charge, magnetic moment, size, mass; the elastic electron scattering form factors, resonances; and structure functions in DIS, are of fundamental scientific interest.
The isospin is a fundamental property of the nucleon, so both the proton and neutron investigations are important to do.
By using data on the proton and neutron form factors the flavour structure could be explored~\cite{Cates:2011pz}.
It has already provided the most direct evidence for a diquark correlation in the nucleon~\cite{Roberts:2007jh, Segovia:2014aza, Wojtsekhowski:2020tlo}.

Hadron structure, as seen in elastic electron scattering, in one-photon approximation, is defined by two functions of four momentum transfer square.
They are: the helicity conserving Dirac form factor, $F_1$, which describes the distribution of the electric charge, and the helicity non-conserving Pauli form factor, $F_2$, which describes the distribution of the magnetic moment.
These two form factors are the ingredients of the hadronic current.  
They contain information on the transverse charge distribution for an unpolarized and transversely polarized nucleon, respectively, 
in the infinite momentum frame~\cite{Miller:2007uy, Carlson:2007xd}.

The Sachs form factors, \gef~and \gmf, the ratio of which will be extracted directly from the data, are related to $F_1$ and $F_2$ by
%
\begin{equation}
F_{1} = \frac{G_{E} + \tau G_{M}}{1+\tau} \mbox{  and  }
F_{2} = \frac{G_{M} - G_{E}}{\kappa (1+\tau)},
\label{eq:f1f2}
\end{equation}
%
where $\kappa$ is the nucleon anomalous magnetic moment.

Already twenty-four years ago, an important development in QCD phenomenology has been the exploration of the generalized parton distribution (GPD) formalism~\cite{Mueller:1998fv, Ji:1996ek, Radyushkin:1996nd}, which provides relations between inclusive and exclusive observables.
The nucleon elastic form factors $F_1$ and $F_2$ are given by the first moments of the GPDs
%
\begin{equation}
F_1(t) = \sum_q \int^1_0 H^q (x,\xi,t,\mu) dx\ 
 \mbox{  and\  }
F_2(t) = \sum_q \int^1_0 E^q (x,\xi,t,\mu) dx,
\label{eq:F1/2}
\end{equation}
%
where $H^q$ and $E^q$ are two of the generalized parton distributions, $x$ is the standard Bjorken $x$, $\xi$ is is the ``skewdness'' of the reaction, $t$ is the four-momentum transferred by the electron, $\mu$ is a scale parameter necessary for the evolution over $Q^2$, analogous to DIS parton distributions, and the sum is over all quarks and anti-quarks.  
GPDs may be accessed through processes such as deeply virtual Compton scattering, where the interaction is factorized into a hard part with the virtual photon/photon interactions with an individual quark and a soft part of the residual system where the GPD information is contained.

A fundamental nucleon feature, the spin, is related to GPDs, as shown by X.~Ji~\cite{Ji:1996ek}. 
The moments of GPDs can yield information, according to Ji's Angular Momentum Sum Rule, 
on the contribution to the nucleon spin from quarks and gluons, including both the quark spin and orbital angular momentum.

At present, experimental measurements of GPDs are still scarce.  
Until high \qsq~DVCS data becomes available, work has been done to attempt to parameterize these GPDs,  
which rely heavily on data from electromagnetic form factors and parton distributions from DIS as constraints~\cite{Diehl:2013xca}.  
Data at high \qsq~for \gen~would contribute significantly in the development of these models.

%As we presented above the form factors are important components for GPDs development.
%However, the cross section of elastic $e-p$ scattering contains a significant contribution to $\sigma_{_L}$, 
%which at high \qsq~is much larger than theory calculations expected~\cite{Kivel2020ab}.
%Such an alarming observation underlines that understanding of TPE effect is essential for hadron physics. 
%
As we presented above, nucleon elastic form factors provide important input for the modeling of GPDs.
At the same time, the measured cross section of elastic $e-p$ scattering at high \qsq
 is significantly larger than predicted by Born-approximation calculations~\cite{Kivel2020ab}, indicating that TPE effects play
a critical role in the high-\qsq region and therefore must be well understood
before conclusions about GPDs can be drawn.\\
%\fi

\subsection{The role of two-photon exchange in form factors}

As we presented above the form factors are important components for the study of the nucleon structure.
However, the puzzle of the form factor ratio at higher \qsq~$G_E/G_M$ partly blurs our understanding of the measurements.
%which at high \qsq~is much larger than theory calculations expected~\cite{Kivel2020ab}.
Such an observation underlines the importance of the understanding of the two-photon exchange for hadron physics.

There are two different contributions of the two photon-exchange. The first one is the ``soft'' two-photon exchange, where one of the photons energy is very small compared to the other, which is usually included in radiative correction calculations such as Mo and Tsai~\cite{RevModPhys.41.205}.
The second one, which is the one we've referred to so far in this document, is the ``hard'' two-photon exchange, where both photons have a significant energy.
The leading order contribution of the two-photon exchange to the elastic lepton-nucleon scattering is the interference term between the one-photon amplitude term ${\cal{M}}_{1\gamma}$ and the two-photon amplitude term ${\cal{M}}_{2\gamma}$:
%
\begin{equation}
\sigma_{eN} \propto |{\cal{M}}_{1\gamma}|^2 \pm 2 \Re e [{\cal{M}}_{1\gamma} {\cal{M}}_{2\gamma}].
\label{eq:eN_1g_2g}
\end{equation}
%
This interference term depends on the cube of the charge of the lepton involved, {\it i.e.} at first order the sign of the two-photon exchange contribution is naively expected to flip from $e^--N$ to $e^+-N$.
This means that the respective discrepancies between the Rosenbluth slopes of $e^+-N$ and $e^+-N$ and $G_E/G_M$ from polarization transfer should be of same magnitude, but going into different directions. 
Any significant divergence between the two discrepancies would point towards an additional phenomenon contributing to the elastic electron-nucleon scattering cross section beyond two-photon exchange.
The presented measurements purport to measure the Rosenbluth slope in positron-neutron scattering compared to electron-neutron scattering.
