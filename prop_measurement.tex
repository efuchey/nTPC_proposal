\section{Proposed Measurements} 
\label{prop}

We propose to use the same experimental setup of the E12-09-019 experiment. We will add a kinematic point at \qsq~= 4.5 \gevcsq, at a higher beam pass (6.6~GeV/3~pass instead of 4.4~GeV/2~pass), leading to a higher $\epsilon$ value. This additional point along with the data point of the E12-09-019 experiment will allow us to perform the standard Rosenbluth method to obtain (in one-photon approximation) the neutron electric and magnetic form factors. In addition, the ratio method (Sec.\ref{sec:exp_method}), in which the systematic errors are greatly reduced, will be implemented to calculate  the two photon exchange (TPE) contribution. The study of the $\epsilon$ dependence of the reduced cross section will help examine the two photon exchange contribution to the neutron form factor ratio \gen/\gmn.
%Table.~\ref{tab:propkin} displays the kinematic settings of the proposed experiment.
Table.~I displays the kinematic settings of the proposed experiment. 

\begin{table}[h] 
\centering
\begin{tabular}{|c|c|c|c|c|c|c|}
\hline
\small{Point} & $Q\textsuperscript{2}$  & E & E$'$  & $\theta_{BB}$ & $\theta_{SBS}$ & $\epsilon$ \\
& (GeV/c)$^2$ & (GeV) & (GeV)  &\; degrees\; & \; degrees \;  &   \\
\hline
\textcolor{blue} 1 &\textcolor{blue} {4.5} & \textcolor{blue}{4.4} & \textcolor{blue}{2.0} & \textcolor{blue}{41.88}  & \textcolor{blue}{24.67} &\; \textcolor{blue}{0.599} \;\\
\hline
\textcolor{red}{2} & \textcolor{red}{4.5}  &  \textcolor{red}{6.6}  &  \textcolor{red}{4.2}  & \textcolor{red}{23.23}  &  \textcolor{red}{31.2}  &  \textcolor{red}{0.838} \\
\hline
\end{tabular} 
\caption{Kinematic settings of the proposed experiment. The kinematic point with the lowest $\epsilon$ value (blue row) is an existing measurement of the approved  E12-09-019 experiment.}
\label{tab:propkin}
\end{table}

