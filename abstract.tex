%version 1.0 June 1, 2020
{\hskip 2. in {\large {\bf Abstract}}} \\

We propose to make a high precision measurement of the two-photon exchange contribution (TPE) in elastic electron-neutron scattering at a four-momentum transfer \qsq~= 4.5 \gevcsq. 
While significant efforts to study the two-photon-exchange have focused around elastic electron-proton scattering, the impact of TPE on neutron form factors was never examined experimentally. 
The proposed experiment will provide the very first assessment of the two-photon exchange in electron-neutron scattering, which will be important for understanding the nucleon form factor physics. \par
The proposed experiment will be performed in Hall A using the BigBite (BB) spectrometer to detect the scattered electrons and the Super-BigBite (SBS) to detect the protons and neutrons. 
The experiment should run concurrently with the E12-09-019 $G_M^n$ and E12-17-004 $G_E^n$-Recoil experiments, which are expected to run in 2021. 
The experimental setup of the proposed experiment will be identical to that of E12-09-019 experiment. \par
The ``ratio" method will be used to extract the electric form factor of the neutron \gen~by scattering unpolarized electrons from deuterium quasi-elastically at two beam energies 4.4 and 6.6 GeV and electron scattering angles 41.9 and 23.3 degrees respectively. 
In the proposed approach, systematic errors are greatly reduced compared to those in the traditional single electron arm configuration. 
Several experiments at Mainz and JLab have used the ratio method to measure the neutron magnetic form factor in the past years. 
The method can be extended to extract the neutron electric form factor even with less stringent requirements 
on the knowledge of the absolute neutron detection efficiency and experimental kinematics.  

% We propose to add a kinematic point at $Q^2$ = 4.5 (GeV/c)$^2$ with high $\epsilon$ value, which will be used along with one of E12-09-019 kinematic points to extract the form factors ratio $g=G_E/G_M$ using Rosenbluth method. 
